One simple fitness function is exponential directional selection:

\begin{equation}
  w(z) = e^{sz}.
\end{equation}

This fitness function has a number of nice properties. Fitness is multiplicative
across loci so it does not lead to a build up of linkage disequilibrium.
Additionally, the ratio of fitnesses between two phenotypes does not depend on
the reference from which one measures them. \citet{Turelli1990} investigate
exponential directional selection. They calculate

\begin{equation}
  \bar{w} = e^{sz} \exp \left( \frac{s^2V_e}{2} \right)
  \left( 1 + \sum_{i=2}^{\infty} \frac{s^iM_{i,g}}{i!} \right),
\end{equation}

\begin{equation}
  L_1 = s, \qquad L_2\approx s^2, \qquad \text{and} \qquad L_k \approx 0 \qquad \text{for} \qquad k \geq 3.
\end{equation}

The approximations ignore terms of order $s^2$. Thanks to \eqref{eq:dz} we can
see that only the variance and skew of the breeding values will affect the response to selection in this case. 

%%% Local Variables:
%%% TeX-master: "notes.tex"
%%% End:
