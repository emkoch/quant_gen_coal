A basic question in quatitative genetics is what is the distribution
of trait values in a sample of individuals. In their investigation,
\citet{Schraiber2015} found a generating function for this
distribution undes the standard neutral coalescent model and a general
stepping stone model of mutation. We wish to extend this framework to
additional populatio models. A first step is to show a connection
between the distribution of branch lengths on a genealogy and the
distribution of trait values in the sampled individuals. This should
be useful because the generating functions under various models of
population structure have already been investigated in some depth
\citep{Lohse2011}.

Let $\mathbf{T}$ be a random vector containing the lengths of all
possible branches on a coalescent tree. For instance, if we had a
three samples $a$, $b$, and $c$, then
$\mathbf{T}=\{T_a,T_b,T_c,T_{a,b},T_{a,c},T_{b,c}\}$. Let $\mathcal{O}$
contain all possible configurations that coalescent branches can
subtend. For our example, $\mathcal{O}=\{(a),(b),(c),(a,b),(a,c),(b,c)\}$.
Let $\mathbf{Y}$ be a random vector containing the quantitative trait
values in the sampled individuals. For our example,
$\mathbf{Y}=\{Y_a,Y_b,Y_c\}$. It is important to note that these values
indicate the change in the quantitative trait since the value in the
MRCA of the sample, not their actual values.

The moment generating function of $\mathbf{Y}$ is
\begin{equation}
  \varphi_{\mathbf{Y}}(\mathbf{k}) = E\left[ e^{\mathbf{k} \cdot \mathbf{Y}} \right] =
  \int e^{\mathbf{k} \cdot \mathbf{Y}} P(\mathbf{Y}=\mathbf{y}) d\mathbf{y},
\end{equation}
where $\mathbf{k}$ is a vector of dummy variables for each
samples. Although imperfect, this is the notation I will use for
writing integrals over probability distributions. Rewriting this
expression by conditioning on the genealogy we get
\begin{align}
  \varphi_{\mathbf{Y}}(\mathbf{k}) &= \int e^{\mathbf{k} \cdot \mathbf{Y}}
  \int P(\mathbf{Y}=\mathbf{y} | \mathbf{T}=\mathbf{t}) P(\mathbf{T}=\mathbf{t})
  d\mathbf{t} d\mathbf{y}\\
  &= \int \int e^{\mathbf{k} \cdot \mathbf{Y}} P(\mathbf{Y}=\mathbf{y} | \mathbf{T}=\mathbf{t}) d\mathbf{y}
  P(\mathbf{T}=\mathbf{t})
  d\mathbf{t}
\end{align}

We can write each $Y_i$ as a sum of the change in the trait value
occuring on each branch of the genealogy.
\begin{equation}
  Y_i = \sum_{\omega \in \mathcal{O}} Y_{i,\omega}
\end{equation}
Of course, many of these branches will not subtend $i$, so the change
along them is defined to be zero. The $Y_{i,\omega}$ are correlated
because the underlying branch lengths are correlated, but they are
conditionally independent given $\mathbf{T}$. This is due to the
assumption of a stepping stone mutation model and that mutations occur
as a poisson process along the branches. We can therefore factor $\int
e^{\mathbf{k} \cdot \mathbf{Y}} P(\mathbf{Y}=\mathbf{y}
|\mathbf{T}=\mathbf{t}) d\mathbf{y}$ into independent parts. We first write
\begin{equation}
  \mathbf{k} \cdot \mathbf{y} = \sum_{\omega \in \mathcal{O}}\left( \sum_{i \in \omega} k_iy_{i,\omega}\right)
\end{equation}
and
\begin{equation}
  P(\mathbf{Y}=\mathbf{y}|\mathbf{T}=\mathbf{t}) = \prod_{\omega \in \mathcal{O}}
  P(\mathbf{Y}_{\omega}=(y_{i,\omega})_{i \in \omega} | \mathbf{T}=\mathbf{t}).
\end{equation}
This yields
\begin{equation} \label{eq:factor}
  \int e^{\mathbf{k} \cdot \mathbf{Y}} P(\mathbf{Y}=\mathbf{y} |\mathbf{T}=\mathbf{t}) d\mathbf{y} =
  \prod_{\omega \in \mathcal{O}}\int \exp\left(\sum_{i \in \omega}k_iy_{i,\omega}\right)
  P(\mathbf{Y}_{\omega}=(y_{i,\omega})_{i \in \omega} | \mathbf{T}=\mathbf{t})d(y_{i,\omega})_{i \in \omega}.
\end{equation}

To move on we require two results. The first is that the change in a
trait value along some branch of length $t_\omega$ is a compound
Poisson process. The moment generating function for a compound Poisson
process over a time $t$ is $\exp\left(\lambda t (\psi(k)-1)\right)$.
Where $\lambda$ is the rate that events happen and $\psi$ is the
moment generating function for the distribution of mutation effects.
The second results is that the moment generating function for the
distribution of two completely correlated random variables $X_1$ and
$X_2$ that have the same distribution is $\varphi_{X_1}(k_1+k_2)$. Since
the trait change along a branch $t_{\omega}$ is the same for all
samples subtended by the branch, we can write the product terms in
equation \ref{eq:factor} as
\begin{equation}
  \exp\left( \frac{\theta}{2} t_{\omega} \left( \psi\left(\sum_{i \in \omega}k_{\omega}\right) -1 \right)\right).
\end{equation}
This gives
\begin{equation}
  \varphi_{\mathbf{Y}}(\mathbf{k}) = \prod_{\omega \in \mathcal{O}}
  \int \exp\left( \frac{\theta}{2} t_{\omega} \left( \psi\left(\sum_{i \in \omega}k_{\omega}\right) -1 \right)\right)
  P(\mathbf{T}=\mathbf{t})d\mathbf{t}.
\end{equation}
This is simply the moment generating function for the genealogy
$\mathbf{T}$ with $\frac{\theta}{2} \left( \psi(\sum_{i \in
  \omega}k_{\omega}) -1 \right)$ substituted for the dummy variable of
branch $T_{\omega}$. Or,
\begin{equation}
  \varphi_{\mathbf{T}}(\mathbf{s})\Bigr|_{s_{\omega}=\frac{\theta}{2} \left( \psi\left(\sum_{i \in \omega}k_{\omega}\right) -1 \right)}
\end{equation}
