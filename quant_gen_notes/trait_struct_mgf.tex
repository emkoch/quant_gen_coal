To convert this into a moment generating function for branch lengths we would make the substitution specified by
equation \ref{eq:sub}. This gives 
\begin{align}
  \label{eq:gensub}
  \varphi_{\mathbf{Y}}^{\Omega}(\mathbf{k}) &=
  \left( \sum_{i=1}^M \binom{|\Omega_i|}{2}\eta_i  + \sum_{(i,j:i \neq j)} m_{i,j}|\Omega_i| -
  \sum_{i=1}^M \sum_{\omega \in \Omega_i} \T\left( \psi\left(\sum_{a \in \omega}k_{a}\right) -1\right)\right)^{-1} \\
  &\times \left( \sum_{i=1}^M \eta_i \sum_{(a,b) \in \Omega_i:a \neq b}
  \varphi_{\mathbf{Y}}^{\Omega(i:a \cup b)}(\mathbf{k}) + 
  \sum_{(i,j):i\neq j}m_{i,j}\sum_{\omega \in \Omega_i} \varphi_{\mathbf{Y}}^{\Omega\left( i :-\omega, j: + \omega \right)}(\mathbf{k})\right).
\end{align}
If we are only interested in calculating particular moments of this distribution
we can simplify this expression substantially. For instance, if we want the
second moment of $k$, then this depends only on the expected TMRCA. We will be
taking the second derivative of $\varphi$ with respect to an arbitrary $k_i$ and
also substituting zero for each $k_i$. For solving the recursion which we will
then take the second derivative of and evaluate at zero, we define a function
$g(\mathbf{n}, k)$. The recursion for this function is
\begin{align}
  \label{eq:grecur}
  g(\mathbf{n}, k) &= \left( \sum_{i=1}^M \binom{n_i}{2}\eta(i) + \sum_{i=1}^Mn_i\sum_{j=1}^Mm(i,j) + 
    \T \frac{k^2}{2}m_2 \right)^{-1} \nonumber \\
  &\times \left( \sum_{i=1}^M \binom{n_i}{2} \eta(i) g(c(i,\mathbf{n}), k) + 
  \sum_{i=1}^Mn_i\sum_{j=1}^Mm(i,j)g(e(i,j,\mathbf{n}),k)\right).
\end{align}
In this, $c(i,\mathbf{n})$ is a coalescent operation that removes a lineage from
deme $i$. and $e(i,j,\mathbf{n})$ is a migration operation that moves a lineage
from deme $i$ to deme $j$.

%%% Local Variables: 
%%% mode: latex
%%% TeX-master: "notes.tex"
%%% End: 
