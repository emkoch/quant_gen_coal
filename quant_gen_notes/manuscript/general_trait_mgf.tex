We first derive the mgf of the distribution of trait values. Our approach
closely follows that of \citet{Schraiber2015} and \citet{Khaitovich2005}, but
generalizes to arbitrary demographies and population structure.
\citet{Khaitovich2005} considered only a single locus with a fixed genealogy,
while \citet{Schraiber2015} only consider the distribution of genealogies
produced by the standard neutral coalescent instead of considering this
distribution in the abstract. The distribution of trait values is considered
over evolutionary realizations of the combined random processes of drift and
mutation. This distribution is complex in its general form. It has a point mass
at zero corresponding the possibility that no mutations affecting the trait
occur, and the remainder could be discrete or continuous depending on the
mutational distribution. Correlations between individuals arise because of
shared history in the genealogies at individual loci with discrete topologies as
well was where on these genealogies mutations occur. An analytical expression
for the probability distribution of trait values does not exist except under
certain limits.

However, it is sometimes possible to find the mgf of this distribution or at
least use the mgf approach to learn something of the distribution of trait
values. The mgf for a trait controlled by a single nonrecombining locus is
defined as
\begin{equation}
  \label{eq:mgfdef}
  \varphi_{\mathbf{Y}}(\mathbf{k}) = \E\left[ e^{\mathbf{k} \cdot \mathbf{Y}} \right] =
  \int e^{\mathbf{k} \cdot \mathbf{Y}} \Pro(\mathbf{Y}=\mathbf{y}) \mbox{d}\mathbf{y}.
\end{equation}
The vector $\mathbf{k}$ contains dummy variables resulting from the integral
transformation of the probability distribution. Equation \eqref{eq:mgfdef} can be
rewritten by conditioning on the genealogy to give
\begin{align}
  \label{eq:cond}
  \varphi_{\mathbf{Y}}(\mathbf{k}) &= \int e^{\mathbf{k} \cdot \mathbf{Y}}
  \int \Pro(\mathbf{Y}=\mathbf{y} | \mathbf{T}=\mathbf{t}) \Pro(\mathbf{T}=\mathbf{t})
  \mbox{d}\mathbf{t} \mbox{d}\mathbf{y} \nonumber \\
  &= \int \int e^{\mathbf{k} \cdot \mathbf{Y}} \Pro(\mathbf{Y}=\mathbf{y} | \mathbf{T}=\mathbf{t}) \mbox{d}\mathbf{y}
  \Pro(\mathbf{T}=\mathbf{t})
  \mbox{d}\mathbf{t}.
\end{align}

To proceed it is necessary to make assumptions about the mutational process. The
first is that mutations occur as a Poisson process along branches and the second
is that mutations at a locus are additive. Under these assumptions, the changes
in the trait value along each branch are conditionally independent given the
branch lengths. \citet{Khaitovich2005} noted that this describes a compound
Poisson process. The mgf of a compound Poisson process with rate $\lambda$ over
time $t$ is $\exp(\lambda t (\psi(k)-1))$, where $\psi$ is the mgf of the
distribution of the jump sizes caused by events in the Poisson process. Using
this, along with the fact that the mgf of two completely correlated random
variables with the same marginal distribution is $\varphi_{X_1}(k_1+k_2)$, we
can rewrite equation \eqref{eq:cond} to get
\begin{equation}
  \label{eq:fullmgf}
  \varphi_{\mathbf{Y}}(\mathbf{k}) = 
  \int \prod_{\omega \in \Omega} \exp\left( \frac{\theta}{2} t_{\omega} \left( \psi\left(\sum_{a \in \omega}k_{a}\right) -1 \right)\right)
  \Pro(\mathbf{T}=\mathbf{t})\mbox{d}\mathbf{t}.
\end{equation}
This can be recognized as the moment generating function for $\mathbf{T}$ with
$\frac{\theta}{2} \left( \psi(\sum_{a \in \omega}k_{a}) -1 \right)$ substituted
for the dummy variable of branch $T_{\omega}$. Or,
\begin{equation}
  \label{eq:sub}
  \varphi_{\mathbf{T}}(\mathbf{s})\Bigr|_{s_{\omega}=\frac{\theta}{2} \left( \psi\left(\sum_{a \in \omega}k_{a}\right) -1 \right)}.
\end{equation}

Equation \eqref{eq:sub} shows that, if the mgf of the distribution of branch
lengths is known, then the mgf of the trait values can be obtained through a
simple substitution. When the trait is controlled by $L$ independent loci the
mgf is obtained by raising this to the power $L$. \citet{Lohse2011} derived the
mgf of the genealogy in various population models including migration and
splitting of subpopulations. Using their result for a single population it is
possible to get equation (1) of \citet{Schraiber2015} using equation
\eqref{eq:sub}. The same could be done for models with migration between
subpopulations although the number of terms in the recursion for the genealogy
mgf blows up as the sample size increases.

%%% Local Variables:
%%% TeX-master: "short_report.tex"
%%% End:
