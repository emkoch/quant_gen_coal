This general model converges to a normal model when the infinitesimal limit is
taken. This can be done by first substituting Taylor series for the genealogical
and mutational distributions in equation \eqref{eq:fullmgf}. The infinitesimal
limit should correspond to the situation where the effect sizes of mutations
becomes small as the number of loci becomes large. Assuming mutation rates are
low such that at most one mutation occurs per locus, the resulting distribution
is multivariate normal where the expected trait value is $E[T_{MRCA}] \mu$, the
variance is $E[T_{MRCA}]\sigma^2$, and the covariance between trait values in
two individuals $a$ and $b$ is $E[\tau_{a+b}] \sigma^2$. This limit requires
that the products of $L$ and moments three and greater of the mutational
distribution go to zero as the number of loci becomes large and the effect size
per mutation becomes small. This can be thought of as requiring the mutational
distribution to not have too fat of tails. Details of the derivation and a
generalization to allow multiple mutations per locus are given in
Appendix \ref{clt}.

In this normal distribution, $\mu$ can be interpreted as the rate of change in
the mean trait value per generation per genome due to mutational pressure.
$\sigma^2$ can be interpreted as the rate of accumulation of variance in trait
values per generation per genome. Interestingly, the rate of variance
accumulation is proportional to the second moment of the mutational distribution
and not to the variance. The intuition for why this is can be seen by
considering a degenerate distribution where each mutation has the same effect.
In such a case we would still expect variation among individuals because of
differences in the number of mutations they receive even though the variance of
the mutational distribution is zero. The overall variance among individual trait
values thus has a component due to differences in the number of mutations in
addition to a component due to differences in the effects of these mutations.
The first is proportional to the square of the mean mutational effect, while the
second is proportional to the mutational variance, so their sum is proportional
to $m_2$, the mean squared effect. 

Since the trait values are normally distributed, any linear combination of
sampled trait values will be as well. This includes the distributions of
observable quantities like the differences in trait values from some reference
individual or from a sample mean. The distribution of trait differences between
individuals is multivariate normal with mean zero and the following covariance,
\begin{equation}
\Cov[Y_1-Y_2,Y_3-Y_4] = \sigma^2\left( \E[\mathcal{T}_{1,4}] + \E[\mathcal{T}_{2,3}] -
\E[\mathcal{T}_{1,3}] - \E[\mathcal{T}_{2,4}] \right).
\end{equation}
This holds regardless of whether one assumes the per locus mutation rate is low
(Appendix \ref{clt}). The normal model in the infinitesimal limit provides
additional theoretical justification for studies using normal models to look for
differences in selection on quantitative traits between populations
\citep{Ovaskainen2011,Praebel2013,Robinson2015}. Additionally, this implies 
a covariance matrix based on mean coalescent times rather than population split
times should be used when modeling traits as normally distributed in
phylogenetics.

% If we let $L m_1 \to \mu$,
% $L m_2\to \sigma^2$, $L m_i\to 0$ for $i>2$, and $L^i\left(\T\right)^j \to
% 0$ for $i<j$ as $L\to \infty$, this yields
% \begin{equation}
%   \label{eq:clt}
%   \exp \left( \sum_{\omega \in \Omega}\E[T_{\omega}] \left( \mu \left(
%   \sum_{a \in \omega} k_a\right) + \frac{\sigma^2}{2}\left( \sum_{a \in \omega}
%   k_a\right)^2\right)\right).
% \end{equation}

%%% Local Variables:
%%% TeX-master: "short_report.tex"
%%% End:
 
