A situation in which higher order moments of the trait distribution can be
relevant is in the response of the population to selection. Evolutionary
quantitative genetics often assumes the distribution of additive genetic values
in the population remains normally distributed as selection alters the mean and
variance. \citet{Turelli1990} used a multilocus population genetic model to show
how departures from normality affect the response to selection. In their
analysis these departures are due to the build up of linkage disequilibrium.
However, their results are valid regardless of how departures from normality
arise. This theory can be used to analyze the response to selection in the toy
situation of a trait that has evolved neutrally up to the current time and is
then subjected to one generation of selection under a particular fitness
function.

According to \citet{Turelli1990}, the response of the mean phenotype in the
population is
\begin{equation}
  \label{eq:selresp}
  \Delta \bar{Z} = M_2L_1 + M_3L_2 + \gamma_4M^2_2L_3 +
  \left( M_5-4M_3M_2\right)L_4 + \ldots.
\end{equation}
$Z$ refers to a phenotypic value which has an environmental component as opposed
to the trait value $Y$ which is due entirely to genetics. $V_g$ is the variance
in trait values in the population, and $\gamma_4$ is the excess kurtosis above a
normal distribution. The term $M_i$ is again the $i^{th}$ central moment of the
trait value distribution in the population. The $L_i$ are selection gradients in
terms of the genotypic moments and describe the shape of the fitness function on
the trait values. Precisely, $L_i$ is the partial derivative of the log of the
mean fitness with respect to the $i^{th}$ moment of the trait value distribution

In the absence of environmental effects this equation fully describes the
response of the mean trait value to one generation of selection.
Equation \eqref{eq:selresp} shows that whether higher order moments of the trait
distribution contribute to the selection response depends on the shape of the
fitness function through the $L_i$. The excess kurtosis affects the response to
selection linearly with $L_3$, the selection gradient with respect to the third
moment of the trait value distribution. As a very simple example consider
selection acting through a cubic fitness function:
\begin{equation}
  \label{eq:cubsel}
  W_g(Y) = b_0 + b_3(Y-\bar{Y})^3.
\end{equation}
This fitness function models selection on differences from the current mean
breeding value in the population. It can be thought of as directional selection
that acts most strongly on the tails of the distribution. However, the cubic
fitness function can give negative values which is something we would like to
avoid. In order for it to be sensible we must assume that the degree of cubic
selection, $b_3$, is not too large relative to fitness value at the population
mean, $b_0$.

The response to selection is
\begin{equation}
  \label{eq:cubresp}
  \Delta \bar{Z} = \frac{M_4\beta}{1 + M_3\beta},
\end{equation}
where $\beta=b_3/b_0$. Simulations show this predicts the response to selection
up until $\beta\approx 2e-3$. This is another ratio of random quantities, so
calculating the expected response to selection is not feasible. However, we
could conjecture that demographic and mutational process increasing the expected
fourth central moment relative to the third would tend to increase the response
to selection.

Equation \eqref{eq:cubsel} is not the most realistic fitness function because it
is centered on the mean trait value in the population. In the absence of
stabilizing selection, the mean will have drifted since the most recent common
ancestor of the population, so there is not a strong reason for selection to be
centered there. For the selection to be strictly cubic in shape is perhaps
unrealistic, but a large set of fitness functions will have a cubic component.
Equation \eqref{eq:cubresp} shows that this component depends on moments of the
trait distribution that sensitive to demography and genetic architecure.
%%% Local Variables:
%%% TeX-master: "short_report.tex"
%%% End:
