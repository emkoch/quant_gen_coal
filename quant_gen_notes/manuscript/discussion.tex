Neutral models of quantitative trait evolution are important for establishing a
baseline against which to test for selection. \citet{Schraiber2015} recently
analyzed a neutral model of trait evolution that made few assumptions about the
number of loci potentially affecting the trait and the distribution of
mutational effects at these loci. However, they only derived results for
constant-size, panmictic populations. This study extends their results to
populations with arbitrary distribution of coalescent times and therefore
demographies and population structures. As their key result
\citet{Schraiber2015} derived the characteristic function of the distribution of
trait values in a sample. In this paper we instead work with the moment
generating function, but the two approaches are interchangeable as long as the
mgf for the distribution of mutational effects exists (which it will provided
the moments are all finite). Our main result, given in equation \eqref{eq:sub} is
to show that the generating function obtained by \citet{Schraiber2015} is a
special case of a general procedure whereby the moment generation function for a
trait distribution can be obtained by making a simple substitution into the
moment generating function for a distribution over genealogies. The moment
generating functions for many demographic histories of interest and sample sizes
above two are sufficiently complex that solving for them is impractical
\citep{Lohse2011}. However, progress can still be made by using Taylor
expansions to write moment of the trait distribution in terms of moments of the
genealogical and mutational distributions.

This result extends previous work using coalescent theory to investigate neutral
models of quantitative traits \citep{Whitlock1999,Schraiber2015}. Ours is the
most general model yet analyzed and as a natural first step we show that the
infinitesimal limit suggested by \citet{Fisher1918} leads to a model where
phenotypes are normally distributed as the number of loci becomes large and the
variance of effect sizes becomes small. In the limiting distribution, the
variance of the difference in trait values between two individual is
proportional to the expected pairwise coalescent time between them, and the
covariance between a pair of differences more generally is $\Cov[Y_a - Y_b,Y_c -
Y_d] \propto \E[\mathcal{T}_{a,d}] + \E[\mathcal{T}_{b,c}] -
\E[\mathcal{T}_{a,c}] - \E[\mathcal{T}_{b,d}]$. The resulting covariance matrix
completely specifies the neutral distribution under the infinitesimal model.
This is similar to classic models in evolutionary quantitative genetics
considering the neutral divergence of trait values after population splits
\citep{Lande1976,Lynch1989} but holds regardless of the precise details of
population structure and history. \citet{Schraiber2015} derive essentially the
same distribution using a central limit theorem argument. It is worth nothing
that the covariance matrix is scaled by the second moment of the mutational
distribution and not the variance.

Sparse traits were compared to the normal model by calculating how the first
four expected central moments different from those expected under normality.
This showed how demography and the genetic architecture separately influence the
expected deviation from normality. For a fixed expected trait sparsity,
population growth produces greater deviations in the fourth central moment while
population bottlenecks produce lower deviations (Figures
\ref{fig:Qexp},\ref{fig:Qland}). However, for realistic demographic scenarios,
we find that the effects attributable demography are small (Figure
\ref{fig:afeucomp}). No cases were analyzed where the individuals in the
population were not all exchangeable, but this would likely increase deviations
from normality as drift will lead to differences in trait means between
subpopulations.

We next apply the above theory to three simple problems where a coalescent
perspective on the neutral distribution of a quantitative trait could be useful.
The first of these is the question of the appropriate null distribution for
$Q_{ST}$ at the population level. We show how the null distribution under the
normal model can be easily simulated from, providing a much better approximation
when populations are correlated than previous approaches \citep{Whitlock2009}
(Figure \ref{fig:qst_deme}). The second case shows how the single-generation
response to cubic selection depends on the third and fourth central moments of
the trait distribution \citep{Turelli1990}, moments whose expectation depends on
genetic architecture and demography. Lastly, we show it is theoretically
possible for sparse traits to infer the shape of the mutational distribution but
not the mutational target size or magnitude of mutational effects without
assuming a particular shape.

Even though we have broadened the model space for neutral traits, many features
of real populations have not yet been incorporated. Linkage between loci is a
particular concern as these is substantial linkage disequilibrium between QTLs
\citep{Bulik-Sullivan2015}. \citet{Lohse2011} derived the form of the moment
generating function for linked loci and future work will attempt to incorporate
this using equation \eqref{eq:sub}. In particular, it will be interesting to see
how this affects the distribution in the infinitesimal limit. Diploidy,
dominance, and epistasis have also been ignored thus far. The qualitative
effects described here should hold under diploidy, but having trait values
within individuals summed over loci from two copies of the genome will decrease
deviations from normality. Dominance will also lead to a normal distribution as
the effects are independent between loci, but future work is needed to examine
how this interacts with the distribution of genealogies to affect the trait
distribution.

\citet{Barton2017} recently performed a deep mathematical investigation of a
more formal ``infinitesimal model'' different from the infinitesimal limit
considered here. They proved conditions under which the trait values of
offspring within a family are normally distributed with variance independent of
the parental trait values conditional on the pedigree and segregation variance
in the base population. Interestingly, they found the normality for offspring
trait values still holds under some forms of pairwise epistasis that are not too
extreme. This implies it may be possible to include epistasis in the
infinitesimal limit considered here. It would be important to know how epistasis
affects the neutral divergence of trait between populations and species.

Although GWAS of many trait have shown them to be controlled by large numbers of
loci \citep{Boyle2017}, this will not necessarily be the case for every trait of
interest to biologists. It has been suggested, for instance, that gene
expression levels have a sparse genetic architecture \citep{Wheeler2016}. Since
there is much interest in testing whether natural selection has acted on gene
expression levels \citep{Whitehead2006,Gilad2006,Yang2017}, well-calibrated
goodness-of-fit tests will need to take into account the complications that
arise when trait distributions deviate from normality \citep{Khaitovich2005}.
Direct measurements of mutational distributions \citep{Gruber2012,Metzger2016}
could aid in such as calibration. Finally, equation \eqref{eq:emoms} suggests a
means to determine whether sparsity is impacting trait distributions.
Populations with a greater $\mathbbm{T}_{3,3}$ to $\mathbbm{T}_{2,2}$ ratio are also expected to
show a greater $M_3$ to $M_2$ ratio, so this comparison could be made in studies
of multiple populations. Because gene expression studies generally measure a
large number of traits, observing such a trend on average could be a good sign
of deviations from normality.

%% The main result of this work has been to show that the characteristic function
%% for the distribution of trait values derived by \citet{Schraiber2015} is a
%% special case of the general relationship given in equation \eqref{eq:sub}. What
%% this implies is that to get mgf of the trait value distribution one must first
%% be able to write the mgf for the genealogical distribution. These functions have
%% been studied by \citet{Lohse2011} and in subsequent papers. They are recursive
%% in nature and the number of terms becomes quickly intractable as the number of
%% individuals increases. Things are simplified substantially when either a low
%% mutation approximation or infinitesimal model is used. In these cases only the
%% mean branch lengths or expected pair coalescent times, respectively, are
%% necessary to describe the trait value distribution.

%% The focus on kurtosis here is perhaps undue. It is presented mostly as an
%% example of how a simple and intuitive result can be obtained from the given
%% mgfs. For many traits the infinitesimal model surely provides a good null
%% distribution. For traits that do have sparse genetic architectures and
%% mutational distributions with large tails it is possible that aspects of the
%% mutational distribution could be learned from population samples, but it would
%% likely be necessary to somehow pool information across a large number of traits.
%% A possible instance where this could be achieved is for gene expression levels
%% \citep{Wheeler2016}.

%% It would be interesting to extend these results, which are only for the haploid,
%% unlinked, additive case to include factors like dominance, epistasis and
%% linkage. For epistasis this is likely to be very difficult since there are no
%% nice relationships for the mgf of a nonlinear combination of random variables.
%% Still, some approximations may be possible for particular forms of epistasis.
%% These would be of particular interest when testing for departures from
%% neutrality in structured populations as in \citet{Ovaskainen2011}.

%%% Local Variables:
%%% TeX-master: "short_report.tex"
%%% End:
