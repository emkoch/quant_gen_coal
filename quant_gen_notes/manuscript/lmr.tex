The model used so far assumes that loci are unlinked and can experience an
infinite number of mutations. However, a useful simplification is that at most
one mutation per locus occurs. This approximation will be reasonable as long as
the nucleotide positions affecting the trait are loosely linked throughout the
genome. The low-mutation-rate approximation greatly simplifies the mgf of the
trait distribution such that it is no longer necessary to know the full form of
the mgf of the genealogy.
\begin{equation}
\label{eq:lowmut}
\varphi_{\mathbf{Y}}(\mathbf{k}) \approx \left[ 1 + \sum_{\omega \in \Omega}
  \E[T_\omega] \T \left( \psi\left( \sum_{a \in \omega} k_a\right) -1 \right) +
  O\left( \theta^2 \right))\right]^L.
\end{equation}
Equation \eqref{eq:lowmut} ignores terms that are order two and above in the
mutation rate. The convenient aspect of this equation is that it depends only on
the expected length of each branch, as opposed to equation \eqref{eq:fullmgf}
which requires moments of branch lengths order two and greater. We can use this
to express moments of the trait distribution in terms of expected branch lengths
that can be calculated from coalescent models.

%%% Local Variables:
%%% TeX-master: "short_report.tex"
%%% End:
