Similarly to neutral models in genetics, neutral models of quantitative traits
have been useful in providing a null distribution against which to look for the
action of selection using goodness-of-fit tests \citep{Lande1976}, as well as
clarifying the effects on variation of neutral forces such as demography and
mutation in isolation \citet{Lynch1986}. The common approach, as in most of
quantitative genetics, is to begin by modeling phenotypes as normally
distributed whether this is among offspring within a family, among members of a
population, or between species \citep{Turelli2017}. Indeed, it has been
suggested this is the defining characteristic of the field \citep{Rice2004}.
This assumption might be well-justified by the notion that phenotypes are
influenced by a large number of sufficiently independently inherited Mendelian
factors \citep{Fisher1918}, or it could simply appear to be approximately true
in practice.

%% Such work has found its way into many areas of evolution and
%% ecology \citep{Walsh2013,Kruuk2004}. \citet{Lande1983} used this approach to
%% measure the action of natural selection on correlated traits over a single
%% generation.

In any case, neutral models assuming normality have been used in numerous
contexts. \citet{Freckleton2002} and others have used Brownian motion to detect
and correct for phylogenetic dependence in studies of phenotypic
evolution. \citet{Ovaskainen2011} used multivariate normality of traits in
developing a test for spatially varying selection. Normal models of phenotypic
evolution have incorporated genetic drift by including factors like population
size and subdivision \citep{Chakraborty1982,Lynch1986,Lande1992}. These models
have typically examined the dynamics of phenotypic evolution forwards in time as
a balance between mutation creating variance, migration spreading it among
subpopulations, and fixation removing it. This allows one to derive the
equilibrium genetic variance and the rate it is approached.

Of course, there is nothing really remarkable about modeling phenotypes
according to a normal distribution. The strength and broad applicability of
quantitative genetics is that traits can be studied without worrying about the
number of sites where mutations would affect a trait, the genealogies at these
sites, or the distribution of mutational effects. However, heritable phenotypic
variation is ultimately due to discrete mutations at discrete loci, and how this
variation is distributed depends on the genealogies at these loci. When the
number of loci affecting a trait is large the central limit theorem ensures most
of these details can be ignored, but a full model would have to include them.
Importantly, deviations from normality may affect the outcomes of
goodness-of-fit tests that necessarily aim to identify outliers from a normal
model.

The principle modeling framework for genealogical variation is the coalescent
process \citep{Wakeley2008}, but few studies have connected this with
quantitative genetics. The most well-known of these is
\citet{Whitlock1999}, who used a coalescent argument to argue that measures of trait ($Q_{ST}$) and
genetic ($F_{ST}$) differentiation have the same expected value under general
models of population subdivision. \citet{Griswold2007} used coalescent
simulations to investigate the effects of shared ancestry and linkage
disequilibrium on the matrix of genetic variances and covariances between traits
($\mbox{\textbf{G}}$). They found that the eigenvalues of the expected
$\mbox{\textbf{G}}$ and the expected eigenvalues can be different. Although not
explicitly connected to the coalescent,
\citet{Ovaskainen2011} used the formalism of coancestry coefficients to model
the full distribution of correlated traits in related subpopulations. The method
they developed was based on the result that the covariance in trait values,
conditional on the G-matrix in the ancestral population, depends only on the
pairwise coancestry coefficients.

Most recently, \citet{Schraiber2015} developed a very general model of
quantitative trait evolution based on the coalescent that allows for an
arbitrary distribution of mutational effects and number of loci affecting the
trait. They derived the characteristic function for the distribution of
phenotypic values in a sample and showed how such values can deviate strongly
from normality when the number of loci is small or the mutational distribution
has fat tails. Their model is closely related to an earlier one due
to \citet{Khaitovich2005} that attempted to model the evolution of gene
expression values on phylogenetic trees but assumed a single non-recombining
locus.

\citet{Schraiber2015} derived their results for a panmictic, constant-size
population. Natural populations are often far from equilibrium and show
considerable spatial structure, and it is unclear how this might influence
deviation from normality. A great advantage of coalescent theory is its ability
to hand nonequilibrium demographies, and we relax the constant-size assumption
in this work. Additionally, the analysis of quantitative traits in structured
populations is of particular interest when searching for evidence of local
adaptation or stabilizing selection. To this end the $Q_{ST}$ paradigm has been
developed \citep{Whitlock2008,Spitze1993}. $Q_{ST}$, defined as the ratio of the
variance between subpopulations to the total variance in a quantitative trait,
is compared to $F_{ST}$ which measures the same property for genetic variation.
$F_{ST}$ for neutral genetic variation is used to define a null distribution for
$Q_{ST}$. If an observed $Q_{ST}$ is sufficiently extreme relative to this null
it is concluded that natural selection has acted. Modern extensions of this idea
have been developed by \citet{Ovaskainen2011} for genetic values measured in
breeding experiments and by \citet{Berg2014} for genetic values measured in
GWAS. Understanding the neutral distribution of trait values at the sample and
population level is necessary to interpret the results of these goodness-of-fit
tests.

Here, we take another step towards synthesizing two branches of evolutionary
genetics: population and quantitative genetics. We begin by generalizing the
work of \citet{Schraiber2015} to an arbitrary distribution of coalescent times
by deriving the form of the moment generating function (mgf). Their key result,
the characteristic function of the sampling distribution of phenotypic values,
is a special case of this. We then show how a normal models arises by taking the
infinitesimal limit where the effect size per mutation becomes small as the
number of loci potentially affecting the trait becomes large. We then calculate
the third and fourth central moments of the trait distribution in panimictic
populations to illustrate how departures from normality depend both on genetic
parameters and genealogical distributions. Finally, we discuss the consequences
of these results for $Q_{ST}$ tests, the response to selection, and the
inference of genetic architecture.

An improved null distribution for $Q_{ST}$ tests can be derived simply by using
the normal distribution that arises in the infinitesimal limit of our coalescent
model. When selection acts primarily through on the tails of the trait
distribution the, single generation response depends on trait moments that are
sensitive to genetic architecture and demographic. Additionally, it may be
possible to infer features of the mutational distribution when both trait values
and sequence data are available.

%%% Local Variables:
%%% TeX-master: "short_report.tex"
%%% End:
