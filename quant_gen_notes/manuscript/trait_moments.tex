\newcommand{\AAA}{\E[\mathbbm{T}_{4,4}] + \frac{1}{3}\E[\mathbbm{T}_{3,4}] + \frac{2}{9}\E[\mathbbm{T}_{2,4}]}
\newcommand{\BBB}{\frac{1}{9}\E[\mathbbm{T}_{2,4}] + \frac{1}{6}\E[\mathbbm{T}_{3,4}]}
\newcommand{\CCC}{\E[\mathbbm{T}_{4,4}] - \frac{1}{6}\E[\mathbbm{T}_{3,4}] - \frac{1}{9}\E[\mathbbm{T}_{2,4}]}

For most population genetic models and reasonable sample sizes, the recursive
nature of the trait distribution mgf makes it computationally infeasible to
solve under general parameter values. However, it is not necessary to have an
expression for the full mgf in order to derive moments of the trait distribution
in terms of moments of branch lengths and moments of the mutational
distribution. Under the low mutation rate approximation moments can be
calculated by differentiating equation \eqref{eq:lowmut}. Even without making
this approximation, moments can be calculated by taking Taylor expansions in
equation \eqref{eq:cond} and only considering terms contributing to the desired
moment's order. A symbolic math program was written to do this, and the details
are given in Appendix ~\ref{symmath}. The normal distribution is completely
defined by its first two moments, and the extent to which a trait distribution
deviates from normality can be measured by the extent to which its moments
deviate from those of a normal distribution with the same mean and variance.

We have to differentiate between the moments of an individual $Y_a$ and those at
the population level which are also random quantities. The expectation of $Y_a$
is $L \T m_1 \E[T_{MRCA}]$, and the variance is $L \T m_1 \E[T_{MRCA}] +
L(m_1^2\T)\Var[T_{MRCA}]$. Although simple to derive using computer algebra,
expressions for the higher central moments of $Y_a$ are complicated even under
the low mutation rate approximation and there is not much to be gained by
showing them here.

Since $Y_a$ is relative to a value that is not directly observed, more insight
can be gained by considering moments at the population level.
~\citet{Schraiber2015} computed the expected first four central moments of a
constant-size population. These same expectations under an arbitrary demographic
history are
\begin{subequations} \label{eq:emoms}
\begin{align}
  &\E[M_2] = L \T \E[\mathbbm{T}_{2,2}] m_2 \label{eq:emoms2}\\
  &\E[M_3] = L \T \E[\mathbbm{T}_{3,3}] m_3  \label{eq:emoms3}\\
  &\E[M_4] = 3\left(L \T \E[\mathbbm{T}_{2,2}] m_2\right)^2 \nonumber \\
  &+ 3\L \left(\T m_2\right)^2\Var[\mathbbm{T}_{2,2}] + \frac{1}{3}
  L \left(\T m_2\right)^2
    \left( \frac{11}{9} \E[\mathbbm{T}_{2,4}^2]-\frac{1}{3}\E[\mathbbm{T}_{2,4}\mathbbm{T}_{3,4}]-
    \frac{1}{4}\E[\mathbbm{T}_{3,4}^2]\right) \nonumber \\
  %% &\E[M_4] = 3L(L-1)\left(\T \E[T_{2,2}] m_2\right)^2 \nonumber\\
  %% &+ \frac{1}{3}L \left(m_2\T\right)^2(4 \E[T_{2,4}^2] + 8 \E[T_{3,4}T_{2,4}] +
  %%     15 \E[T_{4,4}T_{3,4}] + 6 \E[T_{3,4}^2] + 10 \E[T_{4,4}T_{2,4}] + 9 \E[T_{4,4}^2]) \nonumber\\
  &+ L m_4 \T ( \E[\mathbbm{T}_{4,4}] + \frac{1}{3} \E[\mathbbm{T}_{3,4}] +
    \frac{2}{9} \E[\mathbbm{T}_{2,4}] ).
  \label{eq:emoms4}
\end{align}
\end{subequations}
Some moment calculations done by hand under low mutation rates are presented in
Appendix \ref{kurt}. In the derivation of the normal model it was assumed that
$L\T m_3$, $L \T m_4$, and $L \left( \T \right)^2 m_2$ go to zero as $L\to
\infty$. The expectations of the third and fourth moments thus go to zero and
$3\left(L \T \E[\mathbbm{T}_{2,2}] m_2\right)^2$ as expected for a normal distribution.
Deviations of the population distribution from normality depend on the
distribution of coalescent times as well as the genetic architecture, and they
can be investigated by comparison to the expected moments under normality. Even
though recombination is not included in the model, we can form an idea about how
linkage might impact deviations from normality by looking at line two of
equation \eqref{eq:emoms4}. This line corresponds to the contribution from two
mutations occurring at a locus. The first part indicates that the expected
fourth moment increases with the variance of the pairwise coalescence time. The
second part is much harder to interpret but seems likely also to be positive.

%% The kurtosis of a distribution is defined as
%% \begin{equation*}n
%%   \mbox{Kurt}[X]=\frac{\E[(X-\E[X])^4]}{\E[(X-\E[X])^2]^2}.
%% \end{equation*}
%% This is the fourth central moment divided by the variance. The kurtosis is a
%% measure of the propensity of a distribution to produce outliers
%% \citep{Westfall2014} and it has been used previously to investigate the
%% deviation from normality under different population genetic models
%% \citep{Nei1966,Chakraborty1982}. The kurtosis of a single trait value, where the
%% expectation is over evolutionary replicates of demography and mutation, is
%% \begin{equation}
%%   \label{eq:kurt1}
%%   \mbox{Kurt}[Y_a] \approx \frac{\kappa}{L\T \E[T_{MRCA}]} + 3.
%% \end{equation}
%% The kurtosis of a normal distribution is $3$, so it is immediately clear that
%% the trait distribution will exceed this. This excess is determined by the ratio
%% of the kurtosis of the mutational distribution, denoted by $\kappa$, and the
%% expected number of mutations before the most recent common ancestor of the
%% sample ($L\T \E[T_{MRCA}]$). This make intuitive sense that the kurtosis will be
%% higher when mutations tend to produce more outliers and lower when more of these
%% are added up and approach normality through the central limit theorem.

%% As mentioned before, $Y_a$ is defined relative to the value of the most recent
%% common ancestor of the sample and isn't directly observed. What is observable
%% are differences in trait values between individuals, and equation
%% \eqref{eq:kurt1} also holds for the kurtosis of trait differences if the
%% expected TMRCA is replaced by twice the expected pairwise coalescence time.

%% The trait kurtosis in a populations of individuals is also of interest as
%% individuals with extreme trait values could be important for future selection or
%% could be mistakenly thought to be adapted to a different environment. When
%% individuals are exchangeable, the population kurtosis is given by
%% \begin{equation}
%%   \label{eq:popkurtcoal}
%%   \E[\mbox{Kurt}] \approx 3 + \frac{\kappa( 4\E[T_2] - 6\E[T_3] + 
%%     3\E[T_4])}{L \T \E[T_2]^2},
%% \end{equation}
%% where $T_i$ is the expected $T_{MRCA}$ for a sample of size $i$. The relative
%% values of these different coalescence times affects the population kurtosis. In
%% a constant size population equation \eqref{eq:popkurtcoal} reduces to $3 +
%% \frac{\kappa}{2L\T \E[T_2]}$ as expected. For a constant size population this
%% result agrees with that in \citet{Schraiber2015}. More details on the derivation
%% of both kurtosis values is given in Appendix \ref{kurt}.

%% \begin{figure}
%%   \centering
%%   \includegraphics[width=0.8\textwidth]{kurt_land.pdf}
%%   \caption{\footnotesize The expected population kurtosis under different step
%%     population size changes. Red lines are isoclines where the expected number
%%     of mutations is constant. Green lines are isoclines where the expected
%%     kurtosis is constant.}
%%   \label{fig:kurtscape}
%% \end{figure}

%% The excess kurtosis in Equation \ref{eq:popkurtcoal} can be broken down into
%% contributions from the mutational distribution ($\kappa$), from the sparsity of
%% the trait's genetic architecture ($L \T \E[T_2]$), and from the deviation of
%% the coalescent time distribution from that in a constant-size population
%% ($\frac{ 4\E[T_x2] - 6\E[T_3] + 3\E[T_4] }{ \E[T_2] }$). The interaction of these
%% contributions can be investigated by examining the excess kurtosis under
%% different demographies. A step change in population size is a simple example do
%% this with.

%% Figure \ref{fig:kurtscape} shows expected excess kurtosis when the population
%% size changes at different times (expressed in $N_0$ generations) and to
%% different sizes (expressed relative to $N_0$). As expected, the kurtosis
%% increases when the population size decreases after the step change. The
%% magnitude of this increase is less the longer one waits for the population size
%% to change. However, the main reason for this is that a change in population size
%% affects the expected number of mutations, and more mutations mean a lower
%% kurtosis. The impact of the coalescent time distribution aside from this effect
%% is illustrated by the difference between the red isoclines for the expected
%% number of mutations and the green isoclines for the expected kurtosis. The
%% step-change demography tends to increase the expected kurtosis when the expected
%% number of mutations is small and decrease it when the expected number of
%% mutations is large.

%% In general, for the kurtosis to substantially exceed that of the normal
%% distribution, the expected number of variants affecting the trait must at least
%% be on the same order as the mutational kurtosis. This requires mutation to
%% produce extreme values with considerable frequency or for the trait architecture
%% to be very sparse. It's possible that traits such as expression of certain genes
%% satisfy this criteria \citep{Wheeler2016}.

%%% Local Variables:
%%% TeX-master: "short_report.tex"
%%% End:
