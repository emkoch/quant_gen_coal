Here we will show that the distribution of trait values is multivariate normal
regardless of the distribution of coalescent times. Recall that the moment
generating function for a general distribution of coalescence times is
\begin{equation}
  \varphi_T(\mathbf{s}) = \int \exp \left( \sum_{\omega \in \mathcal{O}} s_{\omega}t_{\omega} \right)
  P(\mathbf{T}=\mathbf{t})d\mathbf{t}.
\end{equation}
The Taylor series expansion of $\exp \left( s_{\omega}t_{\omega} \right)$ is
\begin{equation}
  1 + s_{\omega}t_{\omega} + \frac{s_{\omega}^2t_{\omega}^2}{2} + \ldots. \nonumber
\end{equation}
$\exp \left( \sum_{\omega \in \mathcal{O}} s_{\omega}t_{\omega} \right)$ is therefore
\begin{equation}
  1 + \sum_{\omega \in \mathcal{O}} s_{\omega}t_{\omega} +
  \sum_{\omega_1 \neq \omega_2} s_{\omega_1}t_{\omega_1}s_{\omega_2}t_{\omega_2} + 
  \sum_{\omega \in \mathcal{O}} \frac{s_{\omega}^2t_{\omega}^2}{2} + \ldots. \nonumber
\end{equation}
This of course means that
\begin{equation}
  \label{eq:mgf_L}
  \left(\int \exp \left( \sum_{\omega \in \mathcal{O}} s_{\omega}t_{\omega}
  \right)P(\mathbf{T}=\mathbf{t})d\mathbf{t}\right)^L = \left(1 + \sum_{\omega \in \mathcal{O}}
  s_{\omega}E[t_{\omega}] + \sum_{\omega_1 \neq \omega_2}
  s_{\omega_1}s_{\omega_2}E[t_{\omega_2}t_{\omega_1}] + \sum_{\omega \in
    \mathcal{O}} \frac{s_{\omega}^2E[t_{\omega}^2]}{2} + \ldots\right)^L. 
\end{equation}
Based on equation \ref{eq:sub} we can get the moment generating function for the
trait values by making the substitution $s_{\omega}\to \frac{\theta}{2} \left(
\psi\left(\sum_{a \in \omega}k_{a}\right) -1 \right)$. We will also take the
Taylor series of the moment generating function of the mutational distribution.
Noting that $\frac{d^n}{dk^n}\varphi_X(k)\Bigr|_{k=0} = E[X^n]$, if we let $m_1$
be the mean and $m_2$ be the second moment of the mutational distribution, then
\begin{equation}
  \psi\left( \sum_{a \in \omega} k_a \right) = 1 + m_1 \left( \sum_{a \in \omega}
  k_a\right) + m_2/2\left( \sum_{a \in \omega} k_a\right)^2 + \ldots.
  \nonumber
\end{equation}
Substituting this into equation \ref{eq:mgf_L} we get
\begin{align}
  \biggl( 1 &+ \frac{\theta}{2} \sum_{\omega \in \mathcal{O}} E[t_{\omega}]\left( m_1 \left(
  \sum_{a \in \omega} k_a \right) + \frac{m_2}{2} \left( \sum_{a \in \omega}
  k_a\right)^2 + \ldots \right) \\
  &+ \sum_{\omega_1 \neq \omega_2} \T\left( m_1 \sum_{a \in \omega_1} k_a + \ldots \right)
  \T \left( m_1 \sum_{b \in \omega_2} k_b + \ldots \right)
  E[t_{\omega_1\omega_2}] \\
  &+ \sum_{\omega \in \mathcal{O}} \left(\T\right)^2\left(m_1 \sum_{a \in \omega} k_a +
  \ldots \right)^2 E[t^2_{\omega}] \biggr)^L.
\end{align}
Now if we take the limit such that $L\T m_1 \to \mu$ and $L\T m_2\to \sigma^2$
as $L \to \infty$ we get
\begin{equation}
  \exp \left( \frac{\theta}{2} \sum_{\omega \in \mathcal{O}}E[t_{\omega}] \left( \mu \left(
  \sum_{a \in \omega} k_a\right) + \frac{\sigma^2}{2}\left( \sum_{a \in \omega}
  k_a\right)^2\right)\right),
\end{equation}
providing that the product of $L$ and moments of the mutational distribution
higher than two go to zero. This also requires that the limit of $L(\T)^2m_1$
goes to zero as the number of sites goes to infinity. $\mu$ can be interpreted
as the rate of change in the mean trait value per generation per genome due to
mutational pressure. $\sigma^2$ can be interpreted as the rate of accumulation
of variance in trait values per generation per genome. Ignoring terms with
$(\T)^2$ is equivalent to the assumption that mutation rates are low enough that
only one mutation per locus occurs. It is clear from inspection that this is the
moment generating function for a multivariate normal distribution where the
expected trait value is $E[T_{MRCA}] \mu$, the variance in a trait value is
$E[T_{MRCA}]\sigma^2$, and the covariance in trait values is
$(T_{MRCA}-E[\tau_{1,2}]) \sigma^2$. This covariance is proprtional to the
expected amount of shared branch length before the most recent common ancestor
of the sample.
%%% Local Variables:
%%% TeX-master: "notes.tex"
%%% End:
