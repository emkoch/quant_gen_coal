Recall that the moment generating function for the distribution of trait values
from a single locus is
\begin{equation*}
  \varphi_{\mathbf{Y}}(\mathbf{k}) = \int \exp \left( \sum_{\omega \in \Omega} s_{\omega}t_{\omega} \right)
  \Pro(\mathbf{T}=\mathbf{t})\mathrm{d}\mathbf{t}
  \Bigr|_{s_{\omega}=\frac{\theta}{2} \left( \psi\left(\sum_{a \in \omega}k_{a}\right) -1 \right)}.
\end{equation*}
If we substitute in the Taylor series expansions for the moment generating
function of the trait value distribution we get
\begin{equation*}
  \int \prod_{\Omega} \exp\left[ t_{\omega} \T \left( \sum_{n=1}^{\infty} \frac{m_n}{n!}
    \left( \sum_{a \in \omega} k_a \right)^n \right) \right]
  \Pro(\mathbf{T} = \mathbf{t}) \mathrm{d}\mathbf{t}.
\end{equation*}
If we then write the Taylor series of each exponential function we get
\begin{equation*}
  \int \prod_{\Omega} \left[ \sum_{j=0}^\infty \frac{t_{\omega}^j}{j!}
  \left( \T \right)^j \left( \sum_{n=1}^{\infty} \frac{m_n}{n!}
  \left( \sum_{a \in \omega} k_a \right)^n \right)^j \right]
  \Pro(\mathbf{T} = \mathbf{t}) \mathrm{d}\mathbf{t},
\end{equation*}
which is equivalent to
\begin{equation*}
  1 + \sum_{\Omega} \E[T_\omega] \T \sum_{n=1}^{\infty} \frac{m_n}{n!} \left(
  \sum_{a \in \omega} k_a \right)^n +
  \sum_{\Omega \times \Omega} \frac{1}{2} E[T_{\omega_1}T_{\omega_2}]
  \sum_{n=1}^{\infty} \T \frac{m_n}{n!} \left(\sum_{a \in \omega_1} k_a \right)^n
  \sum_{n=1}^{\infty} \T \frac{m_n}{n!} \left(\sum_{a \in \omega_2} k_a \right)^n + \ldots
\end{equation*}

This is raised to the power $L$ for a trait controlled by $L$ loci. We want the
limit as the number of loci increases while the size of mutational decreases.
This can be expressed by the limits $L \T m_1 \to \mu$, $L \T m_2\to \sigma^2$
and $L \T m_i\to 0$ for $i>2$ as $L \to \infty$. These limits also imply we
should retain terms of order $L m_1^2$, so we will also let $L \left(\T\right)^2
m_1^2 \to \eta$ as $L \to \infty$. Knowing we will not be retaining $m_3$ and
above we can rewrite the mgf as
\begin{equation*}
  \left(1 + \sum_{\Omega} \E[T_\omega] \T \left(m_1 \left(\sum_{a \in \omega} k_a \right) +
  \frac{m_2}{2} \left(\sum_{a \in \omega} k_a \right)^2\right) +
  \sum_{\Omega \times \Omega} \frac{1}{2} \E[T_{\omega_1}T_{\omega_2}]
  \T m_1 \left(\sum_{a \in \omega_1} k_a \right)
  \T m_1 \left(\sum_{a \in \omega_2} k_a \right)\right)^L.
\end{equation*}
Taking the appropriate limit here is severely annoying to do. 

Doing so is tedious, but the result of taking these limits is
\begin{equation}
  \label{eq:norm_mgf}
  \exp \left( \sum_{\omega \in \Omega}\E[T_{\omega}] \left( \mu 
  \sum_{a \in \omega} k_a + \frac{\sigma^2}{2}\left( \sum_{a \in \omega}
  k_a\right)^2\right) +
\sum_{\Omega \times \Omega} \Cov[T_{\omega_1},T_{\omega_2}]\frac{\eta}{2}
\left(\sum_{a \in \omega_1} k_a\right) \left(\sum_{b \in \omega_1} k_b\right)\right).
\end{equation}

This is multivariate normal distribution with mean equal to $\E[T_{MRCA}]\mu$,
variance equal to $\E[T_{MRCA}]\sigma^2 + \Var[T_{MRCA}]\eta$, and covariance
between $Y_a$ and $Y_b$ equal to $\E[\tau_{a+b}]\sigma^2 + \Var[T_{MRCA}]\eta$.
This can be seen from equation \eqref{eq:norm_mgf} by noting that in the mgf for
a multivariate normal distribution the coefficient in the exponential of $k_a$
is the mean of $Y_a$ and the coefficient of $k_ak_b$ is $2\Cov[Y_a,Y_b]$ if
$a\neq b$ and $\Var[Y_a]$ if $a=b$. The terms depending on $\eta$ and the
variance of the $T_{MRCA}$ correspond to more than on mutation occuring per
locus. A low mutation rate approximation can be taken by ignoring these.

These $Y$ value can never be directly observed. Rather, differences beteween
individual trait values are what analyses would be based on. Since the $Y$ are
normal so will be differences like $Y_a-Y_b$. Conviently, the second term that
depends on the variance of the $T_{MRCA}$ in the variance and covariance terms
of the normal distribution does not appear in the distribution of trait
differences.
\begin{align*}
  \Var[Y_a-Y_b] &= Var[Y_a] + \Var[Y_b] - 2\Cov[Y_a,Y_b]\\
                &= 2(\E[T_{MRCA}]\sigma^2 + \Var[T_{MRCA}]\eta) -
                  2(\E[\tau_{a+b}]\sigma^2 + \Var[T_{MRCA}]\eta)\\
                &= 2(\E[T_{MRCA}] - \E[\tau_{a+b}])\sigma^2.
\end{align*}
This also tells us that the distribution of trait differences will be the same
in the infinitesimal limit with and without a low mutation rate approximation.

A much simpler heuristic derivation of the limiting normal distribution can be
done by calculating the variance and covariance at a single locus. This
derivation is very similar to that done by \citet{Schraiber2015}. Using the law
of total variance we can write
\begin{equation*}
  \Var[Y] = \E\left[\Var[Y|T]\right] +
  \Var\left[\E[Y|T]\right]
\end{equation*}
The variance conditional on $T$ can be calculated again using the law of total
variance and conditioning on the number of mutation at the locus. 
\begin{align*}
  \Var[Y|T] &= \E[\Var[Y|M]|T] + \Var[E[Y|M]|T]\\
            &= \E[M(m_2-m_1^2)|T] + \Var[Mm_1|T]\\
            &= \T T (m_2-m_1^2) + \T T m_1^2\\
            &= \T T m_2
\end{align*}
\begin{align*}
  \E[Y|T] = \T T m_1\\
  \Var[\T T m_1] = \left( \T m_1\right)^2\Var[T]
\end{align*}
Therefore we have
\begin{equation}
  \Var[T] = \T m_2 \E[T_{MRCA}] + (\T m_1)^2 \Var[T_{MRCA}].
\end{equation}

The same procedure can be done for the covariance.
\begin{equation*}
  \Cov[Y_a,Y_b] = \E[\Cov[Y_a,Y_b|T]] + \Cov[E[Y_a|T],E[Y_b|T]]
\end{equation*}
We can break $Y_a$ and $Y_b$ into a shared part, $Y_S$ and unshared parts for
each, $Y_{\delta a}$ and $Y_{\delta b}$.
\begin{align*}
  \Cov[Y_S + Y_{\delta a}, Y_S + Y_{\delta b}|T] = \Var[Y_s|T] = \T \tau_{a+b} m_2.\\
  E[\T \tau_{a+b} m_2] = \T m_2\E[\tau_{a+b}]
\end{align*}
\begin{align*}
  \Cov[E[Y_a|T],E[Y_b|T]] = \Cov[\T T m_1, \T T m_1] = \left( \T m_1 \right)^2\Var[T_{MRCA}]
\end{align*}
Therefore we have
\begin{equation}
  \Cov[Y_a,Y_b] = \T m_2 \E[\tau_{a+b}] + \left( \T m_1 \right)^2 \Var[T_{MRCA}].
\end{equation}

%% Here we will show that the distribution of trait values is multivariate normal
%% regardless of the distribution of coalescent times. Recall that the moment
%% generating function for a general distribution of coalescence times is
%% \begin{equation}
%%   \varphi_T(\mathbf{s}) = \int \exp \left( \sum_{\omega \in \mathcal{O}} s_{\omega}t_{\omega} \right)
%%   P(\mathbf{T}=\mathbf{t})d\mathbf{t}.
%% \end{equation}
%% The Taylor series expansion of $\exp \left( s_{\omega}t_{\omega} \right)$ is
%% \begin{equation}
%%   1 + s_{\omega}t_{\omega} + \frac{s_{\omega}^2t_{\omega}^2}{2} + \ldots. \nonumber
%% \end{equation}
%% $\exp \left( \sum_{\omega \in \mathcal{O}} s_{\omega}t_{\omega} \right)$ is therefore
%% \begin{equation}
%%   1 + \sum_{\omega \in \mathcal{O}} s_{\omega}t_{\omega} +
%%   \sum_{\omega_1 \neq \omega_2} s_{\omega_1}t_{\omega_1}s_{\omega_2}t_{\omega_2} + 
%%   \sum_{\omega \in \mathcal{O}} \frac{s_{\omega}^2t_{\omega}^2}{2} + \ldots. \nonumber
%% \end{equation}
%% This of course means that
%% \begin{equation}
%%   \label{eq:mgf_L}
%%   \left(\int \exp \left( \sum_{\omega \in \mathcal{O}} s_{\omega}t_{\omega}
%%   \right)P(\mathbf{T}=\mathbf{t})d\mathbf{t}\right)^L = \left(1 + \sum_{\omega \in \mathcal{O}}
%%   s_{\omega}E[t_{\omega}] + \sum_{\omega_1 \neq \omega_2}
%%   s_{\omega_1}s_{\omega_2}E[t_{\omega_2}t_{\omega_1}] + \sum_{\omega \in
%%     \mathcal{O}} \frac{s_{\omega}^2E[t_{\omega}^2]}{2} + \ldots\right)^L. 
%% \end{equation}
%% Based on equation \ref{eq:sub} we can get the moment generating function for the
%% trait values by making the substitution $s_{\omega}\to \frac{\theta}{2} \left(
%% \psi\left(\sum_{a \in \omega}k_{a}\right) -1 \right)$. We will also take the
%% Taylor series of the moment generating function of the mutational distribution.
%% Noting that $\frac{d^n}{dk^n}\varphi_X(k)\Bigr|_{k=0} = E[X^n]$, if we let $m_1$
%% be the mean and $m_2$ be the second moment of the mutational distribution, then
%% \begin{equation}
%%   \psi\left( \sum_{a \in \omega} k_a \right) = 1 + m_1 \left( \sum_{a \in \omega}
%%   k_a\right) + m_2/2\left( \sum_{a \in \omega} k_a\right)^2 + 
%% m_3/2\left( \sum_{a \in \omega} k_a\right)^3 +\ldots.
%%   \nonumber
%% \end{equation}
%% Substituting this into equation \ref{eq:mgf_L} we get
%% \begin{align}
%%   \biggl( 1 &+ \frac{\theta}{2} \sum_{\omega \in \mathcal{O}} E[t_{\omega}]\left( m_1 \left(
%%   \sum_{a \in \omega} k_a \right) + \frac{m_2}{2} \left( \sum_{a \in \omega}
%%   k_a\right)^2 + \ldots \right) \\
%%   &+ \sum_{\omega_1 \neq \omega_2} E[t_{\omega_1}t_{\omega_2}]\T\left( m_1 \sum_{a \in \omega_1} k_a + \ldots \right)
%%   \T \left( m_1 \sum_{b \in \omega_2} k_b + \ldots \right) \\
%%   &+ \sum_{\omega \in \mathcal{O}} E[t^2_{\omega}] \left(\T\right)^2\left(m_1 \sum_{a \in \omega} k_a +
%%   \ldots \right)^2\biggr)^L.
%% \end{align}
%% Now if we take the limit such that $L\T m_1 \to \mu$ and $L\T m_2\to \sigma^2$
%% as $L \to \infty$ we get
%% \begin{equation}
%%   \exp \left( \frac{\theta}{2} \sum_{\omega \in \mathcal{O}}E[t_{\omega}] \left( \mu \left(
%%   \sum_{a \in \omega} k_a\right) + \frac{\sigma^2}{2}\left( \sum_{a \in \omega}
%%   k_a\right)^2\right)\right),
%% \end{equation}
%% providing that the product of $L$ and moments of the mutational distribution
%% higher than two go to zero. This also requires that the limit of $L(\T)^2m_1$
%% goes to zero as the number of sites goes to infinity. $\mu$ can be interpreted
%% as the rate of change in the mean trait value per generation per genome due to
%% mutational pressure. $\sigma^2$ can be interpreted as the rate of accumulation
%% of variance in trait values per generation per genome. Ignoring terms with
%% $(\T)^2$ is equivalent to the assumption that mutation rates are low enough that
%% only one mutation per locus occurs. It is clear from inspection that this is the
%% moment generating function for a multivariate normal distribution where the
%% expected trait value is $E[T_{MRCA}] \mu$, the variance in a trait value is
%% $E[T_{MRCA}]\sigma^2$, and the covariance in trait values is
%% $(T_{MRCA}-E[\tau_{1,2}]) \sigma^2$. This covariance is proprtional to the
%% expected amount of shared branch length before the most recent common ancestor
%% of the sample.
%%% Local Variables:
%%% TeX-master: "notes.tex"
%%% End:
