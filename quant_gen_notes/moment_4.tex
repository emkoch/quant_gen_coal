Due to the large number of terms I only derive the fourth moment of the trait
value distribution for the case when the mean mutational effect is zero. Having
the mean equal to zero is also helpful when comparing against a normal
distribution because higher order moments of the normal distribution are easy to
calculate when the mean is zero. The terms of \eqref{eq:mgf_approx_sub} that
will appear in the fourth moment after we apply \eqref{eq:deriv} are
\begin{equation*}
  L \left(\T\right) \left(\frac{m_2}{2}\right)
  \sum_{\omega:a\in \omega} E[t_\omega] \left(\sum_{a \in \omega} k_a\right)^4
\end{equation*}
for the fourth moment along one branch,
\begin{equation*}
  \binom{L}{L-2,2,\mathbf{0}}\left(\T\right)^2\left(\frac{m_2}{2}\right)^2
  24 \sum_{\omega:a\in \omega} E[t_\omega]^2 \left(\sum_{a \in \omega} k_a\right)^4
\end{equation*}
for the second moment of the same branch chosen twice, and
\begin{equation*}
  \binom{L}{L-2,1,1,\mathbf{0}}\left(\T\right)^2\left(\frac{m_2}{2}\right)^2
  24 \sum_{\omega_1 , \omega_2: a \in \omega_1 , \omega_2} E[t_{\omega_1}]E[t_{\omega_2}]
  \left(\sum_{a \in \omega_1} k_a\right)^2\left(\sum_{a \in \omega_2} k_a\right)^2
\end{equation*}
for the second moments on two different branches. Taking the fourth derivatives
of these in terms of the desired branch we get
\begin{align}
  \label{eq:mom4}
  E[Y_a^4] &= L\T m_4 E[T_{MRCA}] \nonumber \\
  &+ \frac{L(L-1)}{2} \left(\T\right)^2\left( \frac{m_2}{2} \right)^2
  24 \sum_{\omega:a\in \omega} E[t_\omega]^2 \nonumber \\
  &+ L(L-1) \left(\T\right)^2\left( \frac{m_2}{2} \right)^2
  24 \sum_{\omega_1 , \omega_2: a \in \omega_1 , \omega_2} E[t_{\omega_1}]E[t_{\omega_2}] \nonumber \\
  &= L\T m_4 E[T_{MRCA}] +
  3L(L-1)\left( \T m_2 \sum_{\omega:a\in \omega} E[t_\omega] \right)^2x \nonumber \\
  &= L\T m_4 E[T_{MRCA}] +
  3L(L-1)\left( \T m_2 E[T_{MRCA}] \right)^2 \\
  &\approx L\T m_4 E[T_{MRCA}] +
  3\left(L \T m_2 E[T_{MRCA}] \right)^2 \nonumber 
\end{align}
%%% Local Variables:
%%% TeX-master: "notes.tex"
%%% End:
