The model used above is just one possible way in which mutations can affect
quantitative trait values. This model assumes that the contribution to the
phenotype of an individual from a locus $i$ is a linear combination of the
effects of mutations that have occured at the locus. Moreover, the phenotype of
an individual is a linear combination of all the mutations that have occured in
the genome of that individual. The big assumption is that each mutation at a
locus adds a random effect to the current effect of that locus. This model is
called the stepping stone model and was developed by \citet{Kimura1965}. An
alternative model is one in which each mutation causes the effect of that locus
to be drawn from some distribution, erasing the effects of all other mutations
that had occured at that locus in the past. This mutational model is called the
house of cards model and was developed by \citet{Kingman1978}. 

Here I derive a moment generating function for the house of cards model. This
turns out to have a more complicated form than the stepping stone model because
the effect of a mutation on an individual's phenotype depends on what other
mutations occur at that locus.

As before, the moment generating function is defined as 
\begin{equation}
  \varphi_{\mathbf{Y}}(\mathbf{k}) = E\left[ e^{\mathbf{k} \cdot \mathbf{Y}} \right] = 
  \int e^{\mathbf{k} \cdot \mathbf{Y}} P(\mathbf{Y}=\mathbf{y}) d\mathbf{y}.
\end{equation}
We employ the same tool of conditioning on the genealogy, $\mathbf{T}$, but we
additionally condition on the mutational configuration $\mathbf{M}$. The
mutational configuration is a vector whose entry $m_{\omega}$ is one if a
mutation occurs on branch $\omega$ and zero otherwise. This gives
\begin{align}
  \varphi_{\mathbf{Y}}(\mathbf{k}) &= \int e^{\mathbf{k} \cdot \mathbf{Y}}
  \int \int P(\mathbf{Y}=\mathbf{y} | \mathbf{M}=\mathbf{m}) 
  P(\mathbf{M}=\mathbf{m} | \mathbf{T}=\mathbf{t}) P(\mathbf{T}=\mathbf{t})
  d\mathbf{m} d\mathbf{t} d\mathbf{y}.\\
  &= \sum_{\mathbf{m}} \int e^{\mathbf{k} \cdot \mathbf{Y}} P(\mathbf{Y}=\mathbf{y} | \mathbf{M}=\mathbf{m}) d\mathbf{y} 
  \int P(\mathbf{M}=\mathbf{m}|\mathbf{T}=\mathbf{t}) P(\mathbf{T}=\mathbf{t})d\mathbf{t}
\end{align}
Of course, the phenotype values are conditionally independent of the genealogy
given the mutational configuration. The first integral is the difficult one
because it requires knowing the dependence structure of the phenotypic values
given the mutation configuration. Inidividual phenotypes will be zero if no
mutations occur, independent if mutations occur on different branches, and
completely correlated if mutations occur on the same branch. Additionally, only
the lowest level mutations count. By this we mean for each individual only the
mutations on the branch with the smallest number of descendents count. This can
be written succintly as
\begin{equation}
  \int e^{\mathbf{k} \cdot \mathbf{Y}} P(\mathbf{Y}=\mathbf{y} | \mathbf{M}=\mathbf{m}) d\mathbf{y} = 
  \prod_{\omega} \psi\left(\sum_{i \in \omega} k_i \mathscr{G}(i, \omega, \mathbf{m}) \right).
\end{equation}
Where 
\[
 \mathscr{G}(i,\omega,\mathbf{m}) = 
 \begin{cases}
   1 & \text{if }\omega\text{ is lowest level branch in }\mathbf{m}\text{ containing a mutation subtending }i\\
   0 & \text{otherwise}.
 \end{cases}
\]
The second integral can be written as
\begin{equation}
  \int \prod_\omega \left( m_\omega + (-1)^{m_\omega}e^{-\T t_\omega}\right)P(\mathbf{T}=\mathbf{t})d\mathbf{t}.
\end{equation}
This multiplies the probability that one or more mutations occur along a branch.
We can recognize that this will just be a sum of generating functions for the
genealogy evaluated at different $-\T \mathbf{m}$. It can therefore be written as 
\begin{equation}
  \sum_{\mathbf{m}'}(-1)^{\mathbf{m} \cdot \mathbf{m}'}\mathscr{K}(\mathbf{m}, \mathbf{m}')\varphi_T(-\T \mathbf{m}').
\end{equation}
Where
\[
\mathscr{K}(\mathbf{m}, \mathbf{m}') = 
\begin{cases}
  1 & \text{if } |\mathbf{m}| + \mathbf{m} \cdot \mathbf{m}' = sum(\mathbf{m}) + sum(\mathbf{m}')\\
  0 & \text{otherwise}.
\end{cases}
\]
This function returns one if $\mathbf{m}$ and $\mathbf{m}'$ contain none of the
same zero elements, and zero otherwise. 

The moment generating function overall can therefore be written as
\begin{equation}
  \sum_{\mathbf{m}}\left(\prod_{\omega} \psi\left(\sum_{i \in \omega} k_i 
\mathscr{G}(i, \omega, \mathbf{m}) \right) \right)
\left( \sum_{\mathbf{m}'}(-1)^{\mathbf{m} \cdot \mathbf{m}'}\mathscr{K}(\mathbf{m}, \mathbf{m}')
\varphi_T(-\T \mathbf{m}')\right).
\end{equation}
%%% Local Variables:
%%% TeX-master: "notes.tex"
%%% End: