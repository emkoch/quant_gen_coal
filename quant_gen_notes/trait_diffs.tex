The distribution of a single trait value $Y$ is not very interesting because
this quantity cannot be observed. $Y$ gives the change in the trait value since
the most recent common ancestor of the sample, but since we don't know what that
value is we can't measure $Y$ in a sampled individual. What we can observe is
the differences in trait values between individuals, $Y_a-Y_b$. The results for
these differences are what one would expect given the distribution of the trait
values themselves. For instance, the second and fourth moments look exactly like
\eqref{eq:var} and \eqref{eq:mom4} if one substitutes twice the expected
coalescent time for the time to the most recent common ancestor of the sample.
One way to see this is by deriving the moment generating function for the
difference in trait values between two individuals. Using the same arguments 
that were used to derive \eqref{eq:sub} we get
\begin{align}
  \varphi_{Y_a-Y_b}(k) &= \int \int \prod_{\omega_1 \in \Omega_{a/b}}
                         e^{kY_{a,\omega_1}}P(Y_{a,\omega_1}=y_{a,\omega_1}|\mathbf{T}=\mathbf{t})dy_{a,\omega_1} \nonumber \\
                         &\times \int \prod_{\omega_2 \in \Omega_{b/a}}
                         e^{-kY_{b,\omega_2}}P(Y_{b,\omega_2}=y_{b,\omega_2}|\mathbf{T}=\mathbf{t})dy_{a,\omega_2}
                         P(\mathbf{T}=\mathbf{t})d\mathbf{t}.
\end{align}
From this we have removed all branches where $Y_a-Y_b$ is necessarily zero. This
includes branches that subtend both individuals and branches that subtend
neither individual. From this it is clear, as expected, that the distribution of
the trait difference between two individuals depends only on the distribution of
coalescence times for those individuals. Using the same result for compound
Poisson processes as before we get
\begin{align}
  \varphi_{Y_a-Y_b}(k) &= \int \prod_{\omega_1 \in \Omega_{a/b}} 
                         \exp\left( \T t_{\omega_1} \left( \Phi(k) - 1\right) \right) \nonumber \\
                       &\times \prod_{\omega_2 \in \Omega_{b/a}} 
                         \exp\left( \T t_{\omega_2} \left( \Phi(-k) - 1\right) \right) 
                         P(\mathbf{T}=\mathbf{t})d\mathbf{t}.
\end{align}
It is simple to use the low mutation rate approximation as in
\eqref{eq:mgf_approx_sub} to derive the second and fourth moments of the trait
difference. These are:
\begin{equation}
  E[(Y_a-Y_b)^2]=2L\T m_2 E[\tau_{a/b}].
\end{equation}
and
\begin{equation}
  E[(Y_a-Y_b)^4]=2L\T m_4 E[\tau_{a/b}] + 12L(L-1)(\T m_2 E[\tau_{a/b}])^2.
\end{equation}
The kurtosis is therefore
\begin{equation}
  \label{eq:diff_kurt}
  Kurt[Y_a-Y_b]=\frac{Kurt[M]}{2 L\T m_2 E[\tau_{a/b}]} + 3\left(1-\frac{1}{L} \right).
\end{equation}
Again, the kurtosis depends on the kurtosis of the mutational distribution and
the expected number of mutations affecting the trait. The kurtosis of the
difference between individuals in different populations will be smaller than for
individuals sampled within populations because the coalescence times will be
greater for individuals samples from different populations.
%%% Local Variables:
%%% TeX-master: "notes.tex"
%%% End:
