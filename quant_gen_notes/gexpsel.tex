The kurtosis of the trait distribution will not affect the response to selection
when selection is exponential and directional. However, it will have a potential
effect when selection is stronger than exponential. We represent this by
considering the following fitness function

\begin{equation}
  \label{eq:gexpsel}
  w(z) = e^{sz + \epsilon z^2}.
\end{equation}

We can then calculate the fitness function on genotypes by solving

\begin{equation*}
  w_g(Y) = \int w(Y + x)p_e(x)dx,
\end{equation*}

where $p_e$ is the distribution of environmental noise. Assuming this is normal
with mean zero and variance $V_e$, the genotypic fitness function is

\begin{equation}
  \label{eq:wg}
  w_g(Y) = \left( \sqrt{1-2\epsilon V_e} \right)^{-1}
  \exp\left( \frac{Y(Y\epsilon + s)}{1-2\epsilon V_e}\right).
\end{equation}

This requires that $2\epsilon V_e < 1$. We can then differentiate \eqref{eq:wg}
and plug into \eqref{eq:wbar}, \eqref{eq:l1}, and \eqref{eq:li} to calculate the
selection gradients. Doing so gives

\begin{equation}
  \bar{w} = w_g(\bar{Z})\left( 1 + \frac{V_g}{2}\left( A^2 + B \right) +
  \frac{M_{3,g}}{2}AB + \frac{M_{4.g}}{2}B^2\right),
\end{equation}

\begin{equation}
  L_1 = \frac{A + \frac{3V_g}{2}AB + \frac{M_{3,g}}{2}B^2}{1 + \frac{V_g}{2}\left( A^2 + B \right) +
  \frac{M_{3,g}}{2}AB + \frac{M_{4.g}}{2}B^2},
\end{equation}

\begin{equation}
  L_2 = \frac{B+A^2}{2\left( 1 + \frac{V_g}{2}\left( A^2 + B \right) +
  \frac{M_{3,g}}{2}AB + \frac{M_{4.g}}{2}B^2\right)},
\end{equation}

\begin{equation}
  L_3 = \frac{AB}{2\left( 1 + \frac{V_g}{2}\left( A^2 + B \right) +
  \frac{M_{3,g}}{2}AB + \frac{M_{4.g}}{2}B^2\right)},
\end{equation}

and

\begin{equation}
  L_4 = \frac{B^2}{8\left( 1 + \frac{V_g}{2}\left( A^2 + B \right) +
  \frac{M_{3,g}}{2}AB + \frac{M_{4.g}}{2}B^2\right)}
\end{equation}

Here $A = \frac{2\bar{Y}\epsilon + s}{1 - 2\epsilon V_e}$ and $B =
\frac{2\epsilon}{1 - 2\epsilon V_e}.$ In calculating the selection gradients I
have ignored terms greater than order 2 in $A$ and $B$. Unlike with exponential
directional selection, moments of the breeding values as well as the
environmental variance show up in the selection gradients. We can now use these
results to investigate how much steeper than exponential selection must be in
order for the excess kurtosis to have an effect. To get a sense for this we can
plot the effect of excess kurtosis on the response to selection when the
distribution is normal.

\begin{figure}[H]
  \includegraphics[width=\textwidth]{../programs/sel_resp.pdf}
  \caption{The effect of excess kurtosis on the response to selection when the
    fitness function is steeper than exponential. In this example $s=0.3$, the
    mean breeding value is one, and the heritability is held constant at one half.}
  \label{fig:selresp}
\end{figure}

Figure \ref{fig:selresp} shows that the effect of excess kurtosis on the response
to selection is dependent on the coefficient of variation (CV) of the breeding
values. When the CV is large increasing the excess kurtosis has a larger effect
on the response to selection. This makes sense because the excess kurtosis is
scaled relative to the variance. Increasing the excess kurtosis for a trait with
a higher CV means a larger absolute increase in the fourth moment which
\eqref{eq:dz} shows controls the response to selection. In terms of absolute
increases in the fourth moment, distributions with a lower genetic variance will
have a greater proportional increase in the response to selection. 

Is a selection function with $s=0.3$ and $\epsilon=0.1$ realistic? An individual
with breeding value of one will have an average fitness of about two while an
individual with breeding value three will have an average fitness of about
eleven. An individual with a breeding value of five will have an average fitness
of $175$. This is very strong selection that pushes the bounds of what is
realistic. A feature of the fitness function in \eqref{eq:gexpsel} is that
extreme values will potentially have orders of magnitude higher fitnesses. A
higher kurtosis makes these extreme values more likely. The question is then
whether the sparseness of trait architecture can generate enough kurtosis to
substantially impact the response to selection.

%%% Local Variables:
%%% TeX-master: "notes.tex"
%%% End:
