The kurtosis of the trait distribution will not affect the response to selection
when selection is exponential and directional. However, it will have a potential
effect when selection is stronger than exponential. We represent this by
considering the following fitness function

\begin{equation}
  w(z) = e^{(sz + \epsilon z^2)}.
\end{equation}

We can then calculate the fitness function on genotypes by solving

\begin{equation*}
  w_g(Y) = \int w(Y + x)p_e(x)dx,
\end{equation*}

where $p_e$ is the distribution of environmental noise. Assuming this is normal
with mean zero and variance $V_e$, the genotypic fitness function is

\begin{equation}
  \label{eq:wg}
  w_g(Y) = \left( \sqrt{1-2\epsilon V_e} \right)^{-1}
  \exp\left( \frac{Y(Y\epsilon + s)}{1-2\epsilon V_e}\right).
\end{equation}

This requires that $2\epsilon V_e > 1$. We can then differentiate \eqref{eq:wg}
and plug into \eqref{eq:wbar}, \eqref{eq:l1}, and \eqref{eq:li} to calculate the
selection gradients. 

%%% Local Variables:
%%% TeX-master: "notes.tex"
%%% End:
