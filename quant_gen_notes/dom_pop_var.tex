The variance in trait values between individuals determines the expected
variance in the population. This variance can be written as
\begin{equation*}
  \Var( \sum_{l=1}^L f(Y_{1,l,1},Y_{1,l,2}) - \sum_{l=1}^L f(Y_{2,l,1},Y_{2,l,2}) ).
\end{equation*}
The important term to calculate is
$\Cov(f(Y_{1,l,1},Y_{1,l,2}),f(Y_{2,l,1},Y_{2,l,2})) = \E[f(Y_{1,l,1},Y_{1,l,2}),f(Y_{2,l,1},Y_{2,l,2})]$.
We can use the law of total expectation to calculate this again
conditioning on different mutational configurations. The configurations
that lead to a nonzero expectation if one mutation occurs is present in each individual,
three mutations occur, or four mutations occur. There are four ways for the two individuals
to each have one mutant copy. When this occurs the expected product will be $b^2m_w$. There
are four ways for a mutation to be present in three copies and when this occurs the expected
product will be $2b m_2$. There is one way to have a mutation present in all four copies
and the expected product is $4m_2$.

We can then us the probability of each type of mutational configuration occurring
to calculate the covariance. 

%%% Local Variables:
%%% TeX-master: "notes.tex"
%%% End:
