The expected variance of breeding values in the population, and what in this
model would be considered the genetic variance is

\begin{equation}
  \label{eq:popvar}
  E[V_g] = E\left[ \frac{1}{N}\sum \left( Y_i - \frac{\sum Y_j}{N} \right)^2\right].
\end{equation}

By assuming that all the individuals in the sample are exchangeable, we can
expand the inside to get

\begin{align*}
  E\left[ \left(  Y_i - \frac{\sum Y_j}{N} \right)^2 \right] &= 
                                                             E\left[ Y_i^2 - 2Y_i\frac{\sum Y_j}{N} + \left( \frac{\sum Y_j}{N} \right)^2 \right] \\
                                                           &= E[Y_i^2] - \frac{2}{N} E[Y_i^2] - 
                                                             \frac{2(N-1)}{N} E[Y_iY_j] + \frac{1}{N} E[Y_i^2] + 
                                                             \frac{N-1}{N} E[Y_iY_j] \\
                                                           &= \frac{N-1}{N} \left( E[Y_i^2] - E[Y_iY_j]\right).
\end{align*}

Thus when $N$ is even moderately large

\begin{equation}
  E[V_g] = E[Y_i^2] - E[Y_iY_j].
\end{equation}

When $m_1=0$, $E[Y_i^2] \approx L \T m_2 E[T_{MRCA}]$ and
$E[Y_iY_j] \approx L \T m_2 ( E[T_{MRCA}] - E[\tau_{i,j}])$. The expected
genetic variance in the population is then proportional to the expected pairwise
coalescent time as we would expect.
%%% Local Variables: 
%%% mode: latex
%%% TeX-master: "notes.tex"
%%% End: 
