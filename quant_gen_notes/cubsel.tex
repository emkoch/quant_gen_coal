The population kurtosis may become more important if we want to consider
selection where the fitness function is cubic on the breeding values. Such a
fitness function would have the form
\begin{equation}
  \label{eq:cubsel}
  W_g(Y) = b_0 + b_3(Y-\bar{Y})^3.
\end{equation}
This fitness function now models selection on differences from the current mean
breeding value in the population. This is not the most realistic scenario
because this mean will have diverged since the most recent common ancestor of
the population in the absence of stabilizing selection. The reference point
which selection sees as the center is therefore somewhat arbitrary. There's no
good reason for this to be the population mean, but it's also not totally
unreasonable. Another thing we need to consider about the cubic selection
function is that it can give negative values for fitness, which is something we
clearly don't want. We therefore must assume that the degree of cubic selection,
$b_3$ is not too large relative to fitness value at the population mean, $b_0$.
Figure \ref{fig:cubshape} shows a few examples of cubic fitness functions
applied to a simulated population with mean breeding value around four.
Selection gets very steep as $b_3$ increases and one ends up with negative
fitness values that are then set to zero

\begin{figure}
  \centering
  \includegraphics[width=0.6\textwidth]{cub_shape.pdf}
  \caption{Cubic fitness functions with $b_0=50$ applied to a random population
    with 2000 individuals, 100 loci, and a standard deviation of mutational
    effects of one.}
  \label{fig:cubshape}
\end{figure}

Note that we're now considering selection directly on breeding values rather
than starting with selection on phenotypic values and deriving selection on
breeding values assuming a Gaussian distribution for environmental effects. We
might justify this by saying that this should capture the effect of that portion
of the overall fitness function which has a cubic shape with respect to the
breeding values.

Applying \eqref{eq:Li}, \eqref{eq:wbar}, and \eqref{eq:wbar} to
\eqref{eq:cubsel} we get
\begin{equation}
  \bar{w} = b_0 + M_{3,g}b_3,
\end{equation}
\begin{equation}
  L_1 = \frac{3V_gb_3}{\bar{w}},
\end{equation}
\begin{equation}
  L_2 = 0,
\end{equation}
\begin{equation}
  L_3 = \frac{b_3}{\bar{w}}.
\end{equation}
After combining these all into \eqref{eq:dz} we get
\begin{equation}
  \label{eq:cubresp}
  \Delta \bar{Z} = \frac{3 V_g^2b_3 + (M_{4,g}-3V_g^2)b_3}{b_0 + M_{3,g}b_3} =
  \frac{M_{4,g}b_3/b_0}{1 + M_{3,g}b_3/b_0} = \frac{M_{4,g}\beta}{1 +
    M_{3,g}\beta}.
\end{equation}
We can see that the response to cubic selection depends only on $\beta$, the
ratio of the steepness of cubic selction to the baseline fitness, as well as the
third and fourth moments of the distribution of breeding values. In general the
response to cubic selection is linear with respect to the fourth moment of the
distribution of breeding values and hence the kurtosis.

Since the theory of \citet{Turelli1990} is for weak selection, we must
investigate the range of selection strengths for which \eqref{eq:cubsel}
predicts the average response to selection. To do this, for a range of $\beta$
values I simulated 50 populations with 2000 haploid individuals each and a
mutational standard deviation of one. I drew individuals for the next generation
according to fitness values from \eqref{eq:cubsel} and compared the change in
phenotype to that predicted by \eqref{eq:cubresp}. The results of this
experiment are shown in Figure \ref{fig:incrcub}. The weak selection
approximations of \citet{Turelli1990} begin to break down for $\beta$ greater
than $0.001$, which as Figure \ref{fig:cubshape} shows, corresponds to quite
strong selection. 

\begin{figure}
  \centering
  \label{fig:incrcub}
  \includegraphics[width=\textwidth]{cubic_sel.pdf}
    \caption{Simulation results comparing the response to varying strengths of
    cubic selection to the expected response under weak selection theory given
    by \eqref{eq:cubresp}. Simulation realizations for a given $\beta$ have
    different responses to selection because of differences in the fourth moment
    of breeding values arising by chance. Lines show the one-to-one relationship
    for comparison.}
\end{figure}
%%% Local Variables:
%%% TeX-master: "notes.tex"
%%% End:
