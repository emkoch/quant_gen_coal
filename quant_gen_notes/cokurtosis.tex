Similarly to kurtosis, we can calculate another fourth order moment called the
cokurtosis which measures the propensity of extreme values to occur together in
the joint distribution. The cokurtosis is defined as
\begin{equation}
  \label{eq:cokurtosis_def}
  Cokurt(X,X,Y,Y)=\frac{E[(X-E[X])^2(Y-E[Y])^2]}{\sigma_X^2\sigma_Y^2}.
\end{equation}
We will only look at the balanced version of this for now. As with covariance,
the kurtosis of a random variable is its cokurtosis with itself. Courtesies may
be of interest residual signal of individual genealogies may cause individual to
share extreme trait values due to shared ancestry. We calculate the cokurtosis
by first calculating $E[Y_a^2Y_b^2]$ in the same way as was done for the
kurtosis by considering different terms of \eqref{eq:mgf_approx_sub}. One is the
fourth moment term for branches containing both $a$ and $b$:
\begin{equation*}
  L \T \frac{m_4}{4!} \sum_{\omega : a, b \in \omega} E[t_\omega] \left( \sum_{d \in \omega} k_d \right)^4.
\end{equation*}
Another is the second moment terms for pairs of branches both containing $a$ and $b$:
\begin{equation*}
  \frac{L(L-1}{2} \left( \sum_{\omega : a, b \in \omega}
  E[t_\omega] \T \frac{m_2}{2} \left( \sum_{d \in \omega} k_d \right)^2right).
\end{equation*}
Let $\Omega_{a/b}$ be the set of branches containing $a$ but not $b$ and
$\Omega_{a+b}$ be the set of branches containing both $a$ and $b$. We also need
to consider terms from $\Omega_{a/b}\times\Omega_{b}$ and
$\Omega_{b/a}\times\Omega_{a}$. These are
\begin{equation}
  L(L-1) \left( \sum_{\omega_1 \in \Omega_{a/b}} E[t_{\omega_1}] \T \frac{m_2}{2}
  \left( \sum_{d_1 \in \omega} k_{d_1} \right)^2 \right)
  \left( \sum_{\omega_2 \in \Omega_{b}} E[t_{\omega_2}] \T \frac{m_2}{2}
  \left( \sum_{d_2 \in \omega} k_{d_2} \right)^2 \right),
\end{equation}
and it's equivalent. Taking the appropriate derivatives we get
\begin{align*}
  E[Y_a^2Y_b^2] &= L \T m_4 \sum_{\omega \in \Omega_{a+b}} E[t_\omega] \\
  &+ 3L(L-1) \left( \sum_{\omega \in \Omega_{a+b}} \T m_2 E[t_\omega] \right)^2\\
  &+ L(L-1) \left( \sum_{\omega_1 \in \Omega_{a/b}} E[t_{\omega_1}] \T m_2  \right)
  \left( \sum_{\omega_2 \in \Omega_{b}} E[t_{\omega_2}] \T m_2 \right)\\
  &+ L(L-1) \left( \sum_{\omega_1 \in \Omega_{b/a}} E[t_{\omega_1}] \T m_2  \right)
  \left( \sum_{\omega_2 \in \Omega_{a}} E[t_{\omega_2}] \T m_2 \right)
\end{align*}
One way to simplify this is to notice that the last two lines almost sum over
$\Omega_a\times\Omega_b$ except that they are missing the product over terms
where both branches contain both $a$ and $b$. If we subtract one of the three
terms from the second line and add it here we get
\begin{align*}
  E[Y_a^2Y_b^2] &= L \T m_4 \sum_{\omega \in \Omega_{a+b}} E[t_\omega] \\
  &+ 2L(L-1) \left( \sum_{\omega \in \Omega_{a+b}} \T m_2 E[t_\omega] \right)^2\\
  &+ L(L-1) \left( \sum_{\omega_1 \in \Omega_{a}} E[t_{\omega_1}] \T m_2  \right)
  \left( \sum_{\omega_2 \in \Omega_{b}} E[t_{\omega_2}] \T m_2 \right),
\end{align*}
which can be rewritten as
\begin{equation}
  \label{eq:cross4}
  E[Y_a^2Y_b^2] = L \T m_4 E[\tau_{a+b}]+2L(L-1) \left(\T m_2 E[\tau_{a+b}]\right)^2
  + L(L-1) \left( \T m_2 E[\tau_a]\right)\left( \T m_2 E[\tau_b]\right).
\end{equation}
Since $E[\tau_a]$ is the expected length of branches in the entire genealogy
containing $a$, this is also the same as $E[T_{MRCA}]$.
%%% Local Variables:
%%% TeX-master: "notes.tex"
%%% End:
