After introducing Kimura's integro-differential equation for the evolution of the distribution of a quantitive trait in
population (equation IV.1.3), B{\"u}rger states that a simple property of the equilibrium solution is that
\begin{equation}
-\mu \leq \hat{\bar{m}} < 0.
\end{equation}
This follows first from the assumption that the mutational distribution and the fitness function are symmetric about
zero, so the mean \emph{phenotype} in the population will be zero.  Since the fitness function, $m(x)$ is always
negative, $\hat{\bar{m}} < 0$. The $-\mu \leq \hat{\bar{m}}$ half follows from
\begin{equation}
p(x)\left[ \mu + \hat{\bar{m}} - m(x) \right] = \mu \int u(x-y) p(y) dy \geq 0,
\end{equation}
\begin{equation}
\mu + \hat{\bar{m}} = \mu \int u(x-y) p(y) dy / p(x) + m(x).
\end{equation}
Because $m(x) \leq 0$, 
\begin{equation}
\mu + \hat{\bar{m}} \geq \mu \int u(x-y) p(y) dy / p(x) \geq 0.
\end{equation}
Which means $-\mu \leq \hat{\bar{m}}$. The biological implication of this is that the mean decrease in fitness due to
mutation in a population under stabilizing selection is less than the mutation rate. Since we think of per-locus
mutation rates as being quite small, this seems to imply a low per-locus genetic load. This is reminiscent of classic
results saying that the genetic load depends only on the mutation rate up to an approximation.

Next, B{\"u}rger introduces two approximations to the solution of equation IV.1.3:
\begin{equation}
\frac{\partial p(x,t)}{\partial t} = \left[ m(x) - \bar{m}(t) \right]p(x,t) + \mu \left[ \int u(x-y)p(y,t)dy -p(x,t)\right].
\end{equation}
The first approximation is from \citet{Kimura1965} and called the \emph{Gaussian Allelic Approximation}. Starting with
the equilibrium condition
\begin{equation} \label{eq:equi}
\left[ \mu + \bar(m) - m(x) \right] = \mu \int u(x-y) p(y)dy, 
\end{equation}
Kimura performs the transformation $y \to y + x$ to get
\begin{equation} \label{eq:trans}
\int u(x-y) p(y) dy = \int u(-y)p(y+x)dy.
\end{equation}
Taking the Taylor series around $x$, 
\begin{equation}
p(y+x) \approx p(x) + \frac{dp(x)}{dx}\left( x-(y+x) \right) + \frac{1}{2} \frac{d^2p(x)}{dx^2}\left( x-(y+x) \right)^2
\end{equation}
we get
\begin{equation}
\int u(-y)p(y+x)dy \approx p(x) + \frac{dp(x)}{dx} \int u(-y)y dy + \frac{1}{2} \frac{d^2p(x)}{dx^2}\int u(-y)y^2dy.
\end{equation}
when plugging this into equation \ref{eq:trans}. This approximation will be good if the higher order terms of the
mutational distribution, $u$ are small. If we assume that $u$ has mean zero and variance $\gamma^2$ the plugging in to 
equation \ref{eq:equi} gives
\begin{equation}
s\left( x^2 - \int y^2p(y)dy\right)p(x) = \frac{1}{2}\mu\gamma^2\frac{d^2p(x)}{dx^2}
\end{equation}
after some rearranging. Kimura looked up the solution to such an equation, which happens to be Gaussian with mean zero
and variance $\sqrt{\frac{\mu}{s}\frac{\gamma^2}{2}}$. This is interesting because phenotypic distribution depends on
the mutation rate, mutational variance, and strength of selection.

The second equilibrium approximation is from \citet{Turelli1984}, which makes the approximation 
\begin{equation}
\int u(x-y)\hat{p}(y)dy \approx u(x), 
\end{equation}
which corresponds to the variance of the mutational distribution being much greater than that of the equilibrium
phenotypic distribution. If this is true then 
\begin{equation}
\hat{p}(x) = \frac{\mu m(x)}{\mu + \hat{\bar{m}} - m(x)}.
\end{equation}
All that remains is to find $\hat{\bar{m}}$ such that this integrates to one. Using the quadratic fitness function
$m(x)=-sx^2$ and $\mu/(s\gamma^2) \to 0$, \citet{burger2000mathematical} finds the variance of this distribution to be
\begin{equation}
\sigma^2 = \frac{\mu}{s} - \frac{\pi}{2\gamma^2}\left( \frac{\mu}{s^2}\right), 
\end{equation}
Although the further approximation $\sigma^2=\mu/s$ is what Turelli originally found
\subsubsection{Thoughts on this stuff}
All of this is very interesting, but the obvious question comes up about how these models behave when extended to
multiple loci as is more realistic for most traits. Towards that end it will be interesting to read \citet{Lande2007}.
An additional question is how to intuitively understand the differences between these models and what that means in terms 
of the empirical study of quantitative characters. 

foo
