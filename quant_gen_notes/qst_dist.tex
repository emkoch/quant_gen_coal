$Q_{ST}$ is defined as 
\begin{equation}
  \label{eq:qst}
  Q_{ST} := \frac{V_{\text{between}}}{V_{\text{between}} + V_{\text{within}}}.
\end{equation}
This is the proportion of the total phenotypic variance in the population that is due to differences between
populations. As a sample statistic, $Q_{ST}$ reflects the amount of phenotypic differentiation between population. We
would use an estimator of the between population variance, $\hat{V}_{\text{between}}$, and similarly
$\hat{V}_{\text{within}}$ for the within population variance. Let $Y_{ij}$ be the phenotype of individual $j$ from
population $i$, $\bar{Y}_i$ be the mean of individuals in population $i$, and $\bar{Y}$ be the mean phenotype of all
individuals in the sample. An estimate of the between population variance would then be
\begin{equation}
  \label{eq:between}
  \hat{V}_{\text{between}} = \sum_i \frac{(\bar{Y}_i-\bar{Y})^2}{K-1}.
\end{equation}
We want to calculate the sampling distribution of this statistic. $\bar{Y}_i$ and $\bar{Y}$ are both going to be
normally distributed because the individual phenotypes are normal. Their difference will therefore also be normal and
the square of this will distributed as the variance of the difference times a chi-squared random variable with one
degree of freedom. We thus want to calculate the variance of $(\bar{Y}_i-\bar{Y})^2$. 
\begin{equation}
  \label{eq:popmean}
  \bar{Y}_i = \frac{\sum_jY_{ij}}{n_i}
\end{equation}
\begin{equation}
  \label{eq:grandmean}
  \bar{Y} = \frac{\sum_i\sum_jY_{ij}}{\sum_in_i}
\end{equation}
What we want is
\begin{equation}
  \label{eq:varexpand}
  Var[\frac{\sum_jY_{ij}}{n_i} - \frac{\sum_i\sum_jY_{ij}}{\sum_in_i}] = 
  Var[\frac{\sum_jY_{ij}}{n_i}] + Var[\frac{\sum_i\sum_jY_{ij}}{\sum_in_i}] - 2 Cov[\frac{\sum_jY_{ij}}{n_i}, \frac{\sum_i\sum_jY_{ij}}{\sum_in_i}].
\end{equation}
What we will see is that, because each $Var[Y_{ij}] \propto E[T_{MRCA}]$ and
$Cov[Y_{ij},Y_{kl}] \propto E[T_{MRCA}] - E[\tau_{i,k}]$, that the expected time to the most recent common ancestor will
cancle out and the variance will only depend on the expected pairwise coalescence times. For the first term,
\begin{align}
  Var[\frac{\sum_jY_{kj}}{n_k}] &= \frac{1}{n_k^2}Var[\sum_j Y_{kj}] \nonumber \\
                                &= \frac{1}{n_k^2} \left( n_k Var[Y_{kj}] + n_k(n_k-1) Cov[Y_{ki},Y_{kj}] \right) \nonumber \\
                                &\propto \frac{1}{n_k^2} \left( n_k E[T_{MRCA}] + n_k(n_k-1) (E[T_{MRCA}-E[\tau_{k,k}]) \right) \nonumber \\
                                &= E[T_{MRCA}] - \frac{n_k-1}{n_k}E[\tau_{k,k}].
\end{align}
For the second term,
\begin{align}
  Var[\frac{\sum_i\sum_jY_{ij}}{\sum_in_i}] &= \frac{1}{(\sum_in_i)^2} \left( \sum_i\sum_j\sum_l\sum_m Cov[Y_{ij},Y_{lm} ]\right) \nonumber \\
                                            &\propto \frac{1}{(\sum_in_i)^2} \left( \left(\sum_i n_i\right)^2 E[T_{MRCA}] - \sum_in_i\left(\sum_ln_lE[\tau_{i,l}] - E[\tau_{i,i}]\right) \right) \nonumber \\
                                            &= E[T_{MRCA}] - \frac{1}{(\sum_in_i)^2} \left(\sum_in_i\left(\sum_ln_lE[\tau_{i,l}] - E[\tau_{i,i}]\right) \right).
\end{align}
For the third term,
\begin{align}
  Cov[\frac{\sum_jY_{ij}}{n_i}, \frac{\sum_i\sum_jY_{ij}}{\sum_in_i}] &= \frac{1}{n_k\sum_in_i}Cov[\sum_kY_{kj},\sum_i\sum_lY_{il}] \nonumber \\
                                                                      &\propto \frac{1}{n_k\sum_in_i} \left( n_k\sum_in_iE[T_{MRCA}] - n_k\sum_in_iE[\tau_{k,i}] + n_kE[\tau_{k,k}] \right) \nonumber \\
                                                                      &=  E[T_{MRCA}] - \frac{1}{\sum_i n_i} \left( \sum_i n_i E[\tau_{k,i}] - E[\tau_{k,k}] \right)
\end{align}
We can that the $E[T_{MRCA}]$ terms cancel out and that we end up with 
\begin{align}
  \label{eq:diffvar}
  Var[\bar{Y}_k-\bar{Y}] &= \frac{2}{\sum_i n_i} \left( \sum_i n_i E[\tau_{k,i}] - E[\tau_{k,k}] \right) - \frac{n_k-1}{n_k}E[\tau_{k,k}] \nonumber \\
                         &- \frac{1}{(\sum_in_i)^2} \left(\sum_in_i\left(\sum_ln_lE[\tau_{i,l}] - E[\tau_{i,i}]\right)\right)
\end{align}
We will return to the interpretation of this later. For the within population variance an estimator is 
\begin{equation}
  \label{eq:within}
  \hat{V}_{\text{within}}=\frac{\sum_i\sum_j(Y_{ij}-\bar{Y}_i)^2}{N-K}.
\end{equation}
Doing the same thing as for the between population variance we get
\begin{equation}
  \label{eq:withinprop}
  Var[Y_{ij} - \frac{\sum_lY_{il}}{n_i}] \propto \frac{n_i-1}{n_i}E[\tau_{i,i}].
\end{equation}
We can then use these to derive what the sampling distribution for this estimator of $Q_{ST}$ should be. 
%%% Local Variables: 
%%% mode: latex
%%% TeX-master: "notes.tex"
%%% End: 