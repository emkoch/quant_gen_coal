We look at an example from a panmictic population to see how the genealogy generates a correlation structure among the
samples. In particular, we take limits corresponding to an infinitesimal model and show that this results in multivariate 
normal distribution. 

Recall that the moment generating function of a mutational distribution $f(u)$ can be written as 
\begin{equation*}
  \psi_{U}(k) = 1 + kE\left[ U \right] + \frac{k^2}{2!}E\left[ U^2 \right] + \frac{k^3}{3!}E\left[ U^3\right] + \ldots
\end{equation*}
We are going to assume that the mean mutational effect is zero, and therefore $E[U^2]=Var[U]:=\tau^2$. We then require
that $\lim_{L \to \infty}\tau^2L \to \sigma^2$ and $\lim_{L \to \infty}E[U^k] \to 0$ for $k>2$. This roughly corresponds
to a situation where there are very many loci each with small mutational contributions, and that the effects of
mutations are not too skewed. 

Recalling that the moment generating function for a sum of independent random variables is the product of their
generating functions, the moment generating function for a sample of size two in a panmictic population for a trait
controlled by $L$ loci is
\begin{equation}
  \label{eq:two}
  \varphi(k_1,k_2) = \left( \frac{1}{1-\frac{\theta}{2}(\psi(k_1)+\psi(k_2)-2)}\right)^L.
\end{equation}
If we rewrite this as
\begin{equation*}
  \left( 1 + \frac{\frac{\theta}{2}L(\psi(k_1) + \psi(k_2) -2)}{L-\frac{\theta}{2}L(\psi(k_1) + \psi(k_2) -2)}\right)^L,
\end{equation*}
it's simple to show that in the limit as $L \to \infty$ we get 
\begin{equation}
  \label{eq:limexp}
  \exp\left( \frac{\theta}{2} \left( \frac{k_1^2}{2}\sigma^2 + \frac{k_2^2}{2}\sigma^2 \right)\right).
\end{equation}
We can recognize this as the generating function of a multivariate normal distribution, meaning that 
\begin{equation}
  \label{eq:twodist}
  (Y_1,Y_2) \sim \mathcal{N}\left( \mathbf{0},
    \begin{pmatrix}
      \frac{\theta}{2}\sigma^2 & 0 \\
      0 & \frac{\theta}{2}\sigma^2
    \end{pmatrix}
  \right).
\end{equation}

We next extend this logic to a sample of size three to see how shared branches induce correlations.
%%% Local Variables: 
%%% mode: latex
%%% TeX-master: "notes.tex"
%%% End: 
