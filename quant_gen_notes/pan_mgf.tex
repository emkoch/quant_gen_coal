We look at an example from a panmictic population to see how the genealogy generates a correlation structure among the
samples. In particular, we take limits corresponding to an infinitesimal model and show that this results in multivariate 
normal distribution. 

Recall that the moment generating function of a mutational distribution $f(u)$ can be written as 
\begin{equation*}
  \psi_{U}(k) = 1 + kE\left[ U \right] + \frac{k^2}{2!}E\left[ U^2 \right] + \frac{k^3}{3!}E\left[ U^3\right] + \ldots
\end{equation*}
We are going to assume that the mean mutational effect is zero, and therefore $E[U^2]=Var[U]:=\tau^2$. We then require
that $\lim_{L \to \infty}\tau^2L \to \sigma^2$ and $\lim_{L \to \infty}E[U^k] \to 0$ for $k>2$. This roughly corresponds
to a situation where there are very many loci each with small mutational contributions, and that the effects of
mutations are not too skewed. 

Recalling that the moment generating function for a sum of independent random variables is the product of their
generating functions, the moment generating function for a sample of size two in a panmictic population for a trait
controlled by $L$ loci is
\begin{equation}
  \label{eq:two}
  \varphi(k_1,k_2) = \left( \frac{1}{1-\frac{\theta}{2}(\psi(k_1)+\psi(k_2)-2)}\right)^L.
\end{equation}
If we rewrite this as
\begin{equation*}
  \left( 1 + \frac{\frac{\theta}{2}L(\psi(k_1) + \psi(k_2) -2)}{L-\frac{\theta}{2}L(\psi(k_1) + \psi(k_2) -2)}\right)^L,
\end{equation*}
it's simple to show that in the limit as $L \to \infty$ we get 
\begin{equation}
  \label{eq:limexp}
  \exp\left( \frac{\theta}{2} \left( \frac{k_1^2}{2}\sigma^2 + \frac{k_2^2}{2}\sigma^2 \right)\right).
\end{equation}
We can recognize this as the generating function of a multivariate normal distribution, meaning that 
\begin{equation}
  \label{eq:twodist}
  (Y_1,Y_2) \sim \mathcal{N}\left( \mathbf{0},
    \begin{pmatrix}
      \frac{\theta}{2}\sigma^2 & 0 \\
      0 & \frac{\theta}{2}\sigma^2
    \end{pmatrix}
  \right).
\end{equation}

We next extend this logic to a sample of size three to see how shared branches induce correlations. The moment
generating function for this case is
\begin{align}
  \label{eq:three}
  \varphi(k_1,k_2,k_3) &= \Biggl( \frac{1}{3-\T\left( \psi(k_1) + \psi(k_2) + \psi(k_3) -3 \right)} \nonumber \\
                       &\times \biggl[ \frac{1}{ 1-\T \left( \psi(k_1+k_2) + \psi(k_3) - 2 \right) } \nonumber\\
                       &+ \frac{1}{1-\T\left( \psi(k_1+k_3) + \psi(k_2) - 2  \right)} \nonumber \\ 
                       &+ \frac{1}{1-\T\left( \psi(k_2+k_3) + \psi(k_1) -2 \right)} \biggr] \Biggr)^L
\end{align}
We can already see that this is a product of two terms, and we can likely guess that it represents a convolution of two
normal distributions. Rearranging this equation and simplifying notation a bit gives
\begin{align}
  \varphi(k_1,k_2,k_3) &= \Biggl( \frac{1}{1-\frac{1}{3}\T (\psi_1 + \psi_2 + \psi_3 -3)} \nonumber \\
                       &\times \biggl[ 1 + \frac{\frac{1}{3}\T(\psi_{12} + \psi_3-2)}{1-\T(\psi_{12} + \psi_3-2)} \nonumber \\
                       &+ \frac{\frac{1}{3}\T(\psi_{13} + \psi_2-2)}{1-\T(\psi_{13} + \psi_2-2)} \nonumber \\
                       &+ \frac{\frac{1}{3}\T(\psi_{23} + \psi_1-2)}{1-\T(\psi_{23} + \psi_1-2)}\biggr] \Biggr)^L.
\end{align}
In the limit as $L\to\infty$ we get
\begin{align}
  \varphi(k_1,k_2,k_3) &= \exp\left( \frac{1}{3}\T\sigma^2(k_1^2 + k_2^2 + k_3^2)\right) \nonumber \\
                       &\times \exp\left( \frac{1}{3}\T\sigma^2 (3k_1^2 + 3k_2^2 + 3k_3^2 + 2k_1k_2 + 2k_1k_3 + 2k_2k_3) \right).
\end{align}
This is the moment generating function of multivariate normal with means $0$, variance $\T\sigma^2\frac{4}{3}$, and
covariance $\T\sigma^2\frac{1}{3}$. Of course, we can recognize that $\frac{4}{3}$ is the expected TMRCA, and
$\frac{1}{3}$ is the expected shard branch length between two samples. Applying this principle more generally, in a
sample of size $n$ the expected TMRCA will be $2(1-\frac{1}{n})$ and the expected amount of shared branch length will be
\begin{equation}
  \label{eq:shared}
  E[T_{shared}]=2\sum_{i=1}^n\left( \frac{1}{i-1} - \frac{1}{i}\right)\left(1- \prod_{j=i+1}^n\left(1-\frac{2}{j(j-1)} \right)\right)=1-\frac{2}{n}.
\end{equation}
I just solved this with Mathematica, but it should be possible to find this simple result by closer inspection. We
should also get the same result from the generating fuction. The simplicity here is hopefully inspiring to be able to
obtain a nice result in the structured population case. Interestingly, these values quickly converge as the sample size grows
%%% Local Variables: 
%%% mode: latex
%%% TeX-master: "notes.tex"
%%% End: 
