
\documentclass{article}
\usepackage{amsmath}
\usepackage{natbib}
\usepackage[textwidth=5.8in, textheight=8in]{geometry}
\usepackage{lipsum}
\usepackage[mathscr]{eucal}
\usepackage{graphicx}

\newcommand{\T}{\frac{\theta}{2}}
\newcommand{\E}{\mathrm{E}}
\newcommand{\Var}{\mathrm{Var}}
\newcommand{\Cov}{\mathrm{Cov}}
\newcommand{\Pro}{\mathrm{P}}
\newcommand{\Kurt}{\mathrm{Kurt}}

\begin{document}

\title{Quantitative trait distributions and the coalescent process}
\maketitle

\section{Introduction}

Evolutionary quantitative genetics almost always models phenotypes as normally
distributed. Indeed, it has been suggested that this is the defining
characteristic of the field \citep{Rice2004}. In this framework the phenotypic
distribution at all levels between individuals, families, populations, and
species is completely described by of a set of means, variances, covariances
\citep{Falconer1996}. Under the infinitesimal model that began with
\citet{Fisher1919}, the phenotypic trait is controlled by a very large number of
loci and the normality assumption is well-justified.

The infinitesimal-normal model has been a powerful tool in many areas of evolutionary
genetics. Among its many successes, it has been used to find traits under
selection in natural populations \citep{Price1984}, determine which dimensions
in phenotype space are favored by females during sexual selection
\citep{Blows2004}, and test for phylogenetic signal in phenotypic evolution
\citep{Freckleton2002}. Such studies and other provide great biological insights
without worrying about potentially complex evolutionary dynamics at multiple
loci because the entire system can be described in terms of the first and second
order moments.

However, heritable phenotypic variation is ultimately due to discrete mutations
at discrete loci, and how this variation is distributed depends on the
genealogies at these loci. Classical neutral models of phenotypic evolution
incorporate the effects of factors like population size and subdivision on
genealogies and therefore phenotypic variation \citep{Lynch1986,Lande1992}.
These models have examined the dynamics of phenotypic evolution forwards in time
as a balance between mutation that creates variance, mutation that spreads it
among subpopulations, and fixation that removes it. The focus has been to find
equilibrium genetic variances and the rate at which these are approached.

The principle modeling framework for variation in genealogies is the coalescent
process with all its varied extensions \citep{Wakeley2008}, but few studies have
attempted to connect this theory with quantitative genetics. The most well known
of these is \citet{Whitlock1999}, who used a coalescent argument to show that
$Q_{ST}$ and $F_{ST}$ have the same expected value under general models of
population subdivision. \citet{Griswold2007} used coalescent simulations to
investigate the effects of shared ancestry and linkage disequilibrium on the
matrix of genetic variances and covariances between traits
($\mbox{\textbf{G}}$). They found that the eigenvalues of the expected matrix
and the expected eigenvalues can be quite different. \citet{Ovaskainen2011} used the
older (but deeply related \citep{Tavare1984}) formalism of coancestry
coefficients to model the full distribution of correlated traits in related
subpopulations. Their main result is that the covariance in additive genetic
values, conditional on the G-matrix in the ancestral population, depends only on
the set of pairwise coancestry coefficients.

Most recently, \citet{Schraiber2015} developed a very general model of
quantitative trait evolution based on the coalescent that allows for an
arbitrary distribution of mutational effects and number of loci affecting the
trait. They derive the characteristic function for the distribution of
phenotypic values in a sample and show how such values can deviate strongly from
normality when the number of loci is small or the mutational distribution has
fat tails.

The results of \citet{Schraiber2015} were derived for a panmictic, constant-size
population. Here, I generalize their work to an arbitrary distribution of
coalescent times for which their main result, the characteristic function of the
sampling distribution of phenotypic values, is a special case. I then show how
the normal, infinitesimal model arises by taking appropriate limits. I then
calculate the kurtosis of the phenotypic distribution to illustrate the
dependence of departures from normality on both the mutational and genealogical
distributions.

\section{Notation}
Before deriving anything, it is helpful to describe the various quantities of
interest and the notation used to describe them. The principle quantity of
interest are the quantitative trait values. Hereafter I'll simply refer to these
as trait values even though it might be more precise to call them breeding
values or additive genetic values. The random vector of trait values in the
sampled individuals is $\mathbf{Y}$, and an element $Y_a$ of $\mathbf{Y}$ is the
trait value in individual $a$. An important thing to remember about the trait
values is that they really refer to the change in the trait value since the most
recent common ancestor of the sample. Since we do not know what the trait value
of this ancestor was, these trait values are not directly observed. Rather, they
determine other summaries one might calculate from a sample of measured
phenotypes.

The random vector of branch lengths describing the genealogy at a locus is
$\mathbf{T}$, and an element $T_{a,b}$ of $\mathbf{T}$ is the branch length
subtending individuals $a$ and $b$. $\omega$ will be used to denote a particular
set of individuals and hence a branch on the genealogy, and $\Omega$ is the set
of all possible branches. For instance, if there are three sampled individuals,
$a$, $b$, and $c$, then $\mathbf{Y}=\{Y_a,Y_b,Y_c\}$,
$\mathbf{T}=\{T_a,T_b,T_c,T_{a,b},T_{a,c},T_{b,c}\}$, and
$\Omega=\{(a),(b),(c),(a,b),(a,c),(b,c)\}$. Realizations of the random trait
values and branch lengths use the same notation with lowercase letters.
$T_\omega$ is then length of branch $\omega$.

Another useful piece of notation is for sums of branch lengths. Let $\tau_{a+b}$
be the random sum of all branches containing both $a$ and $b$. Conversely,
$\tau_{a/b}$ will be the random sum of all branches containing $a$ and not $b$.
Extensions of this for more than two individuals are used. The same notation is
used when referring to sets of branch indices. So $\Omega_{a+b}$ and
$\Omega_{a/b}$ would be the sets of branches summed to give the above lengths.

Other important genetic quantities are $L$, the number of loci at which
mutations affect the trait, and $\T$, the mutation rate. The moment generating
function for the distribution of mutational effects is $\psi()$ and the moments
of this distribution are $m_i$. Moment generating functions for the distribution
of branch lengths and distribution of trait values are $\varphi_{\mathbf{T}}$
and $\varphi_{\mathbf{Y}}$.

\section{The moment generating function for the distribution of trait values}
The distribution of trait values in a population sample is rather complex in its
general form. The distribution has a point mass at zero corresponding the
possibility that no mutations affecting the trait occur in the history of the
sample. Correlations between individuals arise because of shared history in the
genealogies at individual loci with discrete topologies as well was where on
these genealogies mutations, whose effects may come from a discrete or
continuous distribution, occur. An analytical expression for the distribution of
trait values certainly does not exist.

However, it is possible at least in some cases to find the moment generating
function (mgf) of this distribution. The mgf for the trait values due to a
single nonrecombining locus is defined as
\begin{equation}
  \label{eq:mgfdef}
  \varphi_{\mathbf{Y}}(\mathbf{k}) = \E\left[ e^{\mathbf{k} \cdot \mathbf{Y}} \right] =
  \int e^{\mathbf{k} \cdot \mathbf{Y}} \Pro(\mathbf{Y}=\mathbf{y}) \mbox{d}\mathbf{y}.
\end{equation}
The vector $\mathbf{k}$ contains dummy variables for the integral transformation
of the probability distribution. This can be rewritten by conditioning on the
genealogy to give
\begin{align}
  \varphi_{\mathbf{Y}}(\mathbf{k}) &= \int e^{\mathbf{k} \cdot \mathbf{Y}}
  \int \Pro(\mathbf{Y}=\mathbf{y} | \mathbf{T}=\mathbf{t}) \Pro(\mathbf{T}=\mathbf{t})
  \mbox{d}\mathbf{t} \mbox{d}\mathbf{y}\\
  &= \int \int e^{\mathbf{k} \cdot \mathbf{Y}} \Pro(\mathbf{Y}=\mathbf{y} | \mathbf{T}=\mathbf{t}) \mbox{d}\mathbf{y}
  \Pro(\mathbf{T}=\mathbf{t})
  \mbox{d}\mathbf{t}.
\end{align}

To proceed it is necessary to make assumptions about the mutational process.
These are that mutations occur as a Poisson process along branches and that
mutations at a locus simply add to the effects of previous mutations (the
continuum-of-alleles model \citep{Kimura1965}). Under these assumptions, the
changes in the trait value along each branch are conditionally independent given
the branch lengths. \citet{Schraiber2015} note that this describes a compound
Poisson process. The mgf of a compound Poisson process with rate $\lambda$ over
time $t$ is $\exp(\lambda t (\psi(k)-1))$, where $\psi$ is the mgf of the
distribution of jumps. Additionally, the mgf of two completely correlated random
variables with the same marginal distribution is $\varphi_{X_1}(k_1+k_2)$, where
$\varphi_{X_1}$ is the mgf of the marginal distribution.

The mgf of the trait values is therefore
\begin{equation}
  \label{eq:fullmgf}
  \varphi_{\mathbf{Y}}(\mathbf{k}) = \prod_{\omega \in \mathcal{O}}
  \int \exp\left( \frac{\theta}{2} t_{\omega} \left( \psi\left(\sum_{a \in \omega}k_{a}\right) -1 \right)\right)
  \Pro(\mathbf{T}=\mathbf{t})\mbox{d}\mathbf{t}.
\end{equation}
This is simply the moment generating function for $\mathbf{T}$ with
$\frac{\theta}{2} \left( \psi(\sum_{i \in \omega}k_{\omega}) -1 \right)$
substituted for the dummy variable of branch $T_{\omega}$. Or,
\begin{equation}
  \label{eq:sub}
  \varphi_{\mathbf{T}}(\mathbf{s})\Bigr|_{s_{\omega}=\frac{\theta}{2} \left( \psi\left(\sum_{a \in \omega}k_{a}\right) -1 \right)}.
\end{equation}

Equation \eqref{eq:sub} shows that if the mgf of the distribution of branch
lengths is known, then the mgf of the trait values can be obtained through a
simple substitution. \citet{Lohse2011} derived genealogy mgfs for various
population models including migration and splitting of subpopulations. Using
their result for a single population it is possible to get equation (1) of
\citet{Schraiber2015} using the substitution defined by equation \eqref{eq:sub}.
The same could be done for models with migration between subpopulations although
the number of terms in the recursion for the genealogy mgf quickly becomes very
large. 

\section{The infinitesimal limit}
It is instructive to see how this very general model converges to the
infinitesimal when the right limits are taken. This is done by first
substituting Taylor series for the genealogical and mutational distributions in
equation \eqref{eq:fullmgf}. Taking the limits $L\T m_1 \to \mu$,
$L\T m_2\to \sigma^2$, $L\T m_i\to 0$ for $i>2$, and
$L^i\left(\T\right)^j \to 0$ for $i<j$ as $L\to \infty$ yields
\begin{equation}
  \label{eq:clt}
  \exp \left( \sum_{\omega \in \Omega}\E[T_{\omega}] \left( \mu \left(
  \sum_{a \in \omega} k_a\right) + \frac{\sigma^2}{2}\left( \sum_{a \in \omega}
  k_a\right)^2\right)\right).
\end{equation}
This is the mgf for a multivariate normal distribution where the expected trait
value is $E[T_{MRCA}] \mu$, the variance is $E[T_{MRCA}]\sigma^2$, and the
covariance between trait values in two individuals $a$ and $b$ is
$E[\tau_{a+b}] \sigma^2$. More detail is given in Appendix \ref{clt}.

$\mu$ can be interpreted as the rate of change in the mean trait value per
generation per genome due to mutational pressure. $\sigma^2$ can be interpreted
as the rate of accumulation of variance in trait values per generation per
genome. Interestingly, the trait variance appears to be proportional to the
second moment of the mutational distribution and not the variance.

Since the distribution of trait values is normally distributed, any linear
combination of sampled trait values will be as well. This includes the
distributions of observable quantities like the differences in trait values from
some reference individual or from a sample mean. This provides additional
theoretical justification for studies using normal models to look for
differences in selection on quantitative traits between populations
\citep{Ovaskainen2011,Praebel2013,Berg2014,Robinson2015}. Additionally, this
implies that a covariance matrix based on mean coalescent times rather than
population split should be used when modeling traits as normally distributed in
the phylogenetics of rapid diversification.
\section{Low mutation rate approximation}
The basic model used so far assumes that loci are unlinked and can experience an
infinite number of mutations. However, in the derivation of the normal model the
assumption was also made that only one mutation per locus occurs. This
approximation will be reasonable as long as the SNPs affecting the trait are
reasonably well spread across the genome. The low-mutation-rate approximation
greatly simplifies the mgf of the trait distribution such that it is no longer
necessary to know the full form of the mgf of the genealogy. 
\begin{equation}
\label{eq:lowmut}
\varphi_{\mathbf{Y}}(\mathbf{k}) \approx \left[ 1 + \sum_{\omega \in \mathcal{O}}
  \E[T_\omega] \T \left( \psi\left( \sum_{a \in \omega} k_a\right) -1 \right) \right]^L.
\end{equation}
Equation \eqref{eq:lowmut} ignores any terms that are products of branch lengths
and the mutation rate above order one. The convenient aspect of this equation is
that it depends only on the expected length of each branch. We can use this to
express moments of the trait distribution in terms of expected branch lengths
that can be calculated from coalescent models.
\section{Kurtosis}
The normal distribution is the maximum entropy distribution for a finite mean
and variance, and is thus completely defined by its first two moments. The
extent to which a trait distribution deviates from normality can be measured by
the extent to which its moments deviate from those of a normal distribution with
the same mean and variance. As an example I consider the kurtosis of the trait
value distribution in the simple case where the mutational distribution has mean
zero and is not skewed. Doing so shows how properties of both the mutational and
genealogical distributions determine the deviation from normality. 

The kurtosis of a distribution is defined as
\begin{equation*}
  \mbox{Kurt}[X]=\frac{\E[(X-\E[X])^4]}{(\E[(X-\E[X])^4)^2}.
\end{equation*}
This is the fourth central moment divided by the variance. The kurtosis is a
measure of the propensity of a distribution to produce outliers
\citep{Westfall2014}. The kurtosis of a single trait value, where the
expectation is over evolutionary replicates of demography and mutation, is 
\begin{equation}
  \label{eq:kurt1}
  \mbox{Kurt}[Y_a] \approx \frac{\kappa}{L\T \E[T_{MRCA}]} + 3.
\end{equation}
The kurtosis of a normal distribution is $3$, so it is immediately clear that
the trait distribution will exceed this. This excess is determined by the ratio
of the kurtosis of the mutational distribution, denoted by $\kappa$, and the
expected number of mutations before the most recent common ancestor of the
sample ($L\T \E[T_{MRCA}]$). This make intuitive sense that the kurtosis will be
higher when mutations tend to produce more outliers and lower when more of these
are added up and things look normal through the central limit theorem. 

As mentioned before, $Y_a$ is defined relative to the value of the most recent
common ancestor of the sample and thus can't be observed. What is observable are
differences in trait values between individuals, and equation \eqref{eq:kurt1}
also holds for the kurtosis of trait differences if the expected TMRCA is
replaced by twice the expected coalescence time. 

The expected trait kurtosis in a populations of individuals may also be of
interest as individuals with extreme trait values could be important for future
selection or could be mistakenly thought to be adapted to a different
environment. When individuals are exchangeable, the population kurtosis is given
by
\begin{equation}
  \label{eq:popkurtcoal}
  \E[\mbox{Kurt}] \approx 3 + \frac{\kappa( 4\E[T_2] - 6\E[T_3] + 
    3\E[T_4])}{L \T \E[T_2]},
\end{equation}
where $T_i$ is the expected $T_{MRCA}$ for a sample of size $i$. The relative
values of these different coalescence times affects the population kurtosis. In
a constant size population equation \eqref{eq:popkurtcoal} reduces to
$3 + \frac{\kappa}{2L\T \E[T_2]}$ as expected. More details on the derivation of
both kurtosis values is given in Appendix \ref{kurt}.

In general, for the kurtosis to substantially exceed that of the normal
distribution, the expected number of variants affecting the trait must at least
be on the same order as the mutational kurtosis. This requires mutation to
produce extreme values with considerable frequency or for the trait architecture
to be very sparse. It's possible that traits such as expression of certain genes
satisfy this criteria \citep{Wheeler2016}.
\section{The response to selection}
One situation in which higher order moments of the trait distribution can be
relevant is in the response of the population to selection. The classical theory
of quantitative genetics assumes that the distribution of additive genetic
values in the population remains normally distributed as selection alters the
mean and variance. \citet{Turelli1990} used a multilocus population genetic
model to show how departures from normality can influence the response to
selection. In their analysis these departures are due to the build up of linkage
disequilibrium. However, their results are valid regardless of how the
departures from normality arise.

According to \citet{Turelli1990}, the response of the mean phenotype in the
population is 
\begin{equation}
  \label{eq:selresp}
  \Delta \bar{Z} = V_gL_1 + M_{3,g}L_2 + \gamma_4V^2_gL_3 +
  \left( M_{5,g}-4M_{3,g}V_g\right)L_4 + \ldots.
\end{equation}
Here, $Z$ refers to a phenotypic value which has an environmental component as
opposed to the trait value $Y$ which is due entirely to genetics. $V_g$ is the
variance in breeding values in the population and $\gamma_4$ is the excess
kurtosis above a normal distribution. The terms $M_{i,g}$ are the $i^{th}$
central moments of the breeding value distribution in the population. The terms
$L_i$ describe the effect of selection on the trait values and are selection
gradients in terms of the genotypic moments. In the absence of environmental
effects, or if selection acts directly on the genetic trait values, this
equation holds for the response of the mean trait value.

Equation \eqref{eq:selresp} shows that the importance of higher order moments of
the trait distribution depends on the precise shape of the fitness function. In
particular, the excess kurtosis affects the response to selection linearly with
$L_3$, the partial derivative of the log of the mean fitness with respect to the
third moment of the trait value distribution. As a very simple example we can
consider selection acting only directly on the trait values with a cubic
selection function. This can be represented by
\begin{equation}
  \label{eq:cubsel}
  W_g(Y) = b_0 + b_3(Y-\bar{Y})^3.
\end{equation}
This fitness function models selection on differences from the current mean
breeding value in the population. One thing we need to consider about the cubic
selection function is that it can give negative values for fitness, which is
obviously bad. We therefore must assume that the degree of cubic selection,
$b_3$ is not too large relative to fitness value at the population mean, $b_0$.
It turns out that the response to selection is this simply
\begin{equation}
  \label{eq:cubresp}
  \Delta \bar{Z} = \frac{M_{4,g}\beta}{1 + M_{3,g}\beta},
\end{equation}
where $\beta=b_3/b_0$. Simulations show that this predicts the response to
selection up until about $\beta\approx 2e-3$. Equation \eqref{eq:cubresp}
predicts that, holding the variance constant, the response to selection of the
mean trait value increases linearly with the kurtosis.

This is not the most realistic fitness function because it has been centered on
the mean trait value in the population. The mean will have diverged since the
most recent common ancestor of the population in the absence of stabilizing
selection making this value not so meaningful biologically. Additionally, for
the selection to be strictly cubic in shape is unrealistic. However, a large set
of fitness functions will have a component which is cubic on the trait values.
Based on this result we would expect the importance of that cubic part to depend
on the population trait kurtosis.
\section{Discussion}
The main result of this work has been to show that the characteristic function
for the distribution of trait values derived by \citet{Schraiber2015} is a
special case of the general relationship given in equation \eqref{eq:sub}. What
this implies is that to get mgf of the trait value distribution one must first
be able to write the mgf for the genealogical distribution. These functions have
been studied by \citet{Lohse2011} and in subsequent papers. They are recursive
in nature and the number of terms becomes quickly intractable as the number of
individuals increases. Things are simplified substantially when either a low
mutation approximation or infinitesimal model is used. In these cases only the
mean branch lengths or expected pair coalescent times, respectively, are
necessary to describe the trait value distribution.

The focus on kurtosis here is perhaps undue. It is presented mostly as an
example of how a simple and intuitive result can be obtained from the given
mgfs. For many traits the infinitesimal model surely provides a good null
distribution. For traits that do have sparse genetic architectures and
mutational distributions with large tails it is possible that aspects of the
mutational distribution could be learned from population samples, but it would
likely be necessary to somehow pool information across a large number of traits.
A possible instance where this could be achieved is for gene expression levels
\citep{Wheeler2016}.

It would be interesting to extend these results, which are only for the haploid,
unlinked, additive case to include factors like dominance, epistasis and
linkage. For epistasis this is likely to be very difficult since there are no
nice relationships for the mgf of a nonlinear combination of random variables.
Still, some approximations may be possible for particular forms of epistasis.
These would be of particular interest when testing for departures from
neutrality in structured populations as in \citet{Ovaskainen2011}.

\bibliographystyle{genetics} 
\bibliography{quant_gen}
\clearpage
\appendix
\section{Central limit theorem for the infinitesimal model}
\label{clt}
Recall that the moment generating function for a general distribution of
coalescence times is
\begin{equation*}
  \varphi_T(\mathbf{s}) = \int \exp \left( \sum_{\omega \in \mathcal{O}} s_{\omega}t_{\omega} \right)
  \Pro(\mathbf{T}=\mathbf{t})d\mathbf{t}.
\end{equation*}
The Taylor series expansion of
$\exp \left( \sum_{\omega \in \mathcal{O}} s_{\omega}t_{\omega} \right)$ is
\begin{equation}
  1 + \sum_{\omega \in \mathcal{O}} s_{\omega}t_{\omega} +
  \sum_{\omega_1 \neq \omega_2} s_{\omega_1}t_{\omega_1}s_{\omega_2}t_{\omega_2} + 
  \sum_{\omega \in \mathcal{O}} \frac{s_{\omega}^2t_{\omega}^2}{2} + \ldots. \nonumber
\end{equation}
The $s_\omega$ are each proportional to the mutation rate per site per time unit
when we substitute to get the trait mgf, so terms of order
$s_{\omega_i}t_{\omega_i}s_{\omega_j}t_{\omega_j}$ and above correspond to the
probability that two mutations occur. We'll assume that the probability that two
mutations occur at any of the $L$ sites is low enough to ignore these terms.
This gives 
\begin{equation*}
  \varphi_{\mathbf{Y}}(\mathbf{k}) \approx \left[ 1 + \sum_{\omega \in \mathcal{O}}
    \E[T_\omega] \T \left( \psi\left( \sum_{a \in \omega} k_a\right) -1 \right) \right]^L.
\end{equation*}
We will also take the
Taylor series of the moment generating function of the mutational distribution.
Noting that $\frac{d^n}{dk^n}\varphi_X(k)\Bigr|_{k=0} = \E[X^n]$, if we let $m_1$
be the mean and $m_2$ be the second moment of the mutational distribution, then
\begin{equation*}
  \psi\left( \sum_{a \in \omega} k_a \right) = 1 + m_1 \left( \sum_{a \in \omega}
    k_a\right) + m_2/2\left( \sum_{a \in \omega} k_a\right)^2 + 
  m_3/6\left( \sum_{a \in \omega} k_a\right)^3 + \ldots.
\end{equation*}
Substituting this in for $s_\omega$ we get
\begin{equation*}
  \left( 1 + \frac{\theta}{2} \sum_{\omega \in \mathcal{O}} \E[t_{\omega}]\left( m_1 \left(
  \sum_{a \in \omega} k_a \right) + \frac{m_2}{2} \left( \sum_{a \in \omega}
  k_a\right)^2 + m_3/6\left( \sum_{a \in \omega} k_a\right)^3 + \ldots \right) \right)^L.
\end{equation*}
Taking the limits $L\T m_1 \to \mu$, $L\T m_2\to \sigma^2$, $L\T m_i\to 0$ for
$i>2$, and $L^i\left(\T\right)^j \to 0$ for $i<j$ as $L\to \infty$ yields
\begin{equation*}
  \label{eq:clt}
  \exp \left( \sum_{\omega \in \Omega}\E[T_{\omega}] \left( \mu \left(
  \sum_{a \in \omega} k_a\right) + \frac{\sigma^2}{2}\left( \sum_{a \in \omega}
  k_a\right)^2\right)\right).
\end{equation*}
This is multivariate normal with mean equal to $\E[T_{MRCA}]\mu$ because and
covariance equal to $\E[T_{\Omega_{a+b}}]\sigma^2$ because in the mgf for a
multivariate normal distribution the coefficient in the exponential of $k_a$ is
the mean of $Y_a$ and the coefficient of $k_ak_b$ is $2\Cov[Y_a,Y_b]$ if
$a\neq b$ and $\Var[Y_a]$ if $a=b$.
\section{Kurtosis derivations}
\label{kurt}
We can use the low mutation rate approximation to the moment generating function
to calculate moments of the distribution of trait vales. We'll start by
calculating the first and second moments. We start, as we did in deriving the
normal distribution, by substituting the Taylor series of the mutational mgf.
\begin{equation}
  \label{eq:mgf_approx_sub}
  \varphi_{\mathbf{Y}}(\mathbf{k}) \approx \left[ 1 + \sum_{\omega \in \Omega}
    \E[T_\omega] \T \left( m_1 \sum_{a \in \omega} k_a +
    \frac{m_2}{2!}\left( \sum_{a \in \omega} k_a\right)^2 +
    \frac{m_3}{3!}\left( \sum_{a \in \omega} k_a\right)^3 +
    \frac{m_4}{4!}\left( \sum_{a \in \omega} k_a\right)^4 \ldots \right) \right]^L
\end{equation}
We can expand this out using multinomial coefficients to get
\begin{align}
  \label{eq:mgf_approx_expand}
  \varphi_{\mathbf{Y}}(\mathbf{k}) &\approx 1 +
  L\T \sum_{\omega \in \mathcal{O}} \E[T_{\omega}]\left( m_1 \sum_{a \in \omega} k_a +
  \frac{m_2}{2}\left( \sum_{a \in \omega} k_a\right)^2 + \ldots \right) \nonumber \\
  &+ \frac{L(L-1)}{2} \left(\T\right)^2 \sum_{\omega \in \Omega} E[T_{\omega}]^2
  \left( m_1 \sum_{a \in \omega} k_a +
  \frac{m_2}{2}\left( \sum_{a \in \omega} k_a\right)^2 + \ldots \right)^2 \nonumber \\
  &+ L(L-1)\left(\T\right)^2\sum_{\omega_1, \omega_2 \in \Omega}\E[T_{\omega_1}]\E[T_{\omega_2}]
  \left( m_1 \sum_{a \in \omega_1} k_a + \ldots \right)
  \left( m_1 \sum_{a \in \omega_2} k_a + \ldots \right) + \ldots.
\end{align}
The first coefficient is $\binom{L}{L-1,1,\mathbf{0}}$, the second is
$\binom{L}{L-2,2,\mathbf{0}}$, and the third is $\binom{L}{L-2,1,1,\mathbf{0}}$.
To calculate the moments of this distribution one takes the partial derivatives
of the mgf and sets the dummy variables to zero.
\begin{equation}
  \label{eq:deriv}
  \E[Y_1^{r_1}\ldots Y_n^{r_n}] = \frac{\partial^{r_1 + \ldots + r_n}}{\partial k_1^{r_1} \ldots \partial k_n^{r_n}}
  \varphi_{\mathbf{Y}}(\mathbf{k})\Bigr|_{\mathbf{k}=0}
\end{equation}
Using this to calculate the first moment of the trait distribution we get
\begin{equation}
  \label{eq:mom1}
  \E[Y_a] \approx L\T m_1 \sum_{\omega \in \Omega_a} \E[T_\omega].
\end{equation}
The second moment is more complicated because there are $k_a^2$ terms in all
three lines of equation \ref{eq:mgf_approx_expand}.
\begin{align}
  \E[Y_a^2] &\approx L\T m_2 \sum_{\omega \in \Omega_a} E[t_\omega] \nonumber \\
  &+ \frac{L(L-1)}{2} \left(\T \right)^2 m_1^2 \sum_{\omega \in \Omega_a} 2 \E[T_\omega]^2 \nonumber \\
  &+ L(L-1) \left(\T \right)^2 m_1^2 \sum_{\omega_1 , \omega_2 \in \Omega_{a+b}} 2 \E[T_{\omega_1}]E[T_{\omega_2}]
\end{align}
Terms with $(\T)^2$ are kept because they also include a second order term of
$L$ in front of them. We can now calculated the variance using $\Var[Y]=\E[Y^2] -
\E[Y]^2$. The squared first moment can be written as
\begin{align}
  \left(L\T m_1 \sum_{\omega \in \Omega_a} \E[t_\omega] \right)^2 &=
  L^2\left(\T\right)^2 m_1^2 \sum_{\omega \in \Omega_a} \E[T_\omega]^2 \nonumber \\
  &+ L^2\left(\T\right)^2 m_1^2 \sum_{\omega_1 , \omega_2 \in \Omega_{a+b}} \E[T_{\omega_1}]\E[T_{\omega_2}].
\end{align}
Subtracting this from the second moment gives
\begin{align}
  \label{eq:var}
  \Var[Y_a] &\approx L\T m_2 \sum_{\omega \in \Omega_a} \E[T_\omega] \nonumber \\
  &- L \left(\T\right)^2 m_1^2 \sum_{\omega \in \Omega_a}\E[T_\omega]^2 \nonumber \\
  &-  2L \left(\T\right)^2 m_1^2 \sum_{\omega_1 , \omega_2 \in \Omega_{a+b}} \E[T_{\omega_1}]\E[T_{\omega_2}] \nonumber \\
  &= L\T m_2 \sum_{\omega \in \Omega_a} \E[T_\omega] -
  L\left( \T m_1 \sum_{\omega \in \Omega_a} \E[T_\omega] \right)^2 \nonumber \\
  &= L\T m_2 \E[T_{MRCA}] - L\left( \T m_1 \E[T_{MRCA}] \right)^2  \\
  &\approx L\T m_2 \E[T_{MRCA}].  \nonumber
\end{align}

Due to the large number of terms I only derive the fourth moment of the trait
value distribution for the case when the mean mutational effect is zero. Having
the mean equal to zero is also helpful when comparing against a normal
distribution because higher order moments of the normal distribution are easy to
calculate when the mean is zero. The terms of \eqref{eq:mgf_approx_sub} that
will appear in the fourth moment after we apply \eqref{eq:deriv} are
\begin{equation*}
  L \left(\T\right) \frac{m_4}{24}
  \sum_{\omega \in \Omega_a} \E[T_\omega] \left(\sum_{a \in \omega} k_a\right)^4
\end{equation*}
for the fourth moment along one branch,
\begin{equation*}
  \binom{L}{L-2,2,\mathbf{0}}\left(\T\right)^2\left(\frac{m_2}{2}\right)^2
  24 \sum_{\omega \in \Omega_a} \E[T_\omega]^2 \left(\sum_{a \in \omega} k_a\right)^4
\end{equation*}
for the second moment of the same branch chosen twice, and
\begin{equation*}
  \binom{L}{L-2,1,1,\mathbf{0}}\left(\T\right)^2\left(\frac{m_2}{2}\right)^2
  24 \sum_{\omega_1 , \omega_2 \in \Omega_{a+b}} \E[T_{\omega_1}]\E[T_{\omega_2}]
  \left(\sum_{a \in \omega_1} k_a\right)^2\left(\sum_{a \in \omega_2} k_a\right)^2
\end{equation*}
for the second moments on two different branches. Taking the fourth derivatives
of these in terms of the desired branch we get
\begin{align}
  \label{eq:mom4}
  \E[Y_a^4] &= L\T m_4 \E[T_{MRCA}] \nonumber \\
  &+ \frac{L(L-1)}{2} \left(\T\right)^2\left( \frac{m_2}{2} \right)^2
  24 \sum_{\omega \in \Omega_a} \E[t_\omega]^2 \nonumber \\
  &+ L(L-1) \left(\T\right)^2\left( \frac{m_2}{2} \right)^2
  24 \sum_{\omega_1 , \omega_2 \in \Omega_{a+b}} \E[T_{\omega_1}]\E[T_{\omega_2}] \nonumber \\
  &= L\T m_4 \E[T_{MRCA}] +
  3L(L-1)\left( \T m_2 \sum_{\omega \in \Omega_a} \E[T_\omega] \right)^2x \nonumber \\
  &= L\T m_4 \E[T_{MRCA}] +
  3L(L-1)\left( \T m_2 \E[T_{MRCA}] \right)^2 \\
  &\approx L\T m_4 \E[T_{MRCA}] +
  3\left(L \T m_2 \E[T_{MRCA}] \right)^2. \nonumber 
\end{align}

The kurtosis is defined as
\begin{equation*}
  Kurt[X]=\frac{E[(X-E[X])^4]}{(E[(X-E[X])^4)^2}.
\end{equation*}
This is the fourth central moment divided by the variance. For ease of
calculation, we'll examine this in the case where the mean mutation effect (and
therefore trait value) is zero. If we plug \eqref{eq:var} and \eqref{eq:mom4}
into the expression for the kurtosis we get
\begin{align*}
  \Kurt[Y_a] &= \frac{L\T m_4 \E[T_{MRCA}]}{\left(L\T m_2 \E[T_{MRCA}]\right)^2} +
  \frac{3L(L-1)\left( \T m_2  \E[T_{MRCA}]\right)^2}{\left(L\T m_2 \E[T_{MRCA}]\right)^2} \nonumber \\
  &= \frac{m_4}{L\T m_2^2\E[T_{MRCA}]} + \frac{3(L^2-L)}{L^2} \nonumber \\
  &= \frac{\kappa}{L\T \E[T_{MRCA}]} + 3\left( 1 - \frac{1}{L} \right).
\end{align*}

We also calculate here some additional moments that have less clear
interpretations but are useful later on when calculating the expected population
kurtosis. The first of these is $\E[Y_a^3Y_b]$. The terms of
\eqref{eq:mgf_approx_sub} that will appear in this are
\begin{equation*}
  L \left(\T\right) \frac{m_4}{24} 4 k_a^3k_b \sum_{\omega \in \Omega_{a+b}} \E[T_\omega]
\end{equation*}
and
\begin{equation*}
  L(L-1)\left(\T \frac{m_2}{2}\right)^2 k_a^2 \times 2k_ak_b
  \left( \sum_{\omega \in \Omega_{a+b}} \E[T_\omega] \right) \left( \sum_{\omega \in \Omega_a} \E[T_\omega] \right).
\end{equation*}
This ultimately gives
\begin{equation}
  \label{eq:m31}
  E[Y_a^3Y_b] = L \T m_4 \E[\tau_{a+b}] + 3L(L-1) \left(\T m_2\right)^2 \E[T_{MRCA}]\E[\tau_{a+b}].
\end{equation}
The next fourth moment of interest is $\E[Y_a^2Y_bY_c]$. The terms of
\eqref{eq:mgf_approx_sub} are
\begin{equation*}
  L \T \frac{m_4}{24} 12k_a^2k_bk_c \sum_{\omega \in \Omega_{a+b+c}} \E[T_\omega],
\end{equation*}
\begin{equation*}
  L(L-1) \left(\T \frac{m_2}{2}\right)^2 k_a^2 \times 2k_bk_c
  \left( \sum_{\omega \in \Omega_a} \E[T_\omega] \right)\left( \sum_{\omega \in \Omega_{a+b}} \E[T_\omega] \right),
\end{equation*}
and 
\begin{equation*}
  L(L-1) \left(\T \frac{m_2}{2}\right)^2 2k_ak_b \times 2k_ak_c \left( \sum_{\omega \in \Omega_{a+b}} \E[T_\omega] \right)
  \left( \sum_{\omega \in \omega_{b+c}} \E[T_\omega] \right).
\end{equation*}
Taking the appropriate derivatives of these gives
\begin{equation}
  \label{eq:m211}
  L \T m_4 \E[\tau_{a+b+c}] + L(L-1)\left(\T m_2\right)^2\E[T_{MRCA}]\E[\tau_{b+c}] +
  2L(L-1) \left(\T m_2\right)^2\E[\tau_{a+b}]\E[\tau_{a+c}].
\end{equation}
Individuals in the population are exchangeable as long as it is not structured.
The pairwise expected shared branch lengths are in that case all equal and we
can write \eqref{eq:m211} as
\begin{equation}
  \label{eq:m211s}
  L \T m_4 \E[\tau_{a+b+c}] + L(L-1)\left(\T m_2\right)^2\E[T_{MRCA}]\E[\tau_{a+b}] +
  2L(L-1) \left(\T m_2\right)^2\E[\tau_{a+b}]^2.
\end{equation}
Where $\tau_{a+b}$ just refers to the expected shared branch length of any two
individuals in the population. The final moment we'll look at is
$\E[Y_aY_bY_cY_d]$ which has relevant terms
%% The set of branches containing all individuals
\begin{equation*}
  L \T \frac{m_4}{24} 24k_ak_bk_ck_d \sum_{\omega \in \Omega_{a+b+c+d}} \E[T_\omega],
\end{equation*}
%% Cross set of a-b branches and c-d branches
\begin{equation*}
  L(L-1) \left(\T \frac{m_2}{2}\right)^2 2k_ak_b \times 2k_ck_d \left( \sum_{\omega \in \Omega_{a+b}} \E[T_\omega] \right)
  \left( \sum_{\omega \in \Omega_{c+d}} \E[T_\omega] \right),
\end{equation*}
%% Cross set of a-c branches and b-d branches
\begin{equation*}
  L(L-1) \left(\T \frac{m_2}{2}\right)^2 2k_ak_c \times 2k_bk_d \left( \sum_{\omega \in \Omega_{a+c}} \E[T_\omega] \right)
  \left( \sum_{\omega \in \Omega_{b+d}} \E[T_\omega] \right),
\end{equation*}
and
%% cross set of a-d branches and b-c branches
\begin{equation*}
  L(L-1) \left(\T \frac{m_2}{2}\right)^2 2k_ak_d \times 2k_bk_c \left( \sum_{\omega \in \Omega_{a+d}} \E[T_\omega] \right)
  \left( \sum_{\omega \in \omega_{b+c}} \E[T_\omega] \right).
\end{equation*}
When the appropriate fourth order partial derivatives are taken of this we get
\begin{align*}
  L \T m_4 \E[\tau_{a+b+c+d}] \\
  + L(L-1)\left(\T m_2\right)^2\E[\tau_{a+b}]\E[\tau_{c+d}] \\
  + L(L-1)\left(\T m_2\right)^2\E[\tau_{a+c}]\E[\tau_{b+d}] \\
  + L(L-1)\left(\T m_2\right)^2\E[\tau_{a+d}]\E[\tau_{b+c}].
\end{align*}
We can again simplify this expression for populations with exchangeable
individuals. This gives
\begin{equation}
  \label{eq:m1111}
  L \T m_4 \E[\tau_{a+b+c+d}] + 3L(L-1)\left(\T m_2\right)^2\E[\tau_{a+b}]^2.
\end{equation}

The expected kurtosis in the population is a quotient and therefore annoying to
calculate. Instead we will calculate the expected fourth central moment. 
\begin{equation}
  \E[M_{4,Y}] = \E\left[\frac{1}{N} \sum \left( Y_i - \frac{\sum Y_j}{N} \right)^4 \right].
\end{equation}
Examining the sum inside the expectation we see that 
\begin{align}
  \label{eq:expandkurt}
  \E\left[ \left( Y_i - \frac{\sum Y_j}{N} \right)^4 \right] &= \E[Y_i^4] - 
                                                              4\E\left[ Y_i^3 \frac{\sum Y_j}{N} \right] + 
                                                              6\E\left[Y_i^2\left(\frac{\sum Y_j}{N}\right)^2\right] - 
                                                              4\E\left[Y_i\left(\frac{\sum Y_j}{N}\right)^3\right]+ 
                                                              \left(\frac{\sum Y_j}{N}\right)^4 \nonumber \\
                                                            &= \E[Y_i^4] - 
                                                              \frac{4}{N}\sum_j \E[Y_i^3Y_j] + 
                                                              \frac{6}{N^2}\sum_{j,k} \E[Y_i^2Y_jY_k] \nonumber\\
                                                              &-\frac{4}{N^3}\sum_{j,k,l}\E[Y_iY_jY_kY_l] + 
                                                              \frac{1}{n^4}\sum_{j,k,l,d}\E[Y_jY_kY_lY_d].
\end{align}
In calculating these expectations we have to remember that the value depends
only on the number of times each variable appears in the expectation. That is,
$\E[Y_1^2Y_2Y_3]$ is equivalent to $\E[Y_1Y_2^3Y_3]$ as long as all individuals in
the population are exchangeable. The resulting expansion of
\eqref{eq:expandkurt} is therefore quite ugly. It can be simplified by only
considering terms of order one. Other terms can be ignored since we are
assuming there are a fair number of individuals in the population. This yields
\begin{align}
  \label{eq:popkurt}
  \E\left[ \left( Y_i - \frac{\sum Y_j}{N} \right)^4 \right] &=
  \E[Y_i^4]  - \frac{4(n-1)}{n}\E[Y_i^3Y_j] + \frac{6(n-1)(n-2)}{n^2}\E[Y_i^3Y_jY_k]  \nonumber \\
  &- \frac{4(n-1)(n-2)(n-3)}{n^3}\E[Y_iY_jY_kY_l] \nonumber \\
  &+ \frac{(n-1)(n-2)(n-3)(n-4)}{n^4}\E[Y_jY_kY_lY_d]
  + O(n^{-1}) \nonumber \\
  &\approx \E[Y_i^4]  - 4\E[Y_i^3Y_j] + 6\E[Y_i^2Y_jY_k] - 3\E[Y_iY_jY_kY_l].
\end{align}

The first term, $\E[Y_i^4]$ was derived in equation \eqref{eq:mom4} as
\begin{equation*}
  L\T m_4 \E[T_{MRCA}] + 3L(L-1)\left( \T m_2 \E[T_{MRCA}] \right)^2.
\end{equation*}
The second term, $\E[Y_i^3Y_j]$ was derived in equation \eqref{eq:m31} as
\begin{equation*}
  L \T m_4 \E[\tau_{a+b}] + 3L(L-1) \left(\T m_2\right)^2 \E[T_{MRCA}]\\E[\tau_{a+b}].
\end{equation*}
The third term, $\E[Y_i^2Y_jY_k]$ was derived in equation \eqref{eq:m211} as
\begin{equation*}
  L \T m_4 \E[\tau_{a+b+c}] + L(L-1)\left(\T m_2\right)^2\E[T_{MRCA}]\E[\tau_{a+b}] +
  2L(L-1) \left(\T m_2\right)^2\E[\tau_{a+b}]^2.
\end{equation*}
The fourth term, $\E[Y_iY_jY_kY_l]$ was derived in equation \eqref{eq:m1111} as
\begin{equation*}
  L \T m_4 \E[\tau_{a+b+c+d}] + 3L(L-1)\left(\T m_2\right)^2\E[\tau_{a+b}]^2.
\end{equation*}
Plugging these into \eqref{eq:popkurt} we get
\begin{align}
  \label{eq:popm4coal}
  \E[M_{4,Y}] &\approx L \T m_4 \left( \E[T_{MRCA}] - 4\E[\tau_{a+b}] + 6\E[\tau_{a+b+c}] -3\E[\tau_{a+b+c+d}] \right)\nonumber \\
  &+ 3\left( L \T m_2 \right)^2\left(\E[T_{MRCA}]- \E[\tau_{a+b}]\right)^2.
\end{align}
These expressions approximate $L(L-1)$ as $L^2$ as part of the low mutation rate
approximation. The fourth central moment of a normal distribution is
$3\sigma^4$. Using the expected population  variance we get $3\left(L\T m_2
\E[T_{MRCA}]- \E[\tau_{a+b}]\right)^2$. It is clear from \eqref{eq:popm4coal} that
as $L\T$ gets large the second term will dominate and the population will have
the same expected kurtosis as a normal distribution. The expected population kurtosis
is
\begin{equation*}
  \E[\mbox{Kurt}] \approx 3 + \frac{\kappa( \E[T_{MRCA}] - 4\E[\tau_{a+b}] +
    6\E[\tau_{a+b+c}] -3\E[\tau_{a+b+c+d}])}{L \T \E[T_{MRCA}]- \E[\tau_{a+b}]}.
\end{equation*}
This expression can be written in an easier to interpret form by noting that
$\E[\tau_{a+b}]=\E[T_{MRCA}] - \E[T_2]$, $\E[\tau_{a+b+c}]=\E[T_{MRCA}] - \E[T_3]$,
$\E[\tau_{a+b+c+d}]=\E[T_{MRCA}] - \E[T_4]$, where $T_i$ is the expected time it
takes for $i$ lineages to coalesce.
\begin{equation}
  \label{eq:popkurtcoal}
  \E[\mbox{Kurt}] \approx 3 + \frac{\kappa( 4T_2 - 6T_3 + 3T_4)}{L \T T_2}.
\end{equation}
In a constant size population this would be $3 + \frac{\kappa}{2L\T}$.
\end{document}
