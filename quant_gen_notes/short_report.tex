
\documentclass{article}
\usepackage{amsmath}
\usepackage{natbib}
\usepackage[textwidth=5.8in, textheight=8in]{geometry}
\usepackage{lipsum}
\usepackage[mathscr]{eucal}
\usepackage{graphicx}

\newcommand{\T}{\frac{\theta}{2}}
\newcommand{\E}{\mathrm{E}}
\newcommand{\Var}{\mathrm{Var}}
\newcommand{\Cov}{\mathrm{Cov}}
\newcommand{\Pro}{\mathrm{P}}

\begin{document}

\title{Quantitative trait distributions and the coalescent process}
\maketitle

\section{Introduction}

Evolutionary quantitative genetics almost always models phenotypes as normally
distributed. Indeed, it has been suggested that this is the defining
characteristic of the field \citep{Rice2004}. In this framework the phenotypic
distribution at all levels between individuals, families, populations, and
species is completely described by of a set of means, variances, covariances
\citep{Falconer1996}. Under the infinitesimal model that began with
\citet{Fisher1919}, the phenotypic trait is controlled by a very large number of
loci and the normality assumption is well-justified.

The infinitesimal-normal model has been a powerful tool in many areas of evolutionary
genetics. Among its many successes, it has been used to find traits under
selection in natural populations \citep{Price1984}, determine which dimensions
in phenotype space are favored by females during sexual selection
\citep{Blows2004}, and test for phylogenetic signal in phenotypic evolution
\citep{Freckleton2002}. Such studies and other provide great biological insights
without worrying about potentially complex evolutionary dynamics at multiple
loci because the entire system can be described in terms of the first and second
order moments.

However, heritable phenotypic variation is ultimately due to discrete mutations
at discrete loci, and how this variation is distributed depends on the
genealogies at these loci. Classical neutral models of phenotypic evolution
incorporate the effects of factors like population size and subdivision on
genealogies and therefore phenotypic variation \citep{Lynch1986,Lande1992}.
These models have examined the dynamics of phenotypic evolution forwards in time
as a balance between mutation that creates variance, mutation that spreads it
among subpopulations, and fixation that removes it. The focus has been to find
equilibrium genetic variances and the rate at which these are approached.

The principle modeling framework for variation in genealogies is the coalescent
process with all its varied extensions \citep{Wakeley2008}, but few studies have
attempted to connect this theory with quantitative genetics. The most well known
of these is \citet{Whitlock1999}, who used a coalescent argument to show that
$Q_{ST}$ and $F_{ST}$ have the same expected value under general models of
population subdivision. \citet{Griswold2007} used coalescent simulations to
investigate the effects of shared ancestry and linkage disequilibrium on the
matrix of genetic variances and covariances between traits
($\mbox{\textbf{G}}$). They found that the eigenvalues of the expected matrix
and the expected eigenvalues can be quite different. \citet{Ovaskainen} used the
older (but deeply related \citep{\Tavare1984}) formalism of coancestry
coefficients to model the full distribution of correlated traits in related
subpopulations. Their main result is that the covaraince in additive genetic
values, conditional on the G-matrix in the ancestral population, depends only on
the set of pairwise coancestry coefficients.

Most recently, \citet{Schraiber2015} developed a very general model of
quantitative trait evolution based on the coalescent that allows for an
arbitrary distribution of mutational effects and number of loci affecting the
trait. They derive the characteristic function for the distribution of
phenotypic values in a sample and show how such values can deviate strongly from
normality when the number of loci is small or the mutational distribution has
fat tails.

The results of \citet{Schraiber2015} were derived for a panmictic, constant-size
population. Here, I generalize their work to an arbitrary distribution of
coalescent times for which their main result, the characteristic function of the
sampling distribution of phenotypic values, is a special case. I then show how
the normal, infinitesimal model arises by taking appropriate limits. I then
calculate the kurtosis of the phenotypic distribution to illustrate the
dependence of departures from normality on both the mutational and genealogical
distributions.

\section{Notation}
Before deriving anything, it is helpful to describe the various quantities of
interest and the notation used to describe them. The principle quantity of
interest are the quantitative trait values. Hereafter I'll simply refer to these
as trait values even though it might be more precise to call them breeding
values or additive genetic values. The random vector of trait values in the
sampled individuals is $\mathbf{Y}$, and an element $Y_a$ of $\mathbf{Y}$ is the
trait value in individual $a$. An important thing to remember about the trait
values is that they really refer to the change in the trait value since the most
recent common ancestor of the sample. Since we do not know what the trait value
of this ancestor was, these trait values are not directly observed. Rather, they
determine other summaries one might calculate from a sample of measured
phenotypes.

The random vector of branch lengths describing the genealogy at a locus is
$\mathbf{T}$, and an element $T_{a,b}$ of $\mathbf{T}$ is the branch length
subtending individuals $a$ and $b$. $\omega$ will be used to denote a particular
set of individuals and hence a branch on the genealogy, and $\Omega$ is the set
of all possible branches. For instance, if there are three sampled individuals,
$a$, $b$, and $c$, then $\mathbf{Y}=\{Y_a,Y_b,Y_c\}$,
$\mathbf{T}=\{T_a,T_b,T_c,T_{a,b},T_{a,c},T_{b,c}\}$, and
$\Omega=\{(a),(b),(c),(a,b),(a,c),(b,c)\}$. Realizations of the random trait
values and branch lengths use the same notation with lowercase letters.

Another useful piece of notation is for sums of branch lengths. Let $\tau_{a+b}$
be the random sum of all branches containing both $a$ and $b$. Conversely,
$\tau_{a/b}$ will be the random sum of all branches containing $a$ and not $b$.
Extensions of this for more than two individuals are used. The same notation is
used when referring to sets of branch indices. So $\Omega_{a+b}$ and
$\Omega_{a/b}$ would be the sets of branches summed to give the above lengths.

Other important genetic quantities are $L$, the number of loci at which
mutations affect the trait, and $\T$, the mutation rate. The moment generating
function for the distribution of mutational effects is $\psi()$ and the moments
of this distribution are $m_i$. Moment generating functions for the distribution
of branch lengths and distribution of trait values are $\varphi_{\mathbf{T}}$
and $\varphi_{\mathbf{Y}}$.

\section{The moment generating function for the distribution of trait values}
The distribution of trait values in a population sample is rather complex in its
general form. The distribution has a point mass at zero corresponding the
possibility that no mutations affecting the trait occur in the history of the
sample. Correlations between individuals arise because of shared history in the
genealogies at individual loci with discrete topologies as well was where on
these genealogies mutations, whose effects may come from a discrete or
continuous distribution, occur. An analytical expression for the distribution of
trait values certainly does not exist.

However, it is possible at least in some cases to find the moment generating
function (mgf) of this distribution. The mgf for the trait values due to a
single nonrecombining locus is defined as
\begin{equation}
  \label{eq:mgfdef}
  \varphi_{\mathbf{Y}}(\mathbf{k}) = \E\left[ e^{\mathbf{k} \cdot \mathbf{Y}} \right] =
  \int e^{\mathbf{k} \cdot \mathbf{Y}} \Pro(\mathbf{Y}=\mathbf{y}) \mbox{d}\mathbf{y}.
\end{equation}
The vector $\mathbf{k}$ contains dummy variables for the integral transformation
of the probability distribution. This can be rewritten by conditioning on the
genealogy to give
\begin{align}
  \varphi_{\mathbf{Y}}(\mathbf{k}) &= \int e^{\mathbf{k} \cdot \mathbf{Y}}
  \int \Pro(\mathbf{Y}=\mathbf{y} | \mathbf{T}=\mathbf{t}) \Pro(\mathbf{T}=\mathbf{t})
  \mbox{d}\mathbf{t} \mbox{d}\mathbf{y}\\
  &= \int \int e^{\mathbf{k} \cdot \mathbf{Y}} \Pro(\mathbf{Y}=\mathbf{y} | \mathbf{T}=\mathbf{t}) \mbox{d}\mathbf{y}
  \Pro(\mathbf{T}=\mathbf{t})
  \mbox{d}\mathbf{t}.
\end{align}

To proceed it is necessary to make assumptions about the mutational process.
These are that mutations occur as a Poisson process along branches and that
mutations at a locus simply add to the effects of previous mutations (the
continuum-of-alleles model \citep{Kimura1965}). Under these assumptions, the
changes in the trait value along each branch are conditionally independent given
the branch lengths. \citet{Schraiber2015} note that this describes a compound
Poisson process. The mgf of a compound Poisson process with rate $\lambda$ over
time $t$ is $\exp(\lambda t (\psi(k)-1))$, where $\psi$ is the mgf of the
distribution of jumps. Additionally, the mgf of two completely correlated random
variables with the same marginal distribution is $\varphi_{X_1}(k_1+k_2)$, where
$\varphi_{X_1}$ is the mgf of the marginal distribution.

The mgf of the trait values is therefore
\begin{equation}
  \label{eq:fullmgf}
  \varphi_{\mathbf{Y}}(\mathbf{k}) = \prod_{\omega \in \mathcal{O}}
  \int \exp\left( \frac{\theta}{2} t_{\omega} \left( \psi\left(\sum_{a \in \omega}k_{a}\right) -1 \right)\right)
  \Pro(\mathbf{T}=\mathbf{t})\mbox{d}\mathbf{t}.
\end{equation}
This is simply the moment generating function for $\mathbf{T}$ with
$\frac{\theta}{2} \left( \psi(\sum_{i \in \omega}k_{\omega}) -1 \right)$
substituted for the dummy variable of branch $T_{\omega}$. Or,
\begin{equation}
  \label{eq:sub}
  \varphi_{\mathbf{T}}(\mathbf{s})\Bigr|_{s_{\omega}=\frac{\theta}{2} \left( \psi\left(\sum_{a \in \omega}k_{a}\right) -1 \right)}.
\end{equation}

Equation \eqref{eq:sub} shows that if the mgf of the distribution of branch
lengths is known, then the mgf of the trait values can be obtained through a
simple substitution. \citet{Lohse2011} derived genealogy mgfs for various
population models including migration and splitting of subpopulations. Using
their result for a single population it is possible to get equation (1) of
\citet{Schraiber2015} using the substitution defined by equation \eqref{eq:sub}.
The same could be done for models with migration between subpopulations although
the number of terms in the recursion for the genealogy mgf quickly becomes very
large. 

\section{The infinitesimal limit}
It is instructive to see how this very general model converges to the
infinitesimal when the right limits are taken. This is done by first
substituting Taylor series for the genealogical and mutational distributions in
equation \eqref{eq:fullmgf}. Taking the limits $L\T m_1 \to \mu$,
$L\T m_2\to \sigma^2$, $L\T m_i\to 0$ for $i>2$, and
$L^i\left(\T\right)^j \to 0$ for $i<j$ as $L\to \infty$ yields
\begin{equation}
  \label{eq:clt}
  \exp \left( \frac{\theta}{2} \sum_{\omega \in \Omega}E[t_{\omega}] \left( \mu \left(
  \sum_{a \in \omega} k_a\right) + \frac{\sigma^2}{2}\left( \sum_{a \in \omega}
  k_a\right)^2\right)\right).
\end{equation}
This is the mgf for a multivariate normal distribution where the expected trait
value is $E[T_{MRCA}] \mu$, the variance is $E[T_{MRCA}]\sigma^2$, and the
covariance between trait values in two individuals $a$ and $b$ is
$(E[\tau_{a+b}]) \sigma^2$. $\mu$ can be interpreted as the rate of change in
the mean trait value per generation per genome due to mutational pressure.
$\sigma^2$ can be interpreted as the rate of accumulation of variance in trait
values per generation per genome. Interestingly, the trait variance appears to be
proportional to the second moment of the mutational distribution and not the
variance. 

Since the distribution of trait values is normally distributed, any linear
combination of sampled trait values will be as well. This includes the
distributions of observable quantities like the differences in trait values from
some reference individual or from a sample mean. This provides additional
theoretical justification for studies using normal models to look for
differences in selection on quantitative traits between populations
\citep{Ovaskeinen2011,Praebel2013,Berg2014,Robinson2015}. 
\section{Low mutation rate approximation}



\bibliographystyle{genetics}
\bibliography{quant_gen}

\end{document}
