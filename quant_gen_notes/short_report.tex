\documentclass{article}
\usepackage{amsmath}
\usepackage{natbib}
\usepackage[textwidth=5.8in, textheight=8in]{geometry}
\usepackage{lipsum}
\usepackage[mathscr]{eucal}
\usepackage{graphicx}

\newcommand{\T}{\frac{\theta}{2}}
\newcommand{\E}{\mathrm{E}}
\newcommand{\Var}{\mathrm{Var}}
\newcommand{\Cov}{\mathrm{Cov}}

\begin{document}

\title{Quantitative trait distributions and the coalescent process}
\maketitle

\section{Introduction}

Evolutionary quantitative genetics almost always models phenotypes as normally
distributed. Indeed, it has been suggested that this is the defining
characteristic of the field \citep{Rice2004}. In this framework the phenotypic
distribution at all levels between individuals, families, populations, and
species is completely described by of a set of means, variances, covariances
\citep{Falconer1996}. Under the infinitesimal model, beginning with
\citet{Fisher1919}, the phenotypic trait is controlled by a very large number of
loci and the normality assumption well-justified.

The infinitesimal model has been a powerful tool in many areas of evolutionary
genetics. Among others, it has been used to find traits under selection in
natural populations \citep{Price1984}, determine which dimensions in phenotype
space are favored by females during sexual selection \citep{Blows2004}, and test
for phylogenetic signal in phenotypic evolution \citep{Freckleton2002}. Such
studies and other provide great biological insights without worrying about
potentially complex evolutionary dynamics at multiple loci because the entire
system can be described in terms of the first and second order moments.

However, heritable phenotypic variation is ultimately due to discrete mutations
at discrete loci, and how this variation is distributed depends on the
genealogies at these loci. Classical neutral models of phenotypic evolution
incorporate the effects of factors like population size and subdivision on
genealogies and therefore phenotypic variation \cietp{Lynch1986,Lande1992}.
These models have examined the dynamics of phenotypic evolution forwards in time
as a balance between mutation that creates variance, mutation that spreads it
among subpopulations, and fixation that removes it. The focus has been to find
equilibrium genetic variances and the rate at which these are approached.

The principle modeling framework for variation in genealogies is the coalescent
process with all its varied extensions \citep{Wakeley2008}, but few studies have
attempted to connect this theory with quantitative genetics. The most well known
of these is \citet{Whitlock1999}, who used a coalescent argument to show that
$Q_{ST}$ and $F_{ST}$ have the same expected value under general models of
population subdivision. \citet{Griswold2007} used coalescent simulations to
investigate the effects of shared ancestry and linkage disequilibrium on the
matrix of genetic variances and covariances between traits
($\mbox{\textbf{G}}$). They found that the eigenvalues of the expected matrix
and the expected eigenvalues can be quite different. Most recently,
\citet{Schraiber2015} developed a very general model of quantitative trait
evolution based on the coalescent that allows for an arbitrary distribution of
mutational effects and number of loci affecting the trait. They derive the
characteristic function for the distribution of phenotypic values in a sample
and show how such values can deviate strongly from normality when the number of
loci is small or the mutational distribution has fat tails.

The results of \citet{Schraiber2015} were derived for a panmictic, constant-size
population. Here, I generalize their work to an arbitrary distribution of
coalescent times for which their main result, the characteristic function of the
sampling distribution of phenotypic values, is a special case. I then show how
the normal, infinitesimal model arises by taking appropriate limits. I then
calculate the kurtosis of the phenotypic distribution to illustrate the
dependence of departures from normality on both the mutational and genealogical
distributions.

\section{Notation}
Before deriving anything, it is helpful to describe the various quantities of
interest and the notation used to describe them. The principle quantity of
interest are the quantitative trait values. Hereafter I'll simply refer to these
as trait values even though it might be more precise to call them breeding
values or additive genetic values. The random vector of trait values in the
sampled individuals is $\mathbf{Y}$, and an element $Y_a$ of $\mathbf{Y}$ is the
trait value in individual $a$. An important thing to remember about the trait
values is that they really refer to the change in the trait value since the most
recent common ancestor of the sample. Since we do not know what the trait value
of this ancestor was, these trait values are not directly observed. Rather, they
determine other summaries one might calculate from a sample of measured
phenotypes.

The random vector of branch lengths describing the genealogy at a locus is
$\mathbf{T}$, and an element $T_{a,b}$ of $\mathbf{T}$ is the branch length
subtending individuals $a$ and $b$. $\omega$ will be used to denote a particular
set of individuals and hence a branch on the genealogy, and $\Omega$ is the set
of all possible branches. For instance, if there are three sampled individuals,
$a$, $b$, and $c$, then $\mathbf{Y}=\{Y_a,Y_b,Y_c\}$,
$\mathbf{T}=\{T_a,T_b,T_c,T_{a,b},T_{a,c},T_{b,c}\}$, and
$\Omega=\{(a),(b),(c),(a,b),(a,c),(b,c)\}$. Realizations of the random trait
values and branch lengths use the same notation with lowercase letters.

Another useful piece of notation is for sums of branch lengths. Let $\tau_{a+b}$
be the random sum of all branches containing both $a$ and $b$. Conversely,
$\tau_{a/b}$ will be the random sum of all branches containing $a$ and not $b$.
Extensions of this for more than two individuals are used. The same notation is
used when referring to sets of branch indices. So $\Omega_{a+b}$ and
$\Omega_{a/b}$ would be the sets of branches summed to give the above lengths.

Other important genetic quantities are $L$, the number of loci at which
mutations affect the trait, and $\T$, the mutation rate. The moment generating
function for the distribution of mutational effects is $\psi()$ and the moments
of this distribution are $m_i$. Moment generating functions for the distribution
of branch lengths and distribution of trait values are $\varphi_{\mathbf{T}}$
and $\varphi_{\mathbf{Y}}$.

\section{The moment generating function for the distribution of trait values}


\bibliographystyle{genetics}
\bibliography{quant_gen}

\end{document}
