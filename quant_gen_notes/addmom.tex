We also calculate here some additional moments that have less clear
interpretations but are useful later on when calculating the expected population
kurtosis. The first of these is $E[Y_a^3Y_b]$. The terms of
\eqref{eq:mgf_approx_sub} that will appear in this are
\begin{equation*}
  L \left(\T\right) \frac{m_4}{24} 4 k_a^3k_b \sum_{\omega:a,b \in \omega} E[t_\omega]
\end{equation*}
and
\begin{equation*}
  L(L-1)\left(\T \frac{m_2}{2}\right)^2 k_a^2 \times 2k_ak_b
  \left( \sum_{\omega:a,b \in \omega} E[t_\omega] \right) \left( \sum_{\omega: a \in \omega} E[t_\omega] \right).
\end{equation*}
This ultimately gives
\begin{equation}
  \label{eq:m31}
  E[Y_a^3Y_b] = L \T m_4 E[\tau_{a+b}] + 3L(L-1) \left(\T m_2\right)^2 E[T_{MRCA}]E[\tau_{a+b}].
\end{equation}
The next fourth moment of interest is $E[Y_a^2Y_bY_c]$. The terms of
\eqref{eq:mgf_approx_sub} are
\begin{equation*}
  L \T \frac{m_4}{24} 12k_a^2k_bk_c \sum_{\omega: a,b,c \in \omega} E[t_\omega],
\end{equation*}
\begin{equation*}
  L(L-1) \left(\T \frac{m_2}{2}\right)^2 k_a^2 \times 2k_bk_c
  \left( \sum_{\omega: a \in \omega} E[t_\omega] \right)\left( \sum_{\omega: b,c \in \omega} E[t_\omega] \right),
\end{equation*}
and 
\begin{equation*}
  L(L-1) \left(\T \frac{m_2}{2}\right)^2 2k_ak_b \times 2k_ak_c \left( \sum_{\omega: a,b \in \omega} E[t_\omega] \right)
  \left( \sum_{\omega: a,c \in \omega} E[t_\omega] \right).
\end{equation*}
Taking the appropriate derivatives of these gives
\begin{equation}
  \label{eq:m211}
  L \T m_4 E[\tau_{a+b+c}] + L(L-1)\left(\T m_2\right)^2E[T_{MRCA}]E[\tau_{b+c}] +
  2L(L-1) \left(\T m_2\right)^2E[\tau_{a+b}]E[\tau_{a+c}].
\end{equation}
Individuals in the population are exchangeable as long as it is not structured.
The pairwise expected shared branch lengths are in that case all equal and we
can write \eqref{eq:m211} as
\begin{equation}
  \label{eq:m211s}
  L \T m_4 E[\tau_{a+b+c}] + L(L-1)\left(\T m_2\right)^2E[T_{MRCA}]E[\tau_{a+b}] +
  2L(L-1) \left(\T m_2\right)^2E[\tau_{a+b}]^2.
\end{equation}
Where $\tau_{a+b}$ just refers to the expected shared branch length of any two
individuals in the population. The final moment we'll look at is
$E[Y_aY_bY_cY_d]$ which has relevant terms
%% The set of branches containing all individuals
\begin{equation*}
  L \T \frac{m_4}{24} 24k_ak_bk_ck_d \sum_{\omega: a,b,c,d \in \omega} E[t_\omega],
\end{equation*}
%% Cross set of a-b branches and c-d branches
\begin{equation*}
  L(L-1) \left(\T \frac{m_2}{2}\right)^2 2k_ak_b \times 2k_ck_d \left( \sum_{\omega: a,b \in \omega} E[t_\omega] \right)
  \left( \sum_{\omega: c,d \in \omega} E[t_\omega] \right),
\end{equation*}
%% Cross set of a-c branches and b-d branches
\begin{equation*}
  L(L-1) \left(\T \frac{m_2}{2}\right)^2 2k_ak_c \times 2k_bk_d \left( \sum_{\omega: a,c \in \omega} E[t_\omega] \right)
  \left( \sum_{\omega: b,d \in \omega} E[t_\omega] \right),
\end{equation*}
and
%% cross set of a-d branches and b-c branches
\begin{equation*}
  L(L-1) \left(\T \frac{m_2}{2}\right)^2 2k_ak_d \times 2k_bk_c \left( \sum_{\omega: a,d \in \omega} E[t_\omega] \right)
  \left( \sum_{\omega: b,c \in \omega} E[t_\omega] \right).
\end{equation*}
When the appropriate fourth order partial derivatives are taken of this we get
\begin{align*}
  L \T m_4 E[\tau_{a+b+c+d}] \\
  + L(L-1)\left(\T m_2\right)^2E[\tau_{a+b}]E[\tau_{c+d}] \\
  + L(L-1)\left(\T m_2\right)^2E[\tau_{a+c}]E[\tau_{b+d}] \\
  + L(L-1)\left(\T m_2\right)^2E[\tau_{a+d}]E[\tau_{b+c}].
\end{align*}
We can again simplify this expression for populations with exchangeable
individuals. This gives
\begin{equation}
  \label{eq:m1111}
  L \T m_4 E[\tau_{a+b+c+d}] + 3L(L-1)\left(\T m_2\right)^2E[\tau_{a+b}]^2.
\end{equation}
%%% Local Variables:
%%% mode: latex
%%% TeX-master: "notes.tex"
%%% End: 
