As demonstrated by Whitlock we expect the trait variance between subpopulations
to be proportional to $\bar{t} - \bar{t}_0$ where $\bar{t}$ is the coalescence
times for pairs of lineages sampled at random from the metapopulation and
$\bar{t}_0$ is the expected coalescence time for pairs of lineages sampled
within subpopulations. The variance between subpopulations is defined as
\begin{equation}
  V_{\text{between}} = \sum_i c_i (\bar{Y}_i - \bar{Y})^2,
\end{equation}
where $c_i$ is the fraction of the meta population in subpopulation $i$. We
therefore need to calculate $\E[(\bar{Y}_i - \bar{Y})^2]$. We can break this up as
\begin{align}
  \E[(\bar{Y}_i - \bar{Y})^2] &= \E[(\frac{1}{N_i}\sum_k Y_{i,k})] \\
  &- 2\E[(\frac{1}{N_i}\sum_k Y_{i,k})(\frac{1}{N}\sum_j\sum_lY_{j,l})] \\
  &+ \E[(\frac{1}{N}\sum_j\sum_lY_{j,l})^2].
\end{align}
Computing the first part of this gives
\begin{equation}
  \E[(\frac{1}{N_i}\sum_k Y_{i,k})] = \frac{1}{N_i^2}(N_i\E[Y_{i,k}^2] + N_i(N_i-1)\E[Y_{i,k}Y_{i,j}])
  \approx \E[Y_{i,k}Y_{i,j}].
\end{equation}
Computing the second part gives
\begin{align}
  \E[(\frac{1}{N_i}\sum_k Y_{i,k})(\frac{1}{N}\sum_j\sum_lY_{j,l})] &=
  \frac{1}{N_i}\frac{1}{N}\E[(\sum_k Y_{i,k})(\sum_j\sum_lY_{j,l})] \\
  &= \frac{1}{N_i}\frac{1}{N}\sum_k\E[Y_{i,k}(\sum_j\sum_lY_{j,l})] \\
  &= \frac{1}{N_i}\frac{1}{N}\sum_k(\E[Y_{i,k}] + (N_i-1)\E[Y_{i,k}Y_{i,l}] +
  \sum_{i\neq j} N_j \E[Y_{i,k}Y_{j,l}]) \\
  &\approx c_i \E[Y_{i,k}Y_{i,l}] + \sum_{i\neq j} c_j \E[Y_{i,k}Y_{j,l}].
\end{align}
Computing the last part gives
\begin{align}
  \E[(\frac{1}{N}\sum_j\sum_lY_{j,l})^2] &= \frac{1}{N^2}\E[(\sum_j\sum_lY_{j,l})^2] \\
  &= \frac{1}{N^2}(N\E[Y_{i,k}^2] + \sum_iN_i(N_i)\E[Y_{i,k}Y_{i,l}] +
  \sum_{i,j:i\neqj} N_iN_j \E[Y_{i,k}Y_{j,l}]) \\
  &\approx \sum_{i,j} c_ic_j \E[Y_{i,k}Y_{j,l}].
\end{align}
Combining these three parts and canceling terms corresponding to the mean effect
of mutations and to the expected $T_{MRCA}$ gives an expression in terms of
expected pairwise coalescence times.
\begin{equation}
  \E[(\bar{Y}_i - \bar{Y})^2] \propto 2\sum_{j} c_j \E \tau_{ij} -
  \E \tau_{ii} - \sum_{i,j} c_ic_j \E \tau_{ij}.
\end{equation}
This then means that
\begin{align}
  V_{\text{between}} &\propto \sum_i c_i (2\sum_{j} c_j \E \tau_{ij} -
  \E \tau_{ii} - \sum_{i,j} c_ic_j \E \tau_{ij}) \\
  &= \sum_{i,j} c_ic_j \E \tau_{ij} - \sum_i c_i \E \tau_{ii} \\
  &= \bar{t} - \bar{t}_0.
\end{align}
%%% Local Variables:
%%% mode: latex
%%% TeX-master: "notes.tex"
%%% End: 
