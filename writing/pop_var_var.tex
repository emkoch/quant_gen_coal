The variance of the population variance over evoultionary realizations is
another useful quantity. In the infinitesimal limit for a sufficiently large
population, the variance of the variance will be negligable relative to the
variance itself. That is, the coefficient of variation of the variance will
approach zero. Here I will derive the variance of the variance, from which one
could compute the coefficient of variation.

\begin{align}
  \label{eq:popvarvar}
  \Var[V_g] &= \Var\left[ \frac{1}{N}\sum \left( Y_i - \frac{\sum Y_j}{N} \right)^2\right] \nonumber \\
            &= \frac{1}{N^2}\left( \sum \Var[(Y_i - \frac{\sum Y_j}{N})^2] + 2 \sum_{i \neq j} 
              \Cov[(Y_i - \frac{\sum Y_k}{N})^2, (Y_j - \frac{\sum Y_k}{N})^2]\right).
\end{align}
The first term can be ignored if the population size is large. The second term
is determined by the covariance in the squared deviations from the mean. The
covariance can be calculated by considering its individual parts. 

\begin{align*}
  &\Cov[Y_i^2, Y_j^2] \approx -2\Cov[Y_i^2,Y_j^2] \nonumber \\
  &\Cov[Y_i^2, -2Y_j \frac{\sum Y_k}{N}] \approx -2\Cov[Y_i^2,Y_jY_k] \nonumber \\
  &\Cov[Y_i^2, \left(\frac{\sum Y_k}{N}\right)^2] \approx \Cov[Y_i^2,Y_jY_k] \nonumber \\
  &\Cov[-2Y_i\left(\frac{\sum Y_k}{N}\right), Y_j^2] \approx -2\Cov[Y_i^2,Y_jY_k] \nonumber \\
  &\Cov[-2Y_i\left(\frac{\sum Y_k}{N}\right),-2Y_j\left(\frac{\sum Y_k}{N}\right)] 
  \approx 4\Cov[Y_iY_j,Y_kY_l] \nonumber \\
  &\Cov[-2Y_i\left(\frac{\sum Y_k}{N}\right), \left(\frac{\sum Y_k}{N}\right)] 
  \approx -2\Cov[Y_iY_j,Y_kY_l] \nonumber \\
  &\Cov[\left(\frac{\sum Y_k}{N}\right)^2, Y_j^2] 
  \approx \Cov[Y_j^2,Y_iY_k] \nonumber \\
  &\Cov[\left(\frac{\sum Y_k}{N}\right)^2,-2Y_j\left(\frac{\sum Y_k}{N}\right)] 
  \approx -2\Cov[Y_iY_j,Y_kY_l] \nonumber \\
  &\Cov[\left(\frac{\sum Y_k}{N}\right)^2,\left(\frac{\sum Y_k}{N}\right)^2] 
  \approx \Cov[Y_iY_j,Y_kY_l] \nonumber
\end{align*}

Combining all these gives 
\begin{equation}
  \Cov[(Y_i - \frac{\sum Y_k}{N})^2, (Y_j - \frac{\sum Y_k}{N})^2] = 
  \E[Y_i^2Y_j^2] - \E[Y_i^2]^2 - 2\E[Y_i^2Y_jY_k] + 2\E[Y_i^2]E[Y_jY_k] +
  \E[Y_iY_jY_kY_l] - E[Y_iY_j]^2.
\end{equation}
After calculating the various moments included in this sum and using the low
mutation rate approximation we get
\begin{equation}
  L \T m_4 \left( \frac{\E[T_{2,4}]}{9} + \frac{\E[T_{3,4}]}{6} \right).
  \label{eq:exppopvarvar}
\end{equation}
With this, this coefficient of variation of the variance is
\begin{equation}
  \sqrt{\frac{\kappa}{L \T \E[T_{2,2}]}} \sqrt{\frac{\frac{\E[T_{2,4}]}{9} + \frac{\E[T_{3,4}]}{6}}{\E[T_{2,2}]}}.
\end{equation}
The first term depends on the genetic architecture while the second depends on
demography. From the first we can see that the coefficient of variation of
variance decreases with the square root of the expected number of trait
affecting separating two halpotypes. 

%%% Local Variables: 
%%% mode: latex
%%% TeX-master: "notes.tex"
%%% End: 
