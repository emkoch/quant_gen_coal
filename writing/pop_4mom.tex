A good measure for the deviation from a normal distribution is the kurtosis,
which is the ratio of the fourth central moment to the square of the variance.
For the normal distribution this value is three. The kurtosis measures the
propensity of a distribution to produce outliers. By calculating the expected
kurtosis for a given genetic architecture and population history we could assess
the propensity of those parameters to produce trait distributions (at the
population level) with frequent outliers. 

However, the expected kurtosis in the population is a quotient and therefore
very annoying to calculate. We could approximate the kurtosis by taking the
expected fourth moment divided by the expectation of the square of the variance.
A better approximation could be obtained through a Taylor expansion, but that
would involve calculation of trait moments of order eight and greater. 

Instead we will calculate the expected fourth central moment relative to the
expected fourth moment in a normal distribution. 
\begin{equation}
  E[M_{4,Y}] = E\left[\frac{1}{N} \sum \left( Y_i - \frac{\sum Y_j}{N} \right)^4 \right].
\end{equation}
Examining the sum inside the expectation we see that 
\begin{align}
  \label{eq:expandkurt}
  E\left[ \left( Y_i - \frac{\sum Y_j}{N} \right)^4 \right] &= E[Y_i^4] - 
                                                              4E\left[ Y_i^3 \frac{\sum Y_j}{N} \right] + 
                                                              6E\left[Y_i^2\left(\frac{\sum Y_j}{N}\right)^2\right] - 
                                                              4E\left[Y_i\left(\frac{\sum Y_j}{N}\right)^3\right]+ 
                                                              \left(\frac{\sum Y_j}{N}\right)^4 \nonumber \\
                                                            &= E[Y_i^4] - 
                                                              \frac{4}{N}\sum_j E[Y_i^3Y_j] + 
                                                              \frac{6}{N^2}\sum_{j,k} E[Y_i^2Y_jY_k] \nonumber\\
                                                              &-\frac{4}{N^3}\sum_{j,k,l}E[Y_iY_jY_kY_l] + 
                                                              \frac{1}{n^4}\sum_{j,k,l,d}E[Y_jY_kY_lY_d].
\end{align}
In calculating these expectations we have to remember that the value depends
only on the number of times each variable appears in the expectation. That is,
$E[Y_1^2Y_2Y_3]$ is equivalent to $E[Y_1Y_2^3Y_3]$ as long as all individuals in
the population are exchangeable. The resulting expansion of
\eqref{eq:expandkurt} is therefore quite ugly. It can be simplified by only
considering terms of order one. Other terms can be ignored since we are
assuming there are a fair number of individuals in the population. This yields
\begin{align}
  \label{eq:popkurt}
  E\left[ \left( Y_i - \frac{\sum Y_j}{N} \right)^4 \right] &=
  E[Y_i^4]  - \frac{4(n-1)}{n}E[Y_i^3Y_j] + \frac{6(n-1)(n-2)}{n^2}E[Y_i^3Y_jY_k]  \nonumber \\
  &- \frac{4(n-1)(n-2)(n-3)}{n^3}E[Y_iY_jY_kY_l] \nonumber \\
  &+ \frac{(n-1)(n-2)(n-3)(n-4)}{n^4}E[Y_jY_kY_lY_d]
  + O(n^{-1}) \nonumber \\
  &\approx E[Y_i^4]  - 4E[Y_i^3Y_j] + 6E[Y_i^2Y_jY_k] - 3E[Y_iY_jY_kY_l].
\end{align}

The first term, $E[Y_i^4]$ was derived in equation \eqref{eq:mom4} as
\begin{equation*}
  L\T m_4 E[T_{MRCA}] + 3L(L-1)\left( \T m_2 E[T_{MRCA}] \right)^2.
\end{equation*}
The second term, $E[Y_i^3Y_j]$ was derived in equation \eqref{eq:m31} as
\begin{equation*}
  L \T m_4 E[\tau_{a+b}] + 3L(L-1) \left(\T m_2\right)^2 E[T_{MRCA}]E[\tau_{a+b}].
\end{equation*}
The third term, $E[Y_i^2Y_jY_k]$ was derived in equation \eqref{eq:m211} as
\begin{equation*}
  L \T m_4 E[\tau_{a+b+c}] + L(L-1)\left(\T m_2\right)^2E[T_{MRCA}]E[\tau_{a+b}] +
  2L(L-1) \left(\T m_2\right)^2E[\tau_{a+b}]^2.
\end{equation*}
The fourth term, $E[Y_iY_jY_kY_l]$ was derived in equation \eqref{eq:m1111} as
\begin{equation*}
  L \T m_4 E[\tau_{a+b+c+d}] + 3L(L-1)\left(\T m_2\right)^2E[\tau_{a+b}]^2.
\end{equation*}
Plugging these into \eqref{eq:popkurt} we get
\begin{align}
  \label{eq:popm4coal}
  E[M_{4,Y}] &\approx L \T m_4 \left( E[T_{MRCA}] - 4E[\tau_{a+b}] +
  6E[\tau_{a+b+c}] -3E[\tau_{a+b+c+d}] \right)\nonumber \\ &+ 3\left( L \T m_2
  \right)^2\left(E[T_{MRCA}]- E[\tau_{a+b}]\right)^2.
\end{align}
These expressions approximate $L(L-1)$ as $L^2$ as part of the low mutation rate
approximation. The fourth central moment of a normal distribution is
$3\sigma^4$. Using the expected population variance we get $3\left(L\T m_2
E[T_{MRCA}]- E[\tau_{a+b}]\right)^2$. It is clear from \eqref{eq:popm4coal} that
as $L\T$ gets large the second term will dominate and the population will have
the same expected fourth moment as a normal distribution. The ratio of the
expected fourth moment to that under normality is
\begin{equation*}
  \frac{\E[M_4]}{3\left(L\T m_2 E[T_{MRCA}]- E[\tau_{a+b}]\right)} \approx 3 +
  \frac{\mbox{Kurt}[M]( E[T_{MRCA}] - 4E[\tau_{a+b}] + 6E[\tau_{a+b+c}]
    -3E[\tau_{a+b+c+d}])}{L \T E[T_{MRCA}]- E[\tau_{a+b}]}.
\end{equation*}
This expression can be written in an easier to interpret form by noting that
$E[\tau_{a+b}]=E[T_{MRCA}] - E[T_2]$, $E[\tau_{a+b+c}]=E[T_{MRCA}] - E[T_3]$,
$E[\tau_{a+b+c+d}]=E[T_{MRCA}] - E[T_4]$, where $T_i$ is the expected time it
takes for $i$ lineages to coalesce.
\begin{equation}
  \label{eq:popkurtcoal}
  3 + \frac{\mbox{Kurt}[M]( 4T_2 - 6T_3 + 3T_4)}{L \T T_2^2}.
\end{equation}
In a constant size population this would be $3 + \frac{\mbox{Kurt}[M]}{2L\T}$
%%% mode: latex
%%% TeX-master: "notes.tex"
%%% End: 
