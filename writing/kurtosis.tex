Since we know that the variance in trait values is very close under the low
mutation rate model as in the normal limit, we might next compare kurtosis which
provides a measure of the tailedness of the distribution. The kurtosis is
defined as
\begin{equation}
  Kurt[X]=\frac{E[(X-E[X])^4]}{(E[(X-E[X])^4)^2}.
\end{equation}
This is the fourth central moment divided by the variance. For ease of
calculation, we'll examine this in the case where the mean mutation effect (and
therefore trait value) is zero. If we plug \eqref{eq:var} and \eqref{eq:mom4}
into the expression for the kurtosis we get
\begin{align}
  \label{eq:kurtosis_single}
  Kurt[Y_a] &= \frac{L\T m_4 E[T_{MRCA}]}{\left(L\T m_2 E[T_{MRCA}]\right)^2} +
  \frac{3L(L-1)\left( \T m_2  E[T_{MRCA}]\right)^2}{\left(L\T m_2 E[T_{MRCA}]\right)^2} \nonumber \\
  &= \frac{m_4}{L\T m_2^2E[T_{MRCA}]} + \frac{3(L^2-L)}{L^2} \nonumber \\
  &= \frac{Kurt[M]}{L\T E[T_{MRCA}]} + 3\left( 1 - \frac{1}{L} \right).
\end{align}
What does this mean? The normal approximation will have a kurtosis of $3$, so
having a finite number of loci will act to decrease the kurtosis of the trait
distribution. However, it seems this decrease will only be very slight. The term
on the left shows that the kurtosis is increased by the ratio of the kurtosis of
the mutational distribution relative to the expected number of segregating sites
affecting the trait.
%%% Local Variables:
%%% TeX-master: "notes.tex"
%%% End:
