\newcommand{\AAA}{\E[\mathbbm{T}_{4,4}] + \frac{1}{3}\E[\mathbbm{T}_{3,4}] + \frac{2}{9}\E[\mathbbm{T}_{2,4}]}
\newcommand{\BBB}{\frac{1}{9}\E[\mathbbm{T}_{2,4}] + \frac{1}{6}\E[\mathbbm{T}_{3,4}]}
\newcommand{\CCC}{\E[\mathbbm{T}_{4,4}] - \frac{1}{6}\E[\mathbbm{T}_{3,4}] - \frac{1}{9}\E[\mathbbm{T}_{2,4}]}

For most population genetic models and reasonable sample sizes, the recursive
nature of the trait distribution mgf makes it computationally unfeasible to
solve under general parameter values. However, it is not necessary to have an
expression for the full mgf in order to derive moments of the trait distribution
in terms of moments of branch lengths and mutational effects. Under the low
mutation rate approximation moments can be calculated by differentiating
equation \eqref{eq:lowmut}. Even without making this approximation, moments can
be calculated by taking Taylor expansions in equation \eqref{eq:cond} and only
considering terms contributing to the desired moment's order. We implemented a
symbolic math program to calculate trait moments using this procedure, and the
details are given in Appendix ~\ref{symmath}. As the normal distribution is
completely defined by its first two moments, the extent to which a trait
distribution deviates from normality can be measured by the extent to which its
moments deviate from those of a normal distribution with the same mean and
variance.

We have so far considered the distribution of a trait value $Y_a$ over
evolutionary realizations. The expectation of $Y_a$ is $L \T m_1 \E[T_{MRCA}]$,
and the variance is $L \T m_1 \E[T_{MRCA}] + L(m_1^2\T)\Var[T_{MRCA}]$. Although
simple to derive using computer algebra, expressions for the higher central
moments of $Y_a$ are complicated even under the low mutation rate approximation,
and there is not much to be gained by showing them here.

However, in a given evolutionary realization there will be a distribution of
trait values in the population. The population-level trait distribution can also
be described by its moments, but since this distribution is random the moments
at the population level are also random quantities. Since $Y_a$ is relative to a
value that is not directly observed, the expected population-level moments offer
more insight. In particular, we are interested in how the trait distribution at
the population level might deviate from normality. ~\citet{Schraiber2015}
computed the expected first four central moments of a constant-size population.
We derived the same expectations under an arbitrary demographic history,
\begin{subequations} \label{eq:emoms}
\begin{align}
  &\E[M_2] = L \T \E[\mathbbm{T}_{2,2}] m_2 \label{eq:emoms2}\\
  &\E[M_3] = L \T \E[\mathbbm{T}_{3,3}] m_3  \label{eq:emoms3}\\
  &\E[M_4] = 3\left(L \T \E[\mathbbm{T}_{2,2}] m_2\right)^2 \nonumber \\
  &+ 3L \left(\T m_2\right)^2\Var[\mathbbm{T}_{2,2}] + \frac{1}{3}
  L \left(\T m_2\right)^2
    \left( \frac{11}{9} \E[\mathbbm{T}_{2,4}^2]-\frac{1}{3}\E[\mathbbm{T}_{2,4}\mathbbm{T}_{3,4}]-
    \frac{1}{4}\E[\mathbbm{T}_{3,4}^2]\right) \nonumber \\
  &+ L m_4 \T ( \E[\mathbbm{T}_{4,4}] + \frac{1}{3} \E[\mathbbm{T}_{3,4}] +
    \frac{2}{9} \E[\mathbbm{T}_{2,4}] ).
  \label{eq:emoms4}
\end{align}
\end{subequations}
The normal limit corresponds to equation \eqref{eq:emoms3} and the second two
lines of equation \eqref{eq:emoms4} going to zero. To give insight into the
expressions in equation \eqref{eq:emoms}, moment calculations done by hand under
the low mutation rate approximation are presented in Appendix \ref{moments}.

Equation \eqref{eq:emoms2} gives the expected trait variance in the population.
However, the amount of variance will vary over realizations of the evolutionary
process. The variation in the population variance depends on the sparsity of the
trait and the number of causal loci. A certain expected variance can arise
either by multiple mutations at one locus or by multiple mutations each at a
different, independent locus. The variation in the variance can be quantified
using its coefficient of variation (CVV), the standard deviation of the variance
divided by its expectation. For a constant-size, panmictic population
\begin{equation}
  \label{eq:cvv}
  \mathrm{CVV} = \sqrt{\frac{4}{3}\frac{1}{L} +
    \frac{1}{6}\frac{m_2/m_4}{L\theta \E[\mathbbm{T}_{2,2}]}}.
\end{equation}
Equation \eqref{eq:cvv} shows a contribution due to linkage that decreases like
$1/\sqrt{L}$ and a contribution due to sparsity that decreases like
$1/\sqrt{L\theta \E[\mathbbm{T}_{2,2}]}$. Even when the sparsity is low, i.e.,
when a large number of variants affect the trait, if the trait is only
controlled by a single locus there will be considerable variation in the
population variance (CVV$=\sqrt{\frac{4}{3}}$). On the other hand, the CVV of a
sparse trait controlled by many loci will depend on the ratio of the second and
third non-central moments of the mutational distribution ($m_2/m_4$). For a
mutational distribution with mean zero this is equivalent to the kurtosis. 
%%% Local Variables:
%%% TeX-master: "short_report.tex"
%%% End:
