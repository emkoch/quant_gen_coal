\newcommand{\AAA}{\E[\mathbbm{T}_{4,4}] + \frac{1}{3}\E[\mathbbm{T}_{3,4}] + \frac{2}{9}\E[\mathbbm{T}_{2,4}]}
\newcommand{\BBB}{\frac{1}{9}\E[\mathbbm{T}_{2,4}] + \frac{1}{6}\E[\mathbbm{T}_{3,4}]}
\newcommand{\CCC}{\E[\mathbbm{T}_{4,4}] - \frac{1}{6}\E[\mathbbm{T}_{3,4}] - \frac{1}{9}\E[\mathbbm{T}_{2,4}]}

For most population genetic models and for sample sizes greater than three, the
recursive nature of $\varphi_{\mathbf{Y}}$ makes it computationally
infeasible to obtain general solutions. However, it is possible to derive
moments of the trait distribution in terms of moments of branch lengths and
mutational effects without an expression for the full mgf. Under the low
mutation rate approximation, moments can be calculated by differentiating
equation \eqref{eq:lowmut}. Even without this approximation, moments can be
calculated using equation \eqref{eq:cond} and only considering terms in a Taylor
expansion contributing to the desired moment's order. We implemented a symbolic
math program to calculate trait moments which can be found at
\url{https://github.com/emkoch/trait_mgf_math}. Details of the procedure are
given in Appendix \ref{symmath}. The normal distribution is completely defined
by its first two moments, and the extent to which a trait distribution deviates
from normality can be measured by the extent to which its moments deviate from
those of a normal distribution with the same mean and variance.

Considering the distribution over evolutionary realizations, the expectation of
$Y_a$ is $L \T m_1 \E[T_{MRCA}]$, and the variance is $L \T m_2 \E[T_{MRCA}] +
L(m_1\T)^2\Var[T_{MRCA}]$. Although simple to derive using computer algebra,
expressions for the higher central moments of $Y_a$ are complicated even under
the low mutation rate approximation, and not much is gained by showing them
here.

In a given evolutionary realization there will be a distribution of trait values
within the entire population which can also be described by its moments. Since
this distribution is itself random, the moments of the within-population
distribution are random as well. Since $Y_a$ is defined relative to an
unobservable value, the expected central moments within the whole population
offer more insight. We are particularly interested in how trait distributions
deviate from normality. ~\citet{Schraiber2015} computed the expected first four
central moments of a constant-size population. We derive the same expectations
for an arbitrary demographic history,
\begin{subequations} \label{eq:emoms}
\begin{align}
  &\E[M_2] = L \T \E[\mathbbm{T}_{2,2}] m_2 \label{eq:emoms2}\\
  &\E[M_3] = L \T \E[\mathbbm{T}_{3,3}] m_3  \label{eq:emoms3}\\
  &\E[M_4] = 3\left(L \T \E[\mathbbm{T}_{2,2}] m_2\right)^2 \nonumber \\
  &+ 3L \left(\T m_2\right)^2\Var[\mathbbm{T}_{2,2}] + \frac{1}{3}
  L \left(\T m_2\right)^2
    \left( \frac{11}{9} \E[\mathbbm{T}_{2,4}^2]-\frac{1}{3}\E[\mathbbm{T}_{2,4}\mathbbm{T}_{3,4}]-
    \frac{1}{4}\E[\mathbbm{T}_{3,4}^2]\right) \nonumber \\
  &+ L m_4 \T ( \E[\mathbbm{T}_{4,4}] + \frac{1}{3} \E[\mathbbm{T}_{3,4}] +
    \frac{2}{9} \E[\mathbbm{T}_{2,4}] ).
  \label{eq:emoms4}
\end{align}
\end{subequations}
In the infinitesimal limit equation \eqref{eq:emoms3} and the second two lines
of equation \eqref{eq:emoms4} go to zero. To give more insight into equation
\eqref{eq:emoms}, moment calculations done by hand under the low mutation rate
approximation are presented in Appendix \ref{moments}.

Equation \eqref{eq:emoms2} gives the expected variance in the population, but
this will vary over realizations of the evolutionary process. The variation in
the population variance depends on the expected sparsity of the trait ($L\T
\E[\mathbbm{T}_{2,2}]$) and the number of causal loci. The variation in the
variance can be quantified using its coefficient of variation (CVV): the
standard deviation of the variance divided by its expectation. For a
constant-size, panmictic population
\begin{equation}
  \label{eq:cvv}
  \mathrm{CVV} = \sqrt{\frac{4}{3}\frac{1}{L} +
    \frac{1}{6}\frac{m_4/m_2^2}{L\theta \E[\mathbbm{T}_{2,2}]}}.
\end{equation}
Equation \eqref{eq:cvv} has a contribution due to linked mutations occurring at
a single locus ($\frac{4}{3}\frac{1}{L}$) and a contribution due to sparsity and
mutation ($\frac{1}{6}\frac{m_4/m_2^2}{L\theta \E[\mathbbm{T}_{2,2}]}$). Even
when the sparsity is low, i.e., when a large number of variants affect the
trait, if the trait is only controlled by a single locus there will be
considerable variation in the within-population variance
(CVV$=\sqrt{\frac{4}{3}}$). On the other hand, the CVV of a sparse trait
controlled by many loci will depend on the ratio of the fourth and squared
second non-central moments of the mutational distribution ($m_4/m_2^2$).

%%% Local Variables:
%%% TeX-master: "quant_gen_manu.tex"
%%% End:
