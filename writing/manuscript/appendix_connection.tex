We can reproduce equation (1) of \citet{Schraiber2015} using equation
\eqref{eq:sub} of this paper and equation (5) of \citet{Lohse2011}. Equations
will be rewritten using the notation of this paper. Equation (5) of
\citet{Lohse2011} is a recursion for finding the mgf of a sample of size $n$
lineages from a panmictic population:
\begin{equation}
  \varphi^n_{\mathbf{T}}(\mathbf{s}, \Upsilon)=
  \frac{1}{{n\choose 2}+\sum_{\omega \in \Upsilon}s_\omega}
  \sum_{(\omega_i, \omega_j) \in \Upsilon}\varphi^{n-1}_{\mathbf{T}}
  (\mathbf{s}, \mathcal{C}(\Upsilon, \omega_i, \omega_j)).
\end{equation}
Here, $\Upsilon$ is the set of all external branches of the genealogy, and
$\mathcal{C}(\Upsilon,\omega_i,\omega_j)$ is an operator that removes $\omega_i$
and $\omega_j$ from $\Upsilon$ and adds $\omega_i \cup \omega_j$. Following
equation \eqref{eq:sub} we make the substitution $s_{\omega}=\frac{\theta}{2}
\left( \psi\left(\sum_{i \in \omega}k_{i}\right) -1 \right)$ to get
\begin{equation}
  \varphi^n_{\mathbf{Y}}(\mathbf{k}) =
  \frac{2}{n(n-1) + \theta(\sum_{u=1}^n \psi(k_u)-n)}
  \sum_{(\omega_i, \omega_j) \in \Upsilon}\varphi^{n-1}_{\mathbf{T}}
  (\mathbf{s}, \mathcal{C}(\Upsilon, \omega_i, \omega_j))
  \Bigr|_{s_{\omega}=\frac{\theta}{2} \left( \psi\left(\sum_{i \in \omega}k_{i}\right) -1 \right)}.
\end{equation}
Next we can write
\begin{equation}
\varphi^{n-1}_{\mathbf{T}}
  (\mathbf{s}, \mathcal{C}(\Upsilon, \omega_i, \omega_j))
\Bigr|_{s_{\omega}=\frac{\theta}{2} \left( \psi\left(\sum_{i \in \omega}k_{i}\right) -1 \right)} =
\varphi^{n-1}_{\mathbf{Y}}(\mathscr{C}(\mathbf{k},i,j)).
\end{equation}
The operator $\mathscr{C}(\mathbf{k},i,j)$ removes the $i^{th}$ and $j^{th}$
elements of $\mathbf{k}$ and adds the term $k_i+k_j$. We can use this simple
operator because the order of elements in $\mathbf{k}$ does not matter for a
panmictic population. This yields
\begin{equation}
  \label{eq:schraiber}
  \varphi^n_{\mathbf{Y}}(\mathbf{k}) =
  \frac{2}{n(n-1) + \theta(\sum_{u=1}^n \psi(k_u)-n)}
  \sum_{i, j\leq n, i \neq j} \varphi^{n-1}_{\mathbf{Y}}(\mathscr{C}(\mathbf{k},i,j)).
\end{equation}
This recursive expression is equivalent to equation (1) in \citet{Schraiber2015}.

The complicated appearance of equation \eqref{eq:schraiber} is mostly due to the
need for novel notation. A simple example for a sample of three lineages shows
how the substitution process works. The mgf for a genealogy of sample size three
from a panmictic population is
\begin{align*}
  \varphi_{\mathbf{T}} &= \frac{1}{3 - s_{\{a\}} - s_{\{b\}} - s_{\{c\}}} \\
  &\times \left( \frac{1}{1 - s_{\{a,b\}} - s_{\{c\}}} + \frac{1}{1 - s_{\{a,c\}} - s_{\{b\}}}  + \frac{1}{1 - s_{\{b,c\}} - s_{\{a\}}}\right),
    \end{align*}
and the mgf for the trait distribution after substitution is
\begin{align*}
  \varphi_{\mathbf{Y}} &= \biggl(\frac{1}{3 - \T(\psi(k_a) + \psi(k_b) + \psi(k_c) - 3)} \\
  &\times \biggl( \frac{1}{1 - \T(\psi(k_a+k_b) + \psi(k_c) - 2)} +
  \frac{1}{1 - \T(\psi(k_a+k_c) + \psi(k_b) - 2)} \\
  &+\frac{1}{1 - \T(\psi(k_b+k_c) + \psi(k_a) - 2)}\biggr)
  \biggr)^L.
\end{align*}
%%% Local Variables:
%%% TeX-master: "short_report.tex"
%%% End:
