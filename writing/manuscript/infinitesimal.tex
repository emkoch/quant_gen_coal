This model converges to a normal distribution when we take an infinitesimal
limit. We accomplish this by substituting Taylor series for the genealogical and
mutational distributions in equation \eqref{eq:fullmgf} and taking appropriate
limits (see Appendix \ref{clt}). The infinitesimal limit lets mutation effect
sizes become small as the number of loci becomes large. The resulting
distribution of a trait value $Y$ is multivariate normal with expectation
$\E[T_{MRCA}] \T \mu_1$ and variance $\E[T_{MRCA}] \T \mu_2$. The covariance
between trait values in individuals $a$ and $b$ is $\E[\tau_{a+b}] \T \mu_2$,
where $\tau_{a+b}$ is the shared branch length between a lineage sampled from
individual $a$ and a lineage from individual $b$. $\E[T_{MRCA}]$ is the expected
time to the most recent common ancestor. $\T \mu_1$ is the per genome rate that
mutational bias shifts the mean, and $\T \mu_2$ is the rate per genome that
variance accumulates. The infinitesimal limit requires that the products of $L$
and moments three and greater of the mutational distribution go to zero. This
can be thought of as requiring the mutational distribution to not bex extremely
skewed or heavy tailed. Derivation details are given in Appendix \ref{clt}.

Interestingly, the rate of variance accumulation is proportional to the second
moment of the mutational distribution instead of the variance. We can see the
intuition for this by considering a degenerate distribution where each mutation
has the same effect. We still expect variation among individuals due to
differences in the mutation count each individual receives, even though the
variance of the mutational distribution is zero. The variance among individual
trait values thus has one component due to differences in the number of
mutations and an additional component due to differences in the effects of these
mutations. The first component is proportional to the square of the mean
mutational effect, while the second is proportional to the mutational variance.
Therefore, the sum of the two components is proportional to $m_2$, the mean
squared effect.

Since the trait values are normally distributed, any linear combination of
sampled trait values will be as well. This includes observable quantities like
the differences in trait values from a reference individual or from a sample
mean. The distribution of trait differences between individuals is multivariate
normal with mean zero and covariance between any pair of trait differences given
by
\begin{equation}
  \label{eq:normcov}
  \Cov[Y_a-Y_b,Y_c-Y_d] = \mu_  2\T\left( \E[\mathcal{T}_{a,d}] + \E[\mathcal{T}_{b,c}] -
                         \E[\mathcal{T}_{a,c}] - \E[\mathcal{T}_{b,d}] \right),
\end{equation}
where $\mathcal{T}_{a,b} = 0$ if $a=b$. Classical theory in quantitative
genetics uses a univariate normal distribution of phenotypes in a panmictic
population. We can recover this by considering a population of exchangeable
individuals. In this case $\E[\mathcal{T}_{a,b}] = \E[\mathbbm{T}_{2,2}]$ for
all pairs $a \neq b$. Individual trait values are then conditionally independent
given the mean value in the population and are normally distributed with
variance $\E[\mathbbm{T}_{2,2}]\T\mu_2$.

The normal model in the infinitesimal limit provides additional theoretical
justification for studies using normal models to look for differences in
selection on quantitative traits between populations
\citep{Ovaskainen2011,Praebel2013,Robinson2015}. Additionally, equation
\eqref{eq:normcov} implies that a covariance matrix based on mean pairwise
coalescent times rather than population split times should be used in
phylogenetic models of neutral trait evolution \citep{Mendes2018}.

%%% Local Variables:
%%% TeX-master: "quant_gen_manu.tex"
%%% End:
 
