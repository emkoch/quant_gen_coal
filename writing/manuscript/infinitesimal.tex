This general model converges to a normal distribution when we take the
infinitesimal limit. We accomplish this by first substituting a Taylor series
for the genealogical and mutational distributions in equation \eqref{eq:fullmgf}
(see Appendix \ref{clt}). The infinitesimal limit corresponds to the situation
where the effect sizes of mutations become small as the number of loci becomes
large. The resulting distribution is multivariate normal where the expected
trait value is $E[T_{MRCA}] \T \mu_1$, the variance is $E[T_{MRCA}] \T \mu_2$,
and the covariance between trait values in two individuals $a$ and $b$ is
$E[\tau_{a+b}] \T \mu_2$. $E[T_{MRCA}]$ is the expected time to the most recent
common ancestor in the sample or population. $\T \mu_1$ is the rate per unit
time per genome that mutational bias shifts the mean. $\T \mu_2$ is the rate per
unit time per genome that variance accumulates. This limit requires that the
products of $L$ and moments three and greater of the mutational distribution go
to zero as the number of loci becomes large and the effect size per mutation
becomes small. This can be thought of as requiring the mutational distribution
to not have too heavy of tails. Details of the derivation are given in
Appendix \ref{clt}.

Interestingly, the rate of variance accumulation is proportional to the second
moment of the mutational distribution instead of the variance. We can see the
intuition for this by considering a degenerate distribution where each mutation
has the same effect. Here, we would still expect variation among individuals due
to differences in the number of mutations each individual receives, even though
the variance of the mutational distribution is zero. The variance among
individual trait values thus has one component due to differences in the number
of mutations and an additional component due to differences in the effects of
these mutations. The first component is proportional to the square of the mean
mutational effect, while the second is proportional to the mutational variance.
Therefore, the sum of the two components is proportional to $m_2$, the mean
squared effect.

Since the trait values are normally distributed, any linear combination of
sampled trait values will be as well. This includes the distributions of
observable quantities like the differences in trait values from a reference
individual or from a sample mean. The distribution of trait differences between
individuals is multivariate normal with mean zero and covariance between any
pair of trait differences given by
\begin{equation}
  \label{eq:normcov}
  \Cov[Y_i-Y_j,Y_k-Y_l] = \mu_  2\T\left( \E[\mathcal{T}_{i,l}] + \E[\mathcal{T}_{j,k}] -
                         \E[\mathcal{T}_{i,k}] - \E[\mathcal{T}_{j,l}] \right),
\end{equation}
where $\mathcal{T}_{i,j} = 0$ if $i = j$. Classical theory in quantitative
genetics uses a univariate normal distribution of phenotypes in a panmictic
population. We can recover this by considering a population of exchangeable
individuals. In this case $\E[\mathcal{T}_{i,j}] = \E[\mathbbm{T}_{2,2}]$ for all pairs
$i \neq j$. Individual trait values are then conditionally independent given the
mean value in the population and are normally distributed with variance
$\E[\mathbbm{T}_{2,2}]\T\mu_2$.

The normal model in the infinitesimal limit provides additional theoretical
justification for studies using normal models to look for differences in
selection on quantitative traits between populations
\citep{Ovaskainen2011,Praebel2013,Robinson2015}. Additionally, equation
\eqref{eq:normcov} implies that a covariance matrix based on mean pairwise
coalescent times rather than population split times should be used in
phylogenetic models of neutral trait evolution.

%%% Local Variables:
%%% TeX-master: "quant_gen_manu.tex"
%%% End:
 
