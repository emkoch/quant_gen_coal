This general model converges to a normal model when we take the infinitesimal
limit. We accomplish this by first substituting a Taylor series for the
genealogical and mutational distributions in equation \eqref{eq:fullmgf} (see
Appendix \ref{clt}). The infinitesimal limit corresponds to the situation where
the effect sizes of mutations becomes small as the number of loci becomes large.
The resulting distribution is multivariate normal where the expected trait value
is $E[T_{MRCA}] \T \mu$, the variance is $E[T_{MRCA}] \T \sigma^2$, and the
covariance between trait values in two individuals $a$ and $b$ is
$E[\tau_{a+b}] \T \sigma^2$. This limit requires that the products of $L$ and
moments three and greater of the mutational distribution go to zero as the
number of loci becomes large and the effect size per mutation becomes small. We
can think of this as requiring the mutational distribution to not have too heavy
of tails. Details of the derivation are given in Appendix \ref{clt}.

In this limiting normal distribution, $\T \mu$ can be interpreted as the rate of
change in the mean trait value per time unit per genome due to mutational
pressure. $\T \sigma^2$ can be interpreted as the rate of accumulation of
variance in trait values per unit time per genome. Interestingly, the rate of
variance accumulation is proportional to the second moment of the mutational
distribution but not to the variance. We can see the intuition for this by
considering a degenerate distribution where each mutation has the same effect.
Here, we would still expect variation among individuals due to differences in
the number of mutations each individual receives, even though the variance of
the mutational distribution is zero. The variance among individual trait values
thus has one component due to differences in the number of mutations and an
additional component due to differences in the effects of these mutations. The
first component is proportional to the square of the mean mutational effect,
while the second is proportional to the mutational variance. Therefore, the sum
of the two components is proportional to $m_2$, the mean squared effect.

Since the trait values are normally distributed, any linear combination of
sampled trait values will be as well. This includes the distributions of
observable quantities like the differences in trait values from a reference
individual or from a sample mean. The distribution of trait differences between
individuals is multivariate normal with mean zero and covariance between any
pair of trait differences given by
\begin{equation}
  \label{eq:normcov}
\Cov[Y_i-Y_j,Y_k-Y_l] = \sigma^2\T\left( \E[\mathcal{T}_{i,l}] + \E[\mathcal{T}_{j,k}] -
\E[\mathcal{T}_{i,k}] - \E[\mathcal{T}_{j,l}] \right),
\end{equation}
where $\mathcal{T}_{i,i} = 0$. Classical theory in quantitative genetics uses a
univariate normal distribution of phenotypes in a panmictic population. We can
recover this by considering a population of exchangeable individuals. In this
case $\E[T_{i,j}]$ is the same for all pairs $i \neq j$. Individual trait values
are then conditionally independent given the mean value in the population and
are normally distributed with variance $\E[T_{2,2}]\T\sigma^2$.

The normal model in the infinitesimal limit provides additional theoretical
justification for studies using normal models to look for differences in
selection on quantitative traits between populations
\citep{Ovaskainen2011,Praebel2013,Robinson2015}. Additionally, equation
\eqref{eq:normcov} implies that a covariance matrix based on mean pairwise
coalescent times rather than population split times should be used when modeling
traits as normally distributed in phylogenetics.

%%% Local Variables:
%%% TeX-master: "short_report.tex"
%%% End:
 
