Thus far, the model assumes that loci are unlinked and can experience an
infinite number of mutations. However, a useful simplification is to ignore the
possibility of more than one mutation per locus. This approximation is
reasonable as long as the nucleotide positions affecting the trait are loosely
linked throughout the genome. The low-mutation-rate approximation greatly
simplifies the mgf of the trait distribution such that it is no longer necessary
to know the full form of the mgf of the genealogy:
\begin{equation}
\label{eq:lowmut}
\varphi_{\mathbf{Y}}(\mathbf{k}) \approx \left[ 1 + \sum_{\omega \in \Omega}
  \E[T_\omega] \T \left( \psi\left( \sum_{a \in \omega} k_a\right) -1 \right) +
  O\left( \theta^2 \right))\right]^L.
\end{equation}
Equation \eqref{eq:lowmut} ignores terms that are order two and above in the
mutation rate. Conveniently this equation depends only on the expected length of
each branch, whereas equation \eqref{eq:fullmgf} requires moments of branch
lengths order two and greater. We can use equation \eqref{eq:lowmut} to express
moments of the trait distribution in terms of expected branch lengths calculated
from coalescent models.

%%% Local Variables:
%%% TeX-master: "short_report.tex"
%%% End:
