A useful simplification of the model is to ignore the possibility of more than
one mutation per locus. This approximation is reasonable as long as the
nucleotide positions affecting the trait are loosely linked throughout the
genome. We can greatly simplify the mgf of the trait distribution by grouping
term of order two and above in equation \eqref{eq:fullmgf}:
\begin{equation}
\label{eq:lowmut}
\varphi_{\mathbf{Y}}(\mathbf{k}) \approx \left[ 1 + \T \sum_{\omega \in \Omega}
  \E[T_\omega] \left( \psi\left( \sum_{a \in \omega} k_a\right) -1 \right) +
  O\left( \theta^2 \right))\right]^L.
\end{equation}
Ignoring the $O\left( \theta^2 \right)$ terms corresponds to allowing at most
one mutation per locus. Conveniently, the mgf of the trait values then depends
only on the expected length of each branch, whereas equation \eqref{eq:fullmgf}
requires higher order moments of branch lengths (e.g.
$\E[T_{\omega_1}T_{\omega_2}]$). It is therefore no longer necessary to know the
full form of the mgf of the genealogy, and we can express moments of the trait
distribution using expected branch lengths calculated from coalescent models.

%%% Local Variables:
%%% TeX-master: "short_report.tex"
%%% End:
