This appendix gives the derivation of the low-mutation-rate approximation to the
trait mgf given in equation \eqref{eq:lowmut}. This approximate mgf is derived
starting with equation \eqref{eq:fullmgf} and only allowing for the possibility
of one or zero mutations at each locus. 

Using equation \eqref{eq:fullmgf} and taking a Taylor series approximation to
the exponential function we get
\begin{equation}
  \varphi_{\mathbf{Y}}(\mathbf{k}) =   \int \prod_{\omega \in \Omega}
  \left[1 + \frac{\theta}{2} t_{\omega} \left( \psi\left(\sum_{a \in \omega}k_{a}\right) -1 \right) +
    O\left( \theta^2 \right)\right]
  \Pro(\mathbf{T}=\mathbf{t})\mbox{d}\mathbf{t}.
\end{equation}
The $O\left( \theta^2 \right)$ term captures events where more than one mutation
occurs on the same branch of the genealogy. Events corresponding to more than
one mutation occurring on different branches can also be absorbed into this term
to give
\begin{equation}
  \label{eq:lowmutder}
  \varphi_{\mathbf{Y}}(\mathbf{k}) =
  1 + \T \sum_{\omega \in \Omega}
  \E[T_\omega] \left( \psi\left( \sum_{a \in \omega} k_a\right) -1 \right) +
  O\left( \theta^2 \right)).
\end{equation}
This is the mgf for a trait controlled by only a single locus, and raising it to
the power $L$ to account for multiple independent loci gives the equation
\eqref{eq:lowmut} given in the main text. 

%%% Local Variables:
%%% TeX-master: "quant_gen_manu.tex"
%%% End:
