We can use the low mutation rate approximation to the moment generating function
to calculate moments of the distribution of trait vales. We will start by
calculating the first and second moments. We start, as we did in deriving the
normal distribution, by substituting the Taylor series of the mutational mgf.
\begin{equation}
  \label{eq:mgf_approx_sub}
  \varphi_{\mathbf{Y}}(\mathbf{k}) \approx \left[ 1 + \sum_{\omega \in \Omega}
    \E[T_\omega] \T \left( m_1 \sum_{a \in \omega} k_a +
    \frac{m_2}{2!}\left( \sum_{a \in \omega} k_a\right)^2 +
    \frac{m_3}{3!}\left( \sum_{a \in \omega} k_a\right)^3 +
    \frac{m_4}{4!}\left( \sum_{a \in \omega} k_a\right)^4 \ldots \right) \right]^L
\end{equation}
We can expand this out using multinomial coefficients to get
\begin{align}
  \label{eq:mgf_approx_expand}
  \varphi_{\mathbf{Y}}(\mathbf{k}) &\approx 1 +
  L\T \sum_{\omega \in \mathcal{O}} \E[T_{\omega}]\left( m_1 \sum_{a \in \omega} k_a +
  \frac{m_2}{2}\left( \sum_{a \in \omega} k_a\right)^2 + \ldots \right) \nonumber \\
  &+ \frac{L(L-1)}{2} \left(\T\right)^2 \sum_{\omega \in \Omega} E[T_{\omega}]^2
  \left( m_1 \sum_{a \in \omega} k_a +
  \frac{m_2}{2}\left( \sum_{a \in \omega} k_a\right)^2 + \ldots \right)^2 \nonumber \\
  &+ L(L-1)\left(\T\right)^2\sum_{\omega_1, \omega_2 \in \Omega}\E[T_{\omega_1}]\E[T_{\omega_2}]
  \left( m_1 \sum_{a \in \omega_1} k_a + \ldots \right)
  \left( m_1 \sum_{a \in \omega_2} k_a + \ldots \right) + \ldots.
\end{align}
The first coefficient is $\binom{L}{L-1,1,\mathbf{0}}$, the second is
$\binom{L}{L-2,2,\mathbf{0}}$, and the third is $\binom{L}{L-2,1,1,\mathbf{0}}$.
To calculate the moments of this distribution one takes the partial derivatives
of the mgf and sets the dummy variables to zero.
\begin{equation}
  \label{eq:deriv}
  \E[Y_1^{r_1}\ldots Y_n^{r_n}] = \frac{\partial^{r_1 + \ldots + r_n}}{\partial k_1^{r_1} \ldots \partial k_n^{r_n}}
  \varphi_{\mathbf{Y}}(\mathbf{k})\Bigr|_{\mathbf{k}=0}
\end{equation}
Using this to calculate the first moment of the trait distribution we get
\begin{equation}
  \label{eq:mom1}
  \E[Y_a] \approx L\T m_1 \sum_{\omega \in \Omega_a} \E[T_\omega].
\end{equation}
The second moment is more complicated because there are $k_a^2$ terms in all
three lines of equation \ref{eq:mgf_approx_expand}.
\begin{align}
  \E[Y_a^2] &\approx L\T m_2 \sum_{\omega \in \Omega_a} E[t_\omega] \nonumber \\
  &+ \frac{L(L-1)}{2} \left(\T \right)^2 m_1^2 \sum_{\omega \in \Omega_a} 2 \E[T_\omega]^2 \nonumber \\
  &+ L(L-1) \left(\T \right)^2 m_1^2 \sum_{\omega_1 , \omega_2 \in \Omega_{a+b}} 2 \E[T_{\omega_1}]E[T_{\omega_2}]
\end{align}
Terms with $(\T)^2$ are kept because they also include a second order term of
$L$ in front of them. We can now calculated the variance using $\Var[Y]=\E[Y^2] -
\E[Y]^2$. The squared first moment can be written as
\begin{align}
  \left(L\T m_1 \sum_{\omega \in \Omega_a} \E[t_\omega] \right)^2 &=
  L^2\left(\T\right)^2 m_1^2 \sum_{\omega \in \Omega_a} \E[T_\omega]^2 \nonumber \\
  &+ L^2\left(\T\right)^2 m_1^2 \sum_{\omega_1 , \omega_2 \in \Omega_{a+b}} \E[T_{\omega_1}]\E[T_{\omega_2}].
\end{align}
Subtracting this from the second moment gives
\begin{align}
  \label{eq:var}
  \Var[Y_a] &\approx L\T m_2 \sum_{\omega \in \Omega_a} \E[T_\omega] \nonumber \\
  &- L \left(\T\right)^2 m_1^2 \sum_{\omega \in \Omega_a}\E[T_\omega]^2 \nonumber \\
  &-  2L \left(\T\right)^2 m_1^2 \sum_{\omega_1 , \omega_2 \in \Omega_{a+b}} \E[T_{\omega_1}]\E[T_{\omega_2}] \nonumber \\
  &= L\T m_2 \sum_{\omega \in \Omega_a} \E[T_\omega] -
  L\left( \T m_1 \sum_{\omega \in \Omega_a} \E[T_\omega] \right)^2 \nonumber \\
  &= L\T m_2 \E[T_{MRCA}] - L\left( \T m_1 \E[T_{MRCA}] \right)^2  \\
  &\approx L\T m_2 \E[T_{MRCA}].  \nonumber
\end{align}

Due to the large number of terms we only derive the fourth moment of the trait
value distribution for the case when the mean mutational effect is zero. The
terms of \eqref{eq:mgf_approx_sub} appearing in the fourth moment after we apply
\eqref{eq:deriv} are
\begin{equation*}
  L \left(\T\right) \frac{m_4}{24}
  \sum_{\omega \in \Omega_a} \E[T_\omega] \left(\sum_{a \in \omega} k_a\right)^4
\end{equation*}
for the fourth moment along one branch,
\begin{equation*}
  \binom{L}{L-2,2,\mathbf{0}}\left(\T\right)^2\left(\frac{m_2}{2}\right)^2
  24 \sum_{\omega \in \Omega_a} \E[T_\omega]^2 \left(\sum_{a \in \omega} k_a\right)^4
\end{equation*}
for the second moment of the same branch chosen twice, and
\begin{equation*}
  \binom{L}{L-2,1,1,\mathbf{0}}\left(\T\right)^2\left(\frac{m_2}{2}\right)^2
  24 \sum_{\omega_1 , \omega_2 \in \Omega_{a+b}} \E[T_{\omega_1}]\E[T_{\omega_2}]
  \left(\sum_{a \in \omega_1} k_a\right)^2\left(\sum_{a \in \omega_2} k_a\right)^2
\end{equation*}
for the second moments on two different branches. Taking the fourth derivatives
of these in terms of the desired branch we get
\begin{align}
  \label{eq:mom4}
  \E[Y_a^4] &= L\T m_4 \E[T_{MRCA}] \nonumber \\
  &+ \frac{L(L-1)}{2} \left(\T\right)^2\left( \frac{m_2}{2} \right)^2
  24 \sum_{\omega \in \Omega_a} \E[t_\omega]^2 \nonumber \\
  &+ L(L-1) \left(\T\right)^2\left( \frac{m_2}{2} \right)^2
  24 \sum_{\omega_1 , \omega_2 \in \Omega_{a+b}} \E[T_{\omega_1}]\E[T_{\omega_2}] \nonumber \\
  &= L\T m_4 \E[T_{MRCA}] +
  3L(L-1)\left( \T m_2 \sum_{\omega \in \Omega_a} \E[T_\omega] \right)^2x \nonumber \\
  &= L\T m_4 \E[T_{MRCA}] +
  3L(L-1)\left( \T m_2 \E[T_{MRCA}] \right)^2 \\
  &\approx L\T m_4 \E[T_{MRCA}] +
  3\left(L \T m_2 \E[T_{MRCA}] \right)^2. \nonumber 
\end{align}

The variance and the fourth moment derived in equation \eqref{eq:var} and
\eqref{eq:mom4} can be used to derive the kurtosis of $Y_a$. The kurtosis of a
random variable $X$ is defined as
\begin{equation*}
  \Kurt[X]=\frac{\E[(X-\E[X])^4]}{\E[(X-\E[X])]^2]^2}.
\end{equation*}
This is the fourth central moment divided by the variance. It is possible to
calculate the kurtosis of a single trait value over evolutionary realization.
For ease of calculation, we'll examine this in the case where the mean mutation
effect (and therefore trait value) is zero. If we plug \eqref{eq:var} and
\eqref{eq:mom4} into the expression for the kurtosis we get
\begin{align*}
  \Kurt[Y_a] &= \frac{L\T m_4 \E[T_{MRCA}]}{\left(L\T m_2 \E[T_{MRCA}]\right)^2} +
  \frac{3L(L-1)\left( \T m_2  \E[T_{MRCA}]\right)^2}{\left(L\T m_2 \E[T_{MRCA}]\right)^2} \nonumber \\
  &= \frac{m_4}{L\T m_2^2\E[T_{MRCA}]} + \frac{3(L^2-L)}{L^2} \nonumber \\
  &= \frac{\kappa}{L\T \E[T_{MRCA}]} + 3\left( 1 - \frac{1}{L} \right).
\end{align*}

We also calculate some additional moments that have less clear interpretations
but which are useful when calculating the expected fourth central moment in the
population. The first of these is $\E[Y_a^3Y_b]$. The terms of
\eqref{eq:mgf_approx_sub} appearing in this are
\begin{equation*}
  L \left(\T\right) \frac{m_4}{24} 4 k_a^3k_b \sum_{\omega \in \Omega_{a+b}} \E[T_\omega]
\end{equation*}
and
\begin{equation*}
  L(L-1)\left(\T \frac{m_2}{2}\right)^2 k_a^2 \times 2k_ak_b
  \left( \sum_{\omega \in \Omega_{a+b}} \E[T_\omega] \right) \left( \sum_{\omega \in \Omega_a} \E[T_\omega] \right).
\end{equation*}
This ultimately gives
\begin{equation}
  \label{eq:m31}
  E[Y_a^3Y_b] = L \T m_4 \E[\tau_{a+b}] + 3L(L-1) \left(\T m_2\right)^2 \E[T_{MRCA}]\E[\tau_{a+b}].
\end{equation}
The next fourth moment of interest is $\E[Y_a^2Y_bY_c]$. The terms of
\eqref{eq:mgf_approx_sub} are
\begin{equation*}
  L \T \frac{m_4}{24} 12k_a^2k_bk_c \sum_{\omega \in \Omega_{a+b+c}} \E[T_\omega],
\end{equation*}
\begin{equation*}
  L(L-1) \left(\T \frac{m_2}{2}\right)^2 k_a^2 \times 2k_bk_c
  \left( \sum_{\omega \in \Omega_a} \E[T_\omega] \right)\left( \sum_{\omega \in \Omega_{a+b}} \E[T_\omega] \right),
\end{equation*}
and 
\begin{equation*}
  L(L-1) \left(\T \frac{m_2}{2}\right)^2 2k_ak_b \times 2k_ak_c \left( \sum_{\omega \in \Omega_{a+b}} \E[T_\omega] \right)
  \left( \sum_{\omega \in \omega_{b+c}} \E[T_\omega] \right).
\end{equation*}
Taking the appropriate derivatives of these gives
\begin{equation}
  \label{eq:m211}
  L \T m_4 \E[\tau_{a+b+c}] + L(L-1)\left(\T m_2\right)^2\E[T_{MRCA}]\E[\tau_{b+c}] +
  2L(L-1) \left(\T m_2\right)^2\E[\tau_{a+b}]\E[\tau_{a+c}].
\end{equation}
Individuals in the population are exchangeable as long as it is not structured.
The pairwise expected shared branch lengths are in that case all equal and we
can write \eqref{eq:m211} as
\begin{equation}
  \label{eq:m211s}
  L \T m_4 \E[\tau_{a+b+c}] + L(L-1)\left(\T m_2\right)^2\E[T_{MRCA}]\E[\tau_{a+b}] +
  2L(L-1) \left(\T m_2\right)^2\E[\tau_{a+b}]^2.
\end{equation}
The final moment we'll look at is $\E[Y_aY_bY_cY_d]$ which has relevant terms
%% The set of branches containing all individuals
\begin{equation*}
  L \T \frac{m_4}{24} 24k_ak_bk_ck_d \sum_{\omega \in \Omega_{a+b+c+d}} \E[T_\omega],
\end{equation*}
%% Cross set of a-b branches and c-d branches
\begin{equation*}
  L(L-1) \left(\T \frac{m_2}{2}\right)^2 2k_ak_b \times 2k_ck_d \left( \sum_{\omega \in \Omega_{a+b}} \E[T_\omega] \right)
  \left( \sum_{\omega \in \Omega_{c+d}} \E[T_\omega] \right),
\end{equation*}
%% Cross set of a-c branches and b-d branches
\begin{equation*}
  L(L-1) \left(\T \frac{m_2}{2}\right)^2 2k_ak_c \times 2k_bk_d \left( \sum_{\omega \in \Omega_{a+c}} \E[T_\omega] \right)
  \left( \sum_{\omega \in \Omega_{b+d}} \E[T_\omega] \right),
\end{equation*}
and
%% cross set of a-d branches and b-c branches
\begin{equation*}
  L(L-1) \left(\T \frac{m_2}{2}\right)^2 2k_ak_d \times 2k_bk_c \left( \sum_{\omega \in \Omega_{a+d}} \E[T_\omega] \right)
  \left( \sum_{\omega \in \omega_{b+c}} \E[T_\omega] \right).
\end{equation*}
When the appropriate fourth order partial derivatives are taken we get
\begin{align*}
  L \T m_4 \E[\tau_{a+b+c+d}] \\
  + L(L-1)\left(\T m_2\right)^2\E[\tau_{a+b}]\E[\tau_{c+d}] \\
  + L(L-1)\left(\T m_2\right)^2\E[\tau_{a+c}]\E[\tau_{b+d}] \\
  + L(L-1)\left(\T m_2\right)^2\E[\tau_{a+d}]\E[\tau_{b+c}].
\end{align*}
We can again simplify this expression for populations if we assume that
individual are exchangeable. This gives
\begin{equation}
  \label{eq:m1111}
  L \T m_4 \E[\tau_{a+b+c+d}] + 3L(L-1)\left(\T m_2\right)^2\E[\tau_{a+b}]^2.
\end{equation}

The expected kurtosis in the population is a quotient of random variables and
calculating a second order approximation requires calculating eight order
moments of the trait distribution. Instead we will calculate the expected fourth
central moment.
\begin{equation}
  \E[M_{4,Y}] = \E\left[\frac{1}{N} \sum \left( Y_i - \frac{\sum Y_j}{N} \right)^4 \right].
\end{equation}
Examining the sum inside the expectation we see that 
\begin{align}
  \label{eq:expandkurt}
  \E\left[ \left( Y_i - \frac{\sum Y_j}{N} \right)^4 \right] &= \E[Y_i^4] - 
                                                              4\E\left[ Y_i^3 \frac{\sum Y_j}{N} \right] + 
                                                              6\E\left[Y_i^2\left(\frac{\sum Y_j}{N}\right)^2\right] - 
                                                              4\E\left[Y_i\left(\frac{\sum Y_j}{N}\right)^3\right]+ 
                                                              \left(\frac{\sum Y_j}{N}\right)^4 \nonumber \\
                                                            &= \E[Y_i^4] - 
                                                              \frac{4}{N}\sum_j \E[Y_i^3Y_j] + 
                                                              \frac{6}{N^2}\sum_{j,k} \E[Y_i^2Y_jY_k] \nonumber\\
                                                              &-\frac{4}{N^3}\sum_{j,k,l}\E[Y_iY_jY_kY_l] + 
                                                              \frac{1}{n^4}\sum_{j,k,l,d}\E[Y_jY_kY_lY_d].
\end{align}
In calculating these expectations we have to remember that the value depends
only on the number of times each variable appears in the expectation. That is,
$\E[Y_1^2Y_2Y_3]$ is equivalent to $\E[Y_1Y_2^3Y_3]$ as long as all individuals
in the population are exchangeable. The resulting expansion of
\eqref{eq:expandkurt} can be simplified by only considering terms of order one.
Other terms can be ignored since we are assuming there are a large number of
individuals in the population. This yields
\begin{align}
  \label{eq:popkurt}
  \E\left[ \left( Y_i - \frac{\sum Y_j}{N} \right)^4 \right] &=
  \E[Y_i^4]  - \frac{4(n-1)}{n}\E[Y_i^3Y_j] + \frac{6(n-1)(n-2)}{n^2}\E[Y_i^3Y_jY_k]  \nonumber \\
  &- \frac{4(n-1)(n-2)(n-3)}{n^3}\E[Y_iY_jY_kY_l] \nonumber \\
  &+ \frac{(n-1)(n-2)(n-3)(n-4)}{n^4}\E[Y_jY_kY_lY_d]
  + O(n^{-1}) \nonumber \\
  &\approx \E[Y_i^4]  - 4\E[Y_i^3Y_j] + 6\E[Y_i^2Y_jY_k] - 3\E[Y_iY_jY_kY_l].
\end{align}

The first term, $\E[Y_i^4]$ was derived in equation \eqref{eq:mom4} as
\begin{equation*}
  L\T m_4 \E[T_{MRCA}] + 3L(L-1)\left( \T m_2 \E[T_{MRCA}] \right)^2.
\end{equation*}
The second term, $\E[Y_i^3Y_j]$ was derived in equation \eqref{eq:m31} as
\begin{equation*}
  L \T m_4 \E[\tau_{a+b}] + 3L(L-1) \left(\T m_2\right)^2 \E[T_{MRCA}]\\E[\tau_{a+b}].
\end{equation*}
The third term, $\E[Y_i^2Y_jY_k]$ was derived in equation \eqref{eq:m211} as
\begin{equation*}
  L \T m_4 \E[\tau_{a+b+c}] + L(L-1)\left(\T m_2\right)^2\E[T_{MRCA}]\E[\tau_{a+b}] +
  2L(L-1) \left(\T m_2\right)^2\E[\tau_{a+b}]^2.
\end{equation*}
The fourth term, $\E[Y_iY_jY_kY_l]$ was derived in equation \eqref{eq:m1111} as
\begin{equation*}
  L \T m_4 \E[\tau_{a+b+c+d}] + 3L(L-1)\left(\T m_2\right)^2\E[\tau_{a+b}]^2.
\end{equation*}
Plugging these into \eqref{eq:popkurt} we get
\begin{align}
  \label{eq:popm4coal}
  \E[M_4] &\approx L \T m_4 \left( \E[T_{MRCA}] - 4\E[\tau_{a+b}] + 6\E[\tau_{a+b+c}] -3\E[\tau_{a+b+c+d}] \right)\nonumber \\
  &+ 3\left( L \T m_2 \right)^2\left(\E[T_{MRCA}]- \E[\tau_{a+b}]\right)^2.
\end{align}
With some simple manipulation this can be rewritten in terms of
$\mathbbm{T}_{i,j}$ to give
\begin{align}
  \E[M_4] &= 3\left(L \T \E[\mathbbm{T}_{2,2}] m_2\right)^2 \nonumber \\
  &+ L m_4 \T ( \E[\mathbbm{T}_{4,4}] + \frac{1}{3} \E[\mathbbm{T}_{3,4}] +
    \frac{2}{9} \E[\mathbbm{T}_{2,4}] ).
\end{align}
%%% Local Variables:
%%% TeX-master: "short_report.tex"
%%% End:
