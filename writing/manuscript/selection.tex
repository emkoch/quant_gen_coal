A situation in which higher order moments of the trait distribution can be
relevant is in the response of the population to selection. Evolutionary
quantitative genetics often assumes the distribution of additive genetic values
in the population remains normally distributed as selection alters the mean and
variance. \citet{Turelli1990} used a multilocus population genetic model to show
how departures from normality affect the response to selection. In their
analysis, departures from normality are due to the build up of linkage
disequilibrium. However, their results are valid regardless of how departures
from normality arise. The \citet{Turelli1990} theory can be used to analyze the
response to selection in the toy situation of a trait that has evolved neutrally
up to the current time and is then subjected to one generation of selection
under a particular fitness function.

According to \citet{Turelli1990}, in the absence of environmental effects the
response of the mean phenotype in the population is
\begin{equation}
  \label{eq:selresp}
  \Delta \bar{Y} = M_2L_1 + M_3L_2 + \gamma_4M^2_2L_3 +
  \left( M_5-4M_3M_2\right)L_4 + \ldots.
\end{equation}
$\gamma_4$ is the excess kurtosis of the trait values in the population above a
normal distribution and the $M_i$ terms are again the $i^{th}$ central moments
of the trait value distribution in the population. The $L_i$ are selection
gradients in terms of the moments of the genetic component of the trait value
distribution and describe the shape of the fitness function on the trait values.
In the absence of environmental effects, the $L_i$ are selection gradients in
terms of the moments of the trait value distribution that we have so far
considered in this study.

Equation \eqref{eq:selresp} shows that whether higher order moments of the trait
distribution contribute to the selection response depends on the shape of the
fitness function through the $L_i$. The excess kurtosis affects the response to
selection linearly with $L_3$, which depends on the third moment of the trait
value distribution.

When selection is cubic ($W(Y) = b_0 + b_3(Y-\bar{Y})^3$), the response to
selection is
\begin{equation}
  \label{eq:cubresp}
  \Delta \bar{Z} = \frac{M_4\beta}{1 + M_3\beta},
\end{equation}
where $\beta=b_3/b_0$. Cubic selection represents an idealized fitness function
to investigate the effects of selection acting on the tails of the population
trait distribution. Equation \eqref{eq:cubresp} is a ratio of random quantities,
so calculating the expected response to selection is not feasible. However, we
conjecture that demographic and mutational process increasing the expected
fourth central moment relative to the third would increase the response to
selection. For instance, a high mutational kurtosis and a low skew would likely
increase the response to cubic selection.

%%% Local Variables:
%%% TeX-master: "short_report.tex"
%%% End:
