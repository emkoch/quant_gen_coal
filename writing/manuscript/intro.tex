Similarly to neutral models in genetics, neutral models of quantitative traits
provide a null distribution against which goodness-of-fit tests can be used to
test for the action of natural selection\citep{Lande1976}, and clarify the
effects of neutral forces on variation, such as demography and mutation
\citet{Lynch1986}. The common approach is to first model phenotypes as normally
distributed, either among offspring within a family, among members of a
population, or between species \citep{Turelli2017}. Indeed, it has been
suggested that the normality assumption is the defining characteristic of
quantitative genetics \citep{Rice2004}. This might be justified if phenotypes
are influenced by a large number of sufficiently independent Mendelian factors
\citep{Fisher1918}, or normality may simply appear approximately true in
practice.

Neutral models assuming normality have been used in numerous contexts.
\citet{Freckleton2002} and others used Brownian motion to detect and correct for
phylogenetic dependence in studies of phenotypic evolution. Normal models of
phenotypic evolution incorporate genetic drift by including factors like
population size and subdivision \citep{Chakraborty1982,Lynch1986,Lande1992}, and
the dynamics of phenotypic evolution are examined forwards in time as a balance
between mutation creating variance, migration spreading variance among
subpopulations, and fixation removing it. This allows one to derive the
equilibrium genetic variance and the rate at which equilibrium is approached.
Multivariate normality of traits also underlies many tests for spatially varying
selection such as that from \citet{Ovaskainen2011}.

There is nothing remarkable about modeling phenotypes according to a normal
distribution. The broad applicability of quantitative genetics stems from the
fact that traits can be studied without concern for the number of causal loci
influencing a trait, the genealogies at these sites, or the distribution of
mutational effects. However, heritable phenotypic variation is ultimately due to
discrete mutations at discrete locations in the genome, and how the phenotypic
variance due to these mutations is distributed depends on the genealogies at
these loci. When the number of mutations affecting a trait is large the central
limit theorem ensures that the distributions of genealogies mutational effects
can be ignored, but a full model of phenotypic variation would have to include
them. Importantly, deviations from normality may affect the outcomes of
goodness-of-fit tests that necessarily aim to identify outliers from a normal
model.

A more complete model of neutral phenotypic variation can begin by modeling the
genealogies at causal loci. The principle modeling framework for genealogical
variation is the coalescent process \citep{Wakeley2008}, but few studies have
connected the coalescent to quantitative genetics. \citet{Whitlock1999} argued
that under the coalescent measures of trait ($Q_{ST}$) and genetic ($F_{ST}$)
differentiation have the same expected value given general models of population
subdivision. By simulating from the coalescent with
recombination, \citet{Griswold2007} investigated the effects of shared ancestry
and linkage disequilibrium on the matrix of genetic variances and covariances
between traits ($\mbox{\textbf{G}}$). They found that linkage disequilibrium and
small numbers of causal loci can cause phenotypic covariances not predicted by
the mutational covariance matrix. Although not explicitly connected to the
coalescent, \citet{Ovaskainen2011} developed their test for spatially varying
selection by assuming covariance in trait values, conditional on the G-matrix in
the ancestral population, depends only on the pairwise coancestry coefficients,
which have a clear interpretation in terms of the coalescent
process \citep{Slatkin1991}.

\citet{Khaitovich2005} modeled the evolution of gene expression values on
phylogenetic trees assuming a single non-recombining causal locus but allowed
for an arbitrary distribution of mutational effects. More
recently, \citet{Schraiber2015} developed a similar general model of
quantitative trait evolution at the population level based on the coalescent and
allowing for any number of causal loci. They derived the characteristic function
for the distribution of phenotypic values in a sample and showed how such values
can deviate strongly from normality when the number of loci is small or the
mutational distribution has fat tails.

\citet{Schraiber2015} derived their results for a panmictic, constant-size
population. Natural populations do not tend to have stable population sizes and
show considerable spatial structure, and it is unclear how these violations of
the constant-size, panmictic model might influence deviation from normality. We
take advantage of the ability of coalescent theory to handle nonequilibrium
demographies, and to relax the constant-size assumption. Additionally, the
analysis of quantitative traits in structured populations provides an
opportunity to infer the incidence of local adaptation or stabilizing selection.
The $Q_{ST}$/$F_{ST}$ paradigm was developed to this
end \citep{Whitlock2008,Spitze1993}. $Q_{ST}$, defined as the ratio of the trait
variance between subpopulations to the total trait variance, is compared to
$F_{ST}$, which measures the same property for genetic variation and is
calculated using neutral markers to provide a null distribution. If observed
$Q_{ST}$ sufficiently from the null expectation it is concluded that natural
selection has acted. \citet{Ovaskainen2011} developed a modern extension of the
$Q_{ST}$/$F_{ST}$ paradigm for genetic values measured in breeding experiments
and \citet{Berg2014} developed an extension for genetic values computed from
GWAS summary statistics. Understanding the neutral distribution of trait values
at the sample and population level is necessary for the development of
goodness-of-fit tests suited to populations with complicated histories and
traits with sparse genetic architectures.

Given the importance of nonequilibrium demography and spatial structure, we
generalize the work of \citet{Schraiber2015} to include these scenarios. This
generalization is accomplished by deriving the form of the moment generating
function (mgf) for an arbitrary distribution of coalescent times. The key result
of \citet{Schraiber2015}, the characteristic function of the sampling distribution
of phenotypic values, is a special case of this general generating function. We
then show how a normal models arises by taking the infinitesimal limit where the
effect size per mutation becomes small as the number of loci potentially
affecting the trait becomes large. We then calculate the third and fourth
central moments of the trait distribution in panmictic populations to
illustrate how departures from normality depend both on genetic parameters and
genealogical distributions. Finally, we discuss the consequences of these
results for $Q_{ST}$ tests, the response to selection, and the inference of
genetic architecture.

An improved null distribution for $Q_{ST}$ tests can be derived simply by using
the normal distribution that arises in the infinitesimal limit of our coalescent
model. When selection acts primarily on the tails of the trait distribution, the
single generation response depends on trait moments that are sensitive to
genetic architecture and demographic. Additionally, we show how it may be
possible to infer features of the mutational distribution when both trait values
and sequence data are available.

%%% Local Variables:
%%% TeX-master: "short_report.tex"
%%% End:
