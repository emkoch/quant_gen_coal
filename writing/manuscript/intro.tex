Neutral models of quantitative traits provide a null distribution against which
goodness-of-fit tests can be used to test for natural
selection \citep{Lande1976,Leinonen2013}. Neutral models also clarify the
effects of purely neutral forces such as genetic drift and mutation on trait
distributions \citep{Lynch1986}. Common approaches model phenotypes as normally
distributed among offspring within a family, among members of a population, or
among species \citep{Turelli2017}. Indeed, it has been suggested that the
normality assumption is the defining characteristic of quantitative
genetics \citep{Rice2004}. This is approximately true if phenotypes are
influenced by a large number of sufficiently independent Mendelian factors
\citep{Fisher1918} and selection is weak \citep{Turelli1990}.

The goal of neutral models for quantitative traits is to ask to whether
phenotypic differentiation between groups can be reasonably explained by
processes other than natural selection. On macroevolutionary time scales, models
stemming from \citet{Lande1976} have used Brownian motion to describe the
evolution of the mean value of a quantitative trait in a population. These
models are used in tests for extreme trait divergence between
species \citep{Turelli1988}, test for phylogenetic signal in trait
distributions \citep{Freckleton2002}, and correct for phylogenetic dependence
when calculating correlations between traits \citep{Felsenstein1985}. On shorter
time scales, multivariate normality of neutral trait values also underlie tests
for spatially varying selection in structured populations such as the method
developed by \citet{Ovaskainen2011}. Other neutral models for quantitative
traits have not assumed normality \citep{Chakraborty1982,Lynch1986,Lande1992},
and the dynamics of phenotypic evolution are examined forwards in time as a
balance between mutation creating variance, migration spreading variance among
subpopulations, and fixation removing it. However, these studies were limited to
simple models of population structure and history. Backwards-in-time, coalescent
models would allow for more general demographic scenarios.

Under normality, trait dynamics are modeled without concern for the number of
causal loci, genealogies at these loci, or the distribution of mutational
effects (the distribution that mutations at causal loci draw their effects
from). However, heritable phenotypic variation is ultimately due to discrete
mutations at discrete loci, and how the variation arising from these mutations
is distributed depends on individual genealogies. The distribution of
genealogies in the genome might be influenced by recent population growth and
the distribution of mutational effects could be skewed for due to the details of
a particular developmental pathway. When a large number of mutations affect a
trait, the central limit theorem ensures that the distributions of genealogies
and mutational effects can be ignored, but a full model of phenotypic variation
would have to include them. Importantly, deviations from normality may affect
the outcomes of goodness-of-fit tests that necessarily aim to identify outliers
from a normal model.

A more expansive neutral model of phenotypic variation would consider
genealogies at causal loci. The principle modeling framework for genealogical
variation is the coalescent \citep{Wakeley2008}, but few studies have connected
the coalescent to quantitative genetics. \citet{Whitlock1999} used coalescent
theory to argue that measures of phenotypic ($Q_{ST}$) and genetic ($F_{ST}$)
differentiation have the same expected value in general models of population
subdivision. Using coalescent simulations, \citet{Griswold2007} investigated the
effects of shared ancestry and linkage disequilibrium on the genetic covariance
matrix for a set of traits ($\mbox{\textbf{G}}$ matrix). They found that linkage
disequilibrium and small numbers of causal loci can cause phenotypic covariance
not predicted by mutational covariance. \citet{Mendes2018} also used a
coalescent-based model to show that using a species tree based on population
split times increases false positive rates in phylogenetic comparative methods,
among other problems. Although not explicitly connected, \citet{Ovaskainen2011}
developed a test for spatially varying selection that assumes the genetic
covariance between individuals depends only on pairwise coancestry coefficients,
which have a clear interpretation under the coalescent \citep{Slatkin1991}.

Two studies have asked how the shape of the distribution of mutational effect
sizes, beyond just the variance, impacts trait
distributions. \citet{Khaitovich2005} modeled the evolution of gene expression
on phylogenetic trees assuming a single non-recombining causal locus but allowed
for an arbitrary distribution of mutational effects. This model detected
deviations from normality consistent with asymmetries in the mutational
distribution of gene expression in great apes. More
recently, \citet{Schraiber2015} developed a similar model for trait evolution
within populations based on the coalescent and allowing for any number of causal
loci. They derived the characteristic function for the distribution of
phenotypes in a sample and showed how they can deviate from normality when the
number of loci is small or the mutational distribution has heavy tails.

\citet{Schraiber2015} derived their results for a panmictic, constant-size
population. Natural populations rarely have stable population sizes and show
considerable spatial structure. It is unclear how these violations of the
constant-size, panmictic model influence deviations from normality. We take
advantage of the ability of coalescent theory to handle nonequilibrium
demographies and population structure to relax these modeling assumptions.

Extending coalescent models of quantitative traits to structured populations is
important because the analysis of structured populations provides an opportunity
to infer the incidence of local adaptation or stabilizing selection. In
structured populations, the neutral divergence in suitably normalized trait
values among subpopulations has approximately the same expectation and variance
as the neutral divergence in allele frequency at a single
site \citep{Rogers1983,Whitlock2008}. The $Q_{ST}$/$F_{ST}$ paradigm was
developed to test whether trait divergence in structured populations could be
explained by genetic drift alone \citep{Spitze1993,Whitlock2008,Leinonen2013}.
$Q_{ST}$, defined as the ratio of the trait variance between subpopulations to
the total trait variance, is compared to $F_{ST}$, which measures the same
property for genetic variation at neutral markers. It is concluded that natural
selection has acted if the observed $Q_{ST}$ is sufficiently far from the
neutral expectation. \citet{Ovaskainen2011} developed a modern extension of the
$Q_{ST}$/$F_{ST}$ paradigm for genetic values measured in breeding experiments
and \citet{Berg2014} also extended the paradigm to make use of genetic values
computed from GWAS summary statistics. Understanding the neutral distribution of
trait values is necessary for the development of goodness-of-fit tests that are
applicable to populations with complicated histories and traits with sparse
genetic architectures.

We generalize the work of \citet{Schraiber2015} by deriving the form of the
moment generating function (mgf) for arbitrary distributions of coalescent times
and therefore arbitrary population histories. The key result
of \citet{Schraiber2015}, the characteristic function of the distribution of
trait values, is a special case of this general result. We then show how a
normal model arises by taking an infinitesimal limit where the effect size per
mutation becomes small as the number of loci potentially affecting the trait
becomes large. We then calculate the third and fourth central moments of the
trait distribution in panmictic populations to illustrate how departures from
normality depend both on genetic parameters and genealogical distributions. For
instance, in exchangeable populations the expected third central moment is
proportional to the third noncentral moment of the mutational distribution times
the expected time to the first coalescent event in a sample of size three.

Finally, we discuss the consequences of these results for $Q_{ST}$ tests and the
inference of genetic parameters. We find that using the normal distribution that
arises in the infinitesimal limit gives an improved null distribution of
$Q_{ST}$. Additionally, we show that it is likely not possible to infer useful
features of the mutational distribution using only sampled trait values. Future
work will be necessary to develop tests for selection that take into account
both demography and genetic parameters, but the model developed here provides
the groundwork for such an undertaking.

%%% Local Variables:
%%% TeX-master: "quant_gen_manu.tex"
%%% End:
