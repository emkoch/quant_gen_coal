Similarly to neutral models of genetic variation, neutral models of quantitative
traits provide a null distribution against which goodness-of-fit tests can be
used to test for the action of natural selection \citep{Lande1976}. Neutral
models can also clarify the effects of neutral forces such as population size
and mutation rate have on variation \citep{Lynch1986}. The common approach is to
first model phenotypes as normally distributed, either among offspring within a
family, among members of a population, or between species \citep{Turelli2017}.
Indeed, it has been suggested that the normality assumption is the defining
characteristic of quantitative genetics \citep{Rice2004}. This might be
justified if phenotypes are influenced by a large number of sufficiently
independent Mendelian factors \citep{Fisher1918}, or normality may simply appear
approximately true in practice.

Neutral models for quantitative traits have been developed in a variety of
contexts. The goal of these models is ask whether phenotypic differentiation
between groups can be reasonably explained by processes other than natural
selection. On macroevolutionary time scales, models stemming
from \citet{Lande1976} have used Brownian motion to model the change in the mean
value of quantitative traits. These models are used in statistical methods to
test for extreme trait divergence between species \citep{Turelli1988}, test for
phylogenetic signal in trait distributions \citep{Freckleton2002}, and correct
for phylogenetic dependence when calculating correlations between trait
values \citep{Felsenstein1985}. On shorter time scales, neutral distributions
assuming multivariate normality of trait values also underlie tests for
spatially varying selection in structured populations such as that
from \citet{Ovaskainen2011}. Other neutral models for quantitative traits have
not assumed normality \citep{Chakraborty1982,Lynch1986,Lande1992}, and the
dynamics of phenotypic evolution are examined forwards in time as a balance
between mutation creating variance, migration spreading variance among
subpopulations, and fixation removing it. However, these models can still be
used to parameterize normal distributions by calculating the first two moments
of the distribution of trait values. 

%% Models of phenotypic evolution incorporate genetic drift by including factors
%% like population size and subdivision
%% \citep{Lande1976,Chakraborty1982,Lynch1986}. The dynamics of phenotypic
%% evolution are examined forwards in time as a balance between mutation creating
%% variance, migration spreading variance among subpopulations, and fixation
%% removing it \citep{Lande1992}. Tests based on thse models have been developed to
%% test whether the degree of phenotypic differentiation can be explained by
%% neutral processes alone \citep{Turelli1988}. On macroevolutionary time scales
%% \citet{Freckleton2002} and others have used Brownian motion to model the change
%% in the mean value of quantitative traits within species with the aim of
%% detecting and correcting for phylogenetic dependence. Neutral distributions
%% assuming multivariate normality of trait values also underly tests for spatially
%% varying selection such as that from \citet{Ovaskainen2011}.

The broad applicability of normal models in quantitative genetics stems from the
fact that traits can be studied without concern for the number of causal loci
influencing a trait, the genealogies at these sites, or the distribution of
mutational effects. However, heritable phenotypic variation is ultimately due to
discrete mutations at discrete locations in the genome, and how the phenotypic
variance due to these mutations is distributed depends on the genealogies at
these loci. For instance, the distribution of genealogies in the genome might be
strongly influenced by recent population growth while the distribution of
mutational effects could be skewed for biological reasons. When the number of
mutations affecting a trait is large the central limit theorem ensures that the
distributions of genealogies and mutational effects can be ignored, but a full
model of phenotypic variation would have to include them. Importantly,
deviations from normality may affect the outcomes of goodness-of-fit tests that
necessarily aim to identify outliers from a normal model.

A more complete model of neutral phenotypic variation can begin by modeling the
genealogies at causal loci. The principle modeling framework for genealogical
variation is the coalescent process \citep{Wakeley2008}, but few studies have
connected the coalescent to quantitative genetics. \citet{Whitlock1999} argued
that under the coalescent, measures of trait ($Q_{ST}$) and genetic ($F_{ST}$)
differentiation have the same expected value given general models of population
subdivision. By simulating from the coalescent with recombination,
\citet{Griswold2007} investigated the effects of shared ancestry and linkage
disequilibrium on the matrix of genetic variances and covariances between traits
($\mbox{\textbf{G}}$). They found that linkage disequilibrium and small numbers
of causal loci can cause phenotypic covariances not predicted by the mutational
covariance matrix. Although not explicitly connected to the coalescent,
\citet{Ovaskainen2011} developed their test for spatially varying selection by
assuming that the covariance in trait values, conditional on the G-matrix in the
ancestral population, depends only on the pairwise coancestry coefficients,
which have a clear interpretation in terms of the coalescent process
\citep{Slatkin1991}.

\citet{Khaitovich2005} modeled the evolution of gene expression values on
phylogenetic trees assuming a single non-recombining causal locus but allowed
for an arbitrary distribution of mutational effects. More
recently, \citet{Schraiber2015} developed a similar general model of
quantitative trait evolution at the population level based on the coalescent and
allowing for any number of causal loci. They derived the characteristic function
for the distribution of phenotypic values in a sample and showed how such values
can deviate strongly from normality when the number of loci is small or the
mutational distribution has fat tails.

\citet{Schraiber2015} derived their results for a panmictic, constant-size
population. Natural populations do not tend to have stable population sizes and
show considerable spatial structure, and it is unclear how these violations of
the constant-size, panmictic model might influence deviations from normality. We
take advantage of the ability of coalescent theory to handle nonequilibrium
demographies and population structure to relax these modeling assumptions.

Extending coalescent models of quantitative traits to structured populations is
important because the analysis of structured populations provides an opportunity
to infer the incidence of local adaptation or stabilizing selection. The
$Q_{ST}$/$F_{ST}$ paradigm was developed to this end
\citep{Whitlock2008,Spitze1993}. $Q_{ST}$, defined as the ratio of the trait
variance between subpopulations to the total trait variance, is compared to
$F_{ST}$, which measures the same property for genetic variation and is
calculated using neutral markers to provide a null distribution. If the observed
$Q_{ST}$ is sufficiently far from the null expectation, it is concluded that
natural selection has acted. \citet{Ovaskainen2011} developed a modern extension
of the $Q_{ST}$/$F_{ST}$ paradigm for genetic values measured in breeding
experiments and \citet{Berg2014} also extended the paradigm to make use of
genetic values computed from GWAS summary statistics. Understanding the neutral
distribution of trait values at the sample and population level is necessary for
the development of goodness-of-fit tests suited to populations with complicated
histories and traits with sparse genetic architectures.

We generalize the work of \citet{Schraiber2015} by deriving the form of the
moment generating function (mgf) for arbitrary distributions of coalescent times
(e.g. those arising under exponential growth or an island model of migration).
The key result of \citet{Schraiber2015}, the characteristic function of the
sampling distribution of phenotypic values, is a special case of this general
generating function. We then show how a normal model arises by taking the
infinitesimal limit where the effect size per mutation becomes small as the
number of loci potentially affecting the trait becomes large. We then calculate
the third and fourth central moments of the trait distribution in panmictic
populations to illustrate how departures from normality depend both on genetic
parameters and genealogical distributions. For instance, in exchangeable
populations the expected third central moment is proportional to third central
moment of the mutational distribution times the expected time to the first
coalescent event in a sample of size three. Finally, we discuss the consequences
of these results for $Q_{ST}$ tests, the response to selection, and the
inference of genetic architecture.

We find an improved null distribution for $Q_{ST}$ tests that can be derived
simply by using the normal distribution that arises in the infinitesimal limit
of our coalescent model. Further, we find that when selection acts primarily on
the tails of the trait distribution, the single generation response depends on
trait moments that are sensitive to genetic architecture and demographic.
Additionally, we show how it may be possible to infer features of the mutational
distribution when both trait values and sequence data are available.

%%% Local Variables:
%%% TeX-master: "short_report.tex"
%%% End:
