Neutral models of quantitative traits provide a null distribution against which
various goodness-of-fit tests can be used to test for the action of natural
selection \citep{Lande1976,Leinonen2013}. Neutral models can also clarify the
effects of purely neutral forces such as genetic drift and mutation on trait
distributions \citep{Lynch1986}. A common approach is to first model phenotypes
as normally distributed, either among offspring within a family, among members
of a population, or among species \citep{Turelli2017}. Indeed, it has been
suggested that the normality assumption is the defining characteristic of
quantitative genetics \citep{Rice2004}. This might be justified if phenotypes
are influenced by a large number of sufficiently independent Mendelian
factors \citep{Fisher1918}, or normality may simply appear approximately true in
practice.

Neutral models for quantitative traits have been developed in a variety of
contexts. The goal of these models is to ask to whether phenotypic
differentiation between groups can be reasonably explained by processes other
than natural selection. On macroevolutionary time scales, models stemming
from \citet{Lande1976} have used Brownian motion to describe the evolution of
the mean value of a quantitative trait in a population. These models are used in
statistical methods to test for extreme trait divergence between
species \citep{Turelli1988}, test for phylogenetic signal in trait
distributions \citep{Freckleton2002}, and correct for phylogenetic dependence
when calculating correlations between traits \citep{Felsenstein1985}. On shorter
time scales, neutral distributions assuming multivariate normality of trait
values also underlie tests for spatially varying selection in structured
populations such as the method developed by \citet{Ovaskainen2011}. Other
neutral models for quantitative traits have not assumed
normality \citep{Chakraborty1982,Lynch1986,Lande1992}, and the dynamics of
phenotypic evolution are examined forwards in time as a balance between mutation
creating variance, migration spreading variance among subpopulations, and
fixation removing it. However, these studies were limited to simple models of
population structure and history. Backwards-in-time, coalescent models would
allow for more general demographic scenarios.

Under the normality assumption, quantitative trait dynamics can be modeled
without concern for the number of causal loci influencing the trait, the
genealogies at these sites, or the distribution of mutational effects (the
distribution that new mutations affecting the trait draw their effects from).
However, heritable phenotypic variation is ultimately due to discrete mutations
at discrete locations in the genome, and how the phenotypic variance arising
from these mutations is distributed depends on the genealogies at these loci.
For instance, the distribution of genealogies in the genome might be strongly
influenced by recent population growth and the distribution of mutational
effects could be skewed for biological reasons inherent in the details of a
particular developmental pathway. When the number of mutations affecting a trait
is large, the central limit theorem ensures that the distributions of
genealogies and mutational effects can be ignored, but a full model of
phenotypic variation would have to include them. Importantly, deviations from
normality may affect the outcomes of goodness-of-fit tests that necessarily aim
to identify outliers from a normal model.

A more inclusive model of neutral phenotypic variation can begin by considering
the genealogies at causal loci. The principle modeling framework for
genealogical variation is the coalescent process \citep{Wakeley2008}, but few
studies have connected the coalescent to quantitative
genetics. \citet{Whitlock1999} used coalescent theory to argue that measures of
phenotypic ($Q_{ST}$) and genetic ($F_{ST}$) differentiation have the same
expected value given general models of population subdivision. By simulating
from the coalescent with recombination,
\citet{Griswold2007} investigated the effects of shared ancestry and linkage
disequilibrium on the genetic covariance matrix for a set of traits
($\mbox{\textbf{G}}$ matrix). They found that linkage disequilibrium and small
numbers of causal loci can cause phenotypic covariances not predicted by the
mutational covariance matrix. \citet{Mendes2018} also used a quantitative trait
model based on the coalescent to show that using a species tree based on
population split times can lead to, among other problems, increased false
positive rates in phylogenetic comparative methods. Although not explicitly
connected to the coalescent, \citet{Ovaskainen2011} developed their test for
spatially varying selection by assuming that the covariance in trait values,
conditional on the G-matrix in the ancestral population, depends only on the
pairwise coancestry coefficients, which have a clear interpretation in terms of
the coalescent process \citep{Slatkin1991}.

Two studies have asked how the shape of the distribution of mutational effect
sizes, beyond just the mutational variance, impacts trait
distributions. \citet{Khaitovich2005} modeled the evolution of gene expression
values on phylogenetic trees assuming a single non-recombining causal locus but
allowed for an arbitrary distribution of mutational effects. Using this model
they were able to detect deviations from normality consistent with asymmetries
in the distribution of mutational effects on gene expression in great apes. More
recently, \citet{Schraiber2015} developed a similar general model of
quantitative trait evolution at the population level based on the coalescent and
allowing for any number of causal loci. They derived the characteristic function
for the distribution of phenotypic values in a sample and showed how such values
can deviate strongly from normality when the number of loci is small or the
mutational distribution has heavy tails. \citet{Schraiber2015} note that the
possibility for multimodal trait distributions could lead to incorrect
inferences of divergent selection within populations.

\citet{Schraiber2015} derived their results for a panmictic, constant-size
population. Natural populations rarely have stable population sizes and show
considerable spatial structure, and it is unclear how these violations of the
constant-size, panmictic model might influence deviations from normality. We
take advantage of the ability of coalescent theory to handle nonequilibrium
demographies and population structure to relax these modeling assumptions.

Extending coalescent models of quantitative traits to structured populations is
important because the analysis of structured populations provides an opportunity
to infer the incidence of local adaptation or stabilizing selection. Due to the
build up of linkage disequilibrium between alleles affecting a trait, the
neutral divergence in trait values among different subpopulations in a
structured population is about as variable as the variance in allele
frequencies \citep{Rogers1983}. The $Q_{ST}$/$F_{ST}$ paradigm was developed to
test whether trait divergence in structured populations could be explained by
neutral forces alone \citep{Spitze1993,Whitlock2008,Leinonen2013}. $Q_{ST}$,
defined as the ratio of the trait variance between subpopulations to the total
trait variance, is compared to $F_{ST}$, which measures the same property for
genetic variation and is calculated using neutral markers to provide a null
distribution. If the observed $Q_{ST}$ is sufficiently far from the null
expectation, it is concluded that natural selection has
acted. \citet{Ovaskainen2011} developed a modern extension of the
$Q_{ST}$/$F_{ST}$ paradigm for genetic values measured in breeding experiments
and \citet{Berg2014} also extended the paradigm to make use of genetic values
computed from GWAS summary statistics. An advantage of the
\citet{Berg2014} approach is that by using computing genetic values from GWAS
loci it makes no assumptions about normality at the population level. However,
since suitably sized GWAS's have only been performed in humans, the approach has
not yet been extended to other species. Understanding the neutral distribution
of trait values is therefore necessary for the development of goodness-of-fit
tests that are applicable to populations with complicated histories and traits
with sparse genetic architectures.

We generalize the work of \citet{Schraiber2015} by deriving the form of the
moment generating function (mgf) for arbitrary distributions of coalescent times
(e.g. those arising under exponential growth or an island model of migration).
The key result of \citet{Schraiber2015}, the characteristic function of the
sampling distribution of phenotypic values, is a special case of this general
generating function. We then show how a normal model arises by taking the
infinitesimal limit where the effect size per mutation becomes small as the
number of loci potentially affecting the trait becomes large. We then calculate
the third and fourth central moments of the trait distribution in panmictic
populations to illustrate how departures from normality depend both on genetic
parameters and genealogical distributions. For instance, in exchangeable
populations the expected third central moment is proportional to the third
noncentral moment of the mutational distribution times the expected time to the
first coalescent event in a sample of size three.

Finally, we discuss the consequences of these results for $Q_{ST}$ tests and the
inference of genetic parameters. We find an improved null distribution that
can be derived simply by using the normal distribution that arises in the
infinitesimal limit of our coalescent model. Additionally, we show that it is
likely not possible to infer most features of the mutational distribution using
only trait values sampled from a population. Future work will be necessary to
develop tests for selection that take into account both demography and genetic
parameters, but the model developed here provides the groundwork for such an
undertaking.

%%% Local Variables:
%%% TeX-master: "quant_gen_manu.tex"
%%% End:
