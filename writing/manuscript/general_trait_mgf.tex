We first derive the mgf of the distribution of trait values following closely
the approach of \citet{Schraiber2015} and \citet{Khaitovich2005}, but
generalizing to arbitrary demographies and population structure. We consider the
distribution of trait values over evolutionary realizations of the combined
random processes of drift and mutation. The probability distribution for a trait
is complex in its general form. There is a point mass at zero corresponding the
possibility that no mutations occur, and mutational effects could be drawn from
discrete or continuous distributions. Correlations between individuals arise
because of shared history in genealogies at individual loci. Deriving the
cumulative distribution function for trait values may be possible in some
instances, but the difficulty of integrating over mutational configurations
makes it practically impossible for most cases.

However, we can use the mgf approach to study the distribution of trait values.
Following the definition of the mgf for a vector-valued random
variable \citep{Ross}, the mgf for a trait controlled by a single nonrecombining
locus is
\begin{equation}
  \label{eq:mgfdef}
  \varphi_{\mathbf{Y}}(\mathbf{k}) = \E\left[ e^{\mathbf{k} \cdot \mathbf{y}} \right] =
  \int e^{\mathbf{k} \cdot \mathbf{Y}} \Pro(\mathbf{Y}=\mathbf{y}) \mbox{d}\mathbf{y}.
\end{equation}
The vector $\mathbf{k}$ contains dummy variables for each individual, and the
whole operation is an intergral transform of the probability distribution of
trait values. Equation \eqref{eq:mgfdef} can be rewritten by conditioning on the
genealogy to give
\begin{align}
  \label{eq:cond}
  \varphi_{\mathbf{Y}}(\mathbf{k}) &= \int_{\mathbf{Y}} e^{\mathbf{k} \cdot \mathbf{y}}
  \int_{\mathbf{T}} \Pro(\mathbf{Y}=\mathbf{y} | \mathbf{T}=\mathbf{t}) \Pro(\mathbf{T}=\mathbf{t})
  \mbox{d}\mathbf{t} \mbox{d}\mathbf{y} \nonumber \\
  &= \int \int e^{\mathbf{k} \cdot \mathbf{y}} \Pro(\mathbf{Y}=\mathbf{y} | \mathbf{T}=\mathbf{t}) \mbox{d}\mathbf{y}
  \Pro(\mathbf{T}=\mathbf{t})
  \mbox{d}\mathbf{t}.
\end{align}

To proceed, assumptions about the mutational process must be made. The first is
that mutations occur as a Poisson process along branches and the second is that
mutations at a locus are additive. The changes in the trait value along each
branch are then conditionally independent given the branch
lengths. \citet{Khaitovich2005} and \citet{Schraiber2015} noted that this
describes a compound Poisson process. The mgf of a compound Poisson process with
rate $\lambda$ over time $t$ is $\exp(\lambda t (\psi(k)-1))$, where $\psi$ is
the mgf of the distribution of jump sizes caused by events in the Poisson
process \citep{kingman-poisson-processes}. Here, jump sizes are the effects on
the trait value caused by new mutations. Using the mgf of a compound Poisson
process, along with the fact that the mgf of the joint distribution of two
perfectly correlated random variables with the same marginal distribution is
$\varphi_{X_1,X_2}(k_1,k_2)=\varphi_{X_1}(k_1+k_2)$, we can rewrite
equation \eqref{eq:cond} as
\begin{equation}
  \label{eq:fullmgf}
  \varphi_{\mathbf{Y}}(\mathbf{k}) = 
  \int \prod_{\omega \in \Omega} \exp\left( \frac{\theta}{2} t_{\omega} \left( \psi\left(\sum_{a \in \omega}k_{a}\right) -1 \right)\right)
  \Pro(\mathbf{T}=\mathbf{t})\mbox{d}\mathbf{t}.
\end{equation}
Equation \eqref{eq:fullmgf} is the mgf of $\mathbf{T}$ with the dummy variable
$s_{\omega}$ for branch $T_\omega$ set to
$\frac{\theta}{2} \left( \psi(\sum_{a \in \omega}k_{a}) -1 \right)$. This
implies that
\begin{equation}
  \label{eq:sub}
  \varphi_{\mathbf{T}}(\mathbf{s})\Bigr|_{s_{\omega}=\frac{\theta}{2} \left( \psi\left(\sum_{a \in \omega}k_{a}\right) -1 \right)}.
\end{equation}

Thus, if the mgf of the distribution of branch lengths is known, equation
\eqref{eq:fullmgf} allows us to obtain the mgf of the trait values through a
substitution. When the trait is controlled by $L$ independent loci, the full mgf
is obtained by raising equation \eqref{eq:sub} to the power $L$. This result
obviates the need for separate derivations for particular models of population
history and structure.
\citet{Lohse2011} derived the mgf of the genealogy in various population models
including migration and splitting of subpopulations. Using their result for a
single population we can obtain equation (1) of \citet{Schraiber2015} using
equation \eqref{eq:sub} and equation (5) of \citet{Lohse2011}
(Appendix \ref{slrederive}). 

%%% Local Variables:
%%% TeX-master: "quant_gen_manu.tex"
%%% End:
