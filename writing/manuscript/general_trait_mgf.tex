We first derive the mgf of the distribution of trait values following closely
the approach of \citet{Schraiber2015} and \citet{Khaitovich2005}, but
generalizing to arbitrary demographies and population structure. We consider the
distribution of trait values over evolutionary realizations of the combined
random processes of drift and mutation. This trait distribution is complex in
its general form. There is a point mass at zero corresponding the possibility
that no mutations affecting the trait occur, and mutational effects could be
drawn from discrete or continuous distributions. Correlations between
individuals arise because of shared history in the genealogies at individual
loci with discrete topologies as well as because of where on these genealogies
mutations occur. An analytical expression for the probability distribution of
trait values does not exist except in certain limits such as when the number of
mutations affecting the trait becomes large.

However, even in the absence of a probability distribution function we can use
the mgf approach to learn something about the distribution of trait values. In
some cases, the mgf can be fully specified. Following the definition of the mgf
for a vector-valued random variable, the mgf for a trait controlled by a single
nonrecombining locus is
\begin{equation}
  \label{eq:mgfdef}
  \varphi_{\mathbf{Y}}(\mathbf{k}) = \E\left[ e^{\mathbf{k} \cdot \mathbf{y}} \right] =
  \int e^{\mathbf{k} \cdot \mathbf{Y}} \Pro(\mathbf{Y}=\mathbf{y}) \mbox{d}\mathbf{y}.
\end{equation}
The vector $\mathbf{k}$ contains dummy variables for each individual, and the
whole operation is an intergral transform of the probability distribution of
trait values. Equation \eqref{eq:mgfdef} can be rewritten by conditioning on the
genealogy to give
\begin{align}
  \label{eq:cond}
  \varphi_{\mathbf{Y}}(\mathbf{k}) &= \int_{\mathbf{Y}} e^{\mathbf{k} \cdot \mathbf{y}}
  \int_{\mathbf{T}} \Pro(\mathbf{Y}=\mathbf{y} | \mathbf{T}=\mathbf{t}) \Pro(\mathbf{T}=\mathbf{t})
  \mbox{d}\mathbf{t} \mbox{d}\mathbf{y} \nonumber \\
  &= \int \int e^{\mathbf{k} \cdot \mathbf{y}} \Pro(\mathbf{Y}=\mathbf{y} | \mathbf{T}=\mathbf{t}) \mbox{d}\mathbf{y}
  \Pro(\mathbf{T}=\mathbf{t})
  \mbox{d}\mathbf{t}.
\end{align}

To proceed it is necessary to make assumptions about the mutational process. The
first is that mutations occur as a Poisson process along branches and the second
is that mutations at a locus are additive. Under these assumptions, the changes
in the trait value along each branch are conditionally independent given the
branch lengths. \citet{Khaitovich2005} and \citet{Schraiber2015} noted that this
describes a compound Poisson process. The mgf of a compound Poisson process with
rate $\lambda$ over time $t$ is $\exp(\lambda t (\psi(k)-1))$, where $\psi$ is
the mgf of the distribution of the jump sizes caused by events in the Poisson
process. In this case the jump sizes are the effects on the trait value caused
by new mutations. Using this expression of the mgf of a compound Poisson
process, along with the fact that the mgf of two perfectly correlated random
variables with the same marginal distribution is $\varphi_{X_1}(k_1+k_2)$, we
can rewrite equation \eqref{eq:cond} as
\begin{equation}
  \label{eq:fullmgf}
  \varphi_{\mathbf{Y}}(\mathbf{k}) = 
  \int \prod_{\omega \in \Omega} \exp\left( \frac{\theta}{2} t_{\omega} \left( \psi\left(\sum_{a \in \omega}k_{a}\right) -1 \right)\right)
  \Pro(\mathbf{T}=\mathbf{t})\mbox{d}\mathbf{t}.
\end{equation}
We recognize equation \eqref{eq:fullmgf} as the mgf for $\mathbf{T}$ with the
dummy variable $T_{\omega}$ for branch $s_\omega$ equal to
$\frac{\theta}{2} \left( \psi(\sum_{a \in \omega}k_{a}) -1 \right)$. Or,
\begin{equation}
  \label{eq:sub}
  \varphi_{\mathbf{T}}(\mathbf{s})\Bigr|_{s_{\omega}=\frac{\theta}{2} \left( \psi\left(\sum_{a \in \omega}k_{a}\right) -1 \right)}.
\end{equation}

Thus, if the mgf of the distribution of branch lengths is known, equation
\eqref{eq:fullmgf} allows us to obtain the mgf of the trait values through a
simple substitution. When the trait is controlled by $L$ independent loci, the
full mgf, $\varphi_{\mathbf{Y}}(\mathbf{k})$, is obtained by raising
equation \eqref{eq:sub} to the power $L$. This result obviates the need for any
special derivations for particular models of population history and structure.
\citet{Lohse2011} derived the mgf of the genealogy in various population models
including migration and splitting of subpopulations. Using their result for a
single population we can obtain equation (1) of \citet{Schraiber2015} using
equation \eqref{eq:sub}. 

%%% Local Variables:
%%% TeX-master: "short_report.tex"
%%% End:
