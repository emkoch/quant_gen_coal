Neutral models for quantitative trait evolution are useful for identifying
phenotypes under selection. These models often assume normally distributed
phenotypes. This assumption may be violated when a trait is affected by
relatively few genetic variants or when the effects of those variants arise from
skewed or heavy-tailed distributions. Molecular phenotypes such as gene
expression levels may have these properties. To accommodate deviations from
normality, models making fewer assumptions about the underlying genetics and
patterns of variation are needed. Here, we develop a general neutral model for
quantitative trait variation using a coalescent approach. This model allows
interpretation of trait distributions in terms of familiar population genetic
parameters because it is based on the coalescent. We show how the normal
distribution resulting from the infinitesimal limit, where the number of loci
grows large as the effect size per mutation becomes small, depends only on
expected pairwise coalescent times. We then demonstrate how deviations from
normality depend on demography through the distribution of coalescence times as
well as through genetic parameters. In particular, population growth events
exacerbate deviations while bottlenecks reduce them. We demonstrate the
practical applications of this model by showing how to sample from the neutral
distribution of $Q_{ST}$, the ratio of the variance between subpopulations to
that in the overall population. We further show it is likely impossible to
distinguish sparsity from skewed or heavy-tailed mutational effects using only
sampled trait values. The model analyzed here greatly expands the possible
parameter space for neutral trait models.

%%% Local Variables:
%%% TeX-master: "quant_gen_manu.tex"
%%% End:
