Neutral models for quantitative trait evolution have been useful for identifying
phenotypes under selection in natural populations through goodness-of-fit tests.
Such models nearly always start by assuming phenotypes are normally distributed,
but traits such as gene expression levels may violate this assumption and have
sparse genetic architectures. We develop and analyze a neutral model for
quantitative traits that makes minimal assumptions about the genetic
architecture or the distribution of coalescent times and hence the structure or
demography of the population. This is done from a coalescent perspective by
extending a model developed by \citet{Schraiber2015} whose central result, the
characteristic function for the distribution of trait values, is shown to be a
special case of ours. We analyze this model to show how a normal distribution is
approached in an infinitesimal limit as the number of loci gets large and the
effect size of mutations becomes small. We demonstrate the utility of this
limiting distribution by designing an approach to simulate from the null
distribution of $Q_{ST}$, the ratio of the variance between subpopulations to
that in the overall population, in the infinitesimal limit. We also demonstrate
how deviations from the normal distribution depend on both demography and the
genetic architecture. In particular, population growth exacerbates deviations
while bottlenecks reduce them. Finally, we show that it is likely impossible to
distinguish sparsity from skewed or fat-tailed distributions of mutational
effects using only trait values sampled from a population.

%%% Local Variables:
%%% TeX-master: "quant_gen_manu.tex"
%%% End:
