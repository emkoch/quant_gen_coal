Neutral models for quantitative trait evolution are useful for identifying
phenotypes under selection in natural populations. Models of quantitative traits
often assume phenotypes are normally distributed. This assumption may be
violated when traits are affected by relatively few genetic variants or when
those variants have skewed or heavy-tailed distributions of effects on the trait.
Traits such as gene expression levels and other molecular phenotypes may fall
into this category. To accommodate deviations from normality, models making
minimal assumptions about genetic architecture and patterns of genetic variation
are needed. Here, we develop a general neutral model for quantitative trait
variation using a coalescent approach by extending the framework developed
by \citet{Schraiber2015}. This model allows interpretation of trait
distributions in terms of classical population genetic parameters because it is
based on the coalescent. We show how the normal distribution resulting from the
infinitesimal limit, where the number of loci grows large as the effect size per
mutation becomes small, depends only on expected pairwise coalescent times. We
then demonstrate how deviations from normality depend on demography through the
distribution of coalescence times as well as through the genetic architecture.
In particular, population growth events exacerbate deviations while bottlenecks
reduce them. This model also has practical applications which we demonstrate by
designing an approach to simulate from the null distribution of $Q_{ST}$, the
ratio of the variance between subpopulations to that in the overall population.
We further show that it is likely impossible to distinguish sparsity from skewed
or heavy-tailed distributions of mutational effects using only trait values
sampled from a population. The model analyzed here greatly expands the parameter
space for which neutral trait models can be designed.

%%% Local Variables:
%%% TeX-master: "quant_gen_manu.tex"
%%% End:
