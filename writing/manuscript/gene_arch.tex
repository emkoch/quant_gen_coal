\citet{Schraiber2015} suggested it might be possible to infer the shape of the
distribution of mutational effects through its moments for sparse traits whose
distributions deviate from normality. Using the expressions for the expected
moments of the trait value distribution in equation \eqref{eq:emoms}, we could
try and design a method of moments estimator for the moments of the mutational
distribution. If $\hat{D}_{i,j}$ is an estimator of $\T \E[\mathbbm{T}_{i,j}]$,
the system of equations for the first three central moments under the low
mutation rate approximation is
\begin{align}
\label{eq:genearch}
  &\widehat{M}_2 = \widehat{D}_{2,2}Lm_2 \nonumber \\
  &\widehat{M}_3 = \frac{1}{3}\widehat{D}_{3,3}Lm_3 \nonumber \\
  &\widehat{M}_4 = 3\widehat{M}_2^2 + (\widehat{D}_{4,4} + \frac{1}{3} \widehat{D}_{3,4} + \frac{2}{9} \widehat{D}_{2,4})Lm_4.
\end{align}
From equation \eqref{eq:genearch} we can see that the moments of the trait
distribution only enter through products with the number of loci potentially
affecting the trait ($Lm_i$). If these products were estimated as
$\widehat{Lm_2}$, $\widehat{Lm_3}$, and $\widehat{Lm_4}$, an estimate of the
kurtosis would need an additional estimate of $L$: $\hat{\kappa}
= \frac{\hat{L}\widehat{Lm_4}}{(\widehat{Lm_2})^2}$. This implies that it is not
possible to distinguish, using the trait moments at the population level,
whether the mutational distribution has a large kurtosis or is controlled by a
small number of loci.

%%% Local Variables:
%%% TeX-master: "quant_gen_manu.tex"
%%% End:
