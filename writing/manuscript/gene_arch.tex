\citet{Schraiber2015} suggested it might be possible to infer something about
the genetic architecture and distribution of mutational effects for sparse
traits whose distributions deviate from normality. Using the expressions for the
expected moments in equation \eqref{eq:emoms} we could imagine a method of
moments estimator for these quantities. If $\hat{D}_{i,j}$ is an estimator of
$\T \E[\mathbbm{T}_{i,j}]$, the system of equations for the first three central
moments under the low mutation rate approximation is
\begin{align}
\label{eq:genearch}
  &\hat{M}_2 = \hat{D}_{2,2}Lm_2 \nonumber \\
  &\hat{M}_3 = \frac{1}{3}\hat{D}_{3,3}Lm_3 \nonumber \\
  &\hat{M}_4 = 3\hat{M}_2^2 + (\hat{D}_{4,4} + \frac{1}{3} \hat{D}_{3,4} + \frac{2}{9} \hat{D}_{2,4})Lm_4.
\end{align}
From equation \eqref{eq:genearch} we can see that the moments of the trait
distribution only enter through products with the number of loci potentially
affecting the trait. Thus, while we can infer the shape of the mutational
distribution, we cannot discern the number of loci affecting a trait from the
magnitude of mutational effect sizes.

%%% Local Variables:
%%% TeX-master: "short_report.tex"
%%% End:
