\citet{Schraiber2015} suggested it might be possible to infer the shape of the
distribution of mutational effects through its moments for sparse traits whose
distributions deviate from normality. Using the expressions for the expected
moments of the trait value distribution in equation \eqref{eq:emoms}, we could
try and design a method of moments estimator for the moments of the mutational
distribution. If $\hat{D}_{k,n}$ is an estimator of $\T \E[\mathbbm{T}_{k,n}]$,
the system of equations for the first three central moments under the low
mutation rate approximation is
\begin{align}
\label{eq:genearch}
  &\widehat{M}_2 = \widehat{D}_{2,2}Lm_2 \nonumber \\
  &\widehat{M}_3 = \frac{1}{3}\widehat{D}_{3,3}Lm_3 \nonumber \\
  &\widehat{M}_4 = 3\widehat{M}_2^2 + (\widehat{D}_{4,4} + \frac{1}{3} \widehat{D}_{3,4} + \frac{2}{9} \widehat{D}_{2,4})Lm_4.
\end{align}
From equation \eqref{eq:genearch} we can see that the moments of the trait
distribution only enter through products with the number of loci potentially
affecting the trait ($Lm_i$). If these products were estimated as
$\widehat{Lm_2}$, $\widehat{Lm_3}$, and $\widehat{Lm_4}$, it would be possible
to estimate the ratios $m_2/m_4$ and $m_3/m_4$ of the moments of the mutational
distribution. These ratios are meaningless on their own because any value could
be obtained by changing the scale on which the trait is measured. The quantity
that is identifiable in this system of equation is the compound parameter
$\frac{m_3^2}{m_2m_4}$. This quantity reflects something about the mutational
bias relative to the spread of the distribution. However, it is likely not
possible to distinguish sparsity from skewed or heavy-tailed distributions of
mutational effects. 

%%% Local Variables:
%%% TeX-master: "quant_gen_manu.tex"
%%% End:
