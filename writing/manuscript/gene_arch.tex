\citet{Schraiber2015} suggested it might be possible to estimate moments of the 
distribution of mutational effects when trait distributions deviate from
normality. These moments would provide information about the shape of the
mutational distribution. Using the expected moments of the trait distribution
(equation \eqref{eq:emoms}), we could attempt a method of moments estimator for
the moments of the mutational distribution. The system of equations for the
first three central moments is
\begin{align}
\label{eq:genearch}
  &\widehat{M}_2 = \widehat{D}_{2,2}Lm_2 \nonumber \\
  &\widehat{M}_3 = \frac{1}{3}\widehat{D}_{3,3}Lm_3 \nonumber \\
  &\widehat{M}_4 = 3\widehat{M}_2^2 + (\widehat{D}_{4,4} +
  \frac{1}{3} \widehat{D}_{3,4} + \frac{2}{9} \widehat{D}_{2,4})Lm_4,
\end{align}
where $\hat{D}_{k,n}$ is an estimator of $\T \E[\mathbbm{T}_{k,n}]$, and the low
mutation rate approximation has been used. The moments of the trait distribution
only enter through products with the number of loci potentially affecting the
trait ($Lm_i$). If we have the estimates $\widehat{Lm_2}$, $\widehat{Lm_3}$, and
$\widehat{Lm_4}$, it is possible to estimate the ratios $m_2/m_4$ and $m_3/m_4$
for the mutational distribution. These ratios are meaningless on their own
because any value could be obtained by changing the scale on which the trait is
measured. The identifiable quantity in this system of equation is the compound
parameter $\frac{m_3^2}{m_2m_4}$. This quantity reflects something about the
mutational bias relative to the spread of the distribution, but it is not
possible to distinguish sparsity from skewed or heavy-tailed distributions of
mutational effects.

%%% Local Variables:
%%% TeX-master: "quant_gen_manu.tex"
%%% End:
