In the model we investigate here, there are $L$ unlinked potentially causal loci
at which mutations influence the trait value. Following \citet{Kimura1969}, an
infinite number of mutations are possible within each locus and the mutation
rate per unit of coalescent time to mutations affecting the trait is $\T$. That
is, $\T$ is the mutation rate for the entire locus and not per nucleotide. An
approximation for when at most one mutation per locus is likely is considered in
section \ref{sec:lmr}. Mutations are randomly assigned effects from a
distribution of effect sizes, and effects are additive both within and between
loci. The moment generating function (mgf) of this distribution is written as
$\psi$ and the $i^{th}$ (non-central) moment is $m_i$. Individuals are haploid
and the sum of all mutations occurring in an individual's history determines the
trait value of the individual. An extension to diploidly would be
straightforward but is not considered here. Correlations between individuals
arise when mutations fall on shared portions of genealogies at specific loci.
Because the loci are unlinked we assume their genealogies are independent. This
model is shown schematically in Figure \ref{fig:schema}.

The genealogy at a locus is represented by the random vector of branch lengths,
$\mathbf{T}$. An element $T_{\omega}$ of $\mathbf{T}$ is the branch length
subtending only individuals in the set $\omega$ and no others in the sample. For
example, $T_{\{a,b\}}$ is the length of the branch subtending only individuals
$a$ and $b$. If a branch subtending only $a$ and $b$ does not exist for a given
genealogy, $T_{\{a,b\}}$ is set to zero. In this way $\mathbf{T}$ encodes both
the branch lengths and the topology of a genealogy. $\Omega$ is the set of all
possible branches. If there are three sampled individuals, $a$, $b$, and $c$,
then $\Omega=\{\{a\},\{b\},\{c\},\{a,b\},\{a,c\},\{b,c\}\}$ and
$\mathbf{T}=(T_{\{a\}},T_{\{b\}},T_{\{c\}},T_{\{a,b\}},T_{\{a,c\}},T_{\{b,c\}})$.
The mgf for the distribution of branch lengths is denoted $\varphi_{\mathbf{T}}$.

Phenotypic trait values are the random quantities we are interested in and
result from mutations occuring along the branches of genealogies. We will
hereafter refer to the phenotypic trait simply as trait values and ignore any
environmental component. Starting with a trait controlled by a single locus, the
random vector of trait values in the sampled individuals is $\mathbf{Y}$, such
that for sampled individuals $a$, $b$, and $c$, $\mathbf{Y}=(Y_a,Y_b,Y_c)$. The
trait values are modeled as the change relative to the value in the most recent
common ancestor (MRCA) of the sample at that locus. Since we do not know the
ancestral value, we cannot directly observe the change in trait values. Thus,
for a trait controlled by multiple loci, $\mathbf{Y}$ is the sum over
contributions from these loci, each measured with respect to an arbitrary value.
However, $\mathbf{Y}$ is sufficient to determine measurable quantities such as
differences in trait values between individuals and the sample variance. The
moment generating functions for the distribution of trait values is denoted
$\varphi_{\mathbf{Y}}$.

\begin{figure}
  \centering
  \includegraphics[width=0.9\textwidth]{./figures/schema.png}

  \caption{\textbf{A schematic representation of the model of how we model a
  trait distributions arising from genealogical and mutational distributions.}
  $L$ loci affect the trait in a set of individuals and have independent
  genealogies. Mutations occur within loci as a Poisson process and act
  additively to give individual trait values. Many loci with the potential to
  affect the trait may receive no mutations.}

  \label{fig:schema}
\end{figure}

Here we refer to the genetic parameters of a trait as the combination of
quantities not influenced by the genealogical process: ($L$, $\T$, $\psi$).
Another quantity useful for describing a trait's distribution is its sparsity.
Sparsity should reflect how many mutations segregating in the population
influencing the trait, with a more `sparse' trait being one affected by fewer
segregating mutations. Formally, we measure sparsity as the average number of
pairwise differences between two randomly chosen haplotypes at loci affecting
the trait. A trait with fewer causal pairwise differences is more sparse.
Sparsity thus depends both on the genetic parameters through the mutation rate,
the number of potentially causal loci, and the distribution of coalescence
times.

In populations of exchangeable individuals, a concise way to summarize the
distribution of genealogies is $\mathbbm{T}_{k,n}$ which gives the amount of
time that $i$ lineages remain in the genealogy of a sample of size $j$. The
pairwise coalescent time between a lineage in individual $i$ and in individual
$j$ is written as $\mathcal{T}_{i,j}$. When considering structured populations,
$\mathcal{T}_{a,b}$ is also used to denote the coalescence time between a
randomly chosen lineage from subpopulation $a$ and a randomly chosen lineage
from subpopulation $b$. A final set of quantities are defined for sums of branch
lengths. Let $\tau_{a+b}$ be the sum of all branches ancestral to both $a$ and
$b$, and $\tau_{a/b}$ be the sum of all branches ancestral to $a$ but not $b$.
Extensions of this for more than two individuals are also used. The same
notation is used when referring to sets of branch indices. So $\Omega_{a+b}$ and
$\Omega_{a/b}$ would be the sets of branches added to get $\tau_{a+b}$ and
$\tau_{a/b}$ respectively.

%%% Local Variables:
%%% TeX-master: "short_report.tex"
%%% End:
