\begin{table}
  \caption{Notation used in this article}
  \centering
  \begin{tabular}{l l l}
    \hline
    \multicolumn{1}{c}{Type of quantity} &\multicolumn{1}{c}{Symbol} & \multicolumn{1}{c}{Description} \\
    \hline
    Genetic parameter & $\psi$ & \multicolumn{1}{p{10cm}}{Mgf of the mutational effect size distribution }\\
                                         & $\frac{\theta}{2}$ & \multicolumn{1}{p{10cm}}{Per-locus mutation rate in units of coalescent time}\\
                                         & $m_i$ &  \multicolumn{1}{p{10cm}}{$i^{th}$ non-central moment of the mutational effect size distribution}\\
                                         & $\frac{\theta}{2}\mu_1$ & \multicolumn{1}{p{10cm}}{Rate that mutational bias shifts trait values in the infinitesimal limit}\\
                                         & $\frac{\theta}{2}\mu_2$ & \multicolumn{1}{p{10cm}}{Rate that variance accumulates in the infinitesimal limit}\\
                                         & $L$ & \multicolumn{1}{p{10cm}}{Number of potentially causal loci}\\
    Genealogy value &  $\Omega$ & \multicolumn{1}{p{10cm}}{Set of all possible branches in the genealogy of a given sample}\\
                                         & $\omega$ & \multicolumn{1}{p{10cm}}{A particular branch in a genealogy, defined by all individuals that branch subtends}\\
                                         & $\mathbf{T}$ & \multicolumn{1}{p{10cm}}{Random vector of branch lengths representing the entire genealogy at a locus}\\
                                         & $T_\omega$ & \multicolumn{1}{p{10cm}}{Length of branch $\omega$} \\
                                         & $T_{MRCA}$ & \multicolumn{1}{p{10cm}}{Time to the most recent common ancestor at a particular locus}\\
                                         & $\varphi_{\mathbf{T}}(\mathbf{s})$ & Mgf of the genealogy distribution \\
                                         & $\mathbf{s}$ & Vector of dummy variables for the length of each possible branch\\
                                         & $s_\omega$ & Dummy variable for branch length $T_{\omega}$ \\
                                         & $\mathbbm{T}_{k,n}$ & \multicolumn{1}{p{10cm}}{For a sample of exchangeable lineages, the amount of time that $k$ lineages remain in a sample of size $n$}\\
                                         & $\mathcal{T}_{a,b}$ & \multicolumn{1}{p{10cm}}{Pairwise coalescent time between a lineage sampled from $a$ and a lineage sampled from $b$, depending on the context, $a$ and $b$ may be either individuals or subpopulations}\\
                                         & $\tau_{a+b}$ & Sum of all branches ancestral to individuals $a$ and $b$\\
    Trait value & $\mathbf{Y}$ & Random vector of trait values \\
                                         & $Y_a$ & Trait value of individual $a$\\
                                         & $\varphi_{\mathbf{Y}}(\mathbf{k})$ & Mgf of the trait value distribution \\
                                         & $\mathbf{k}$ & \multicolumn{1}{p{10cm}}{Vector of dummy variables for each individual trait value}\\
                                         & $k_a$ & Dummy variable for the trait value of individual $a$\\
                                         & $M_i$ & \multicolumn{1}{p{10cm}}{$i^{th}$ central moment of the trait value distribution in the entire population}\\
                                         & $CVV$ & \multicolumn{1}{p{10cm}}{Coefficient of variation of trait variance over evolutionary realizations of a population}\\
    \hline
  \end{tabular}
  \label{notation}
\end{table}

We consider a trait controlled by $L$ potentially causal loci (shown
schematically in Figure \ref{fig:schema}). Loci are unlinked and the effects of
recombination are not explored. Following \citet{Kimura1969}, an infinite number
of mutations are possible within each locus (though the number of loci is
finite) and mutations affecting the trait (causal mutations) arise at rate $\T$.
That is, $\T$ is the rate for one entire locus and not per nucleotide. An
approximation restricting the number of mutations per locus to at most one is
considered in Section \ref{sec:lmr}. Mutations receive values from a
distribution of effect sizes. This distribution is described by its moment
generating function (mgf), $\psi$, and its non-central moments, $m_i$.
Individuals are haploid and their trait values are determined by summing, both
within and between loci, the effects of all mutations occurring since the most
recent common ancestor. Environmental effects are not included. An extension to
diploidy would be straightforward but is not considered here.

The genealogy at a locus is represented by the random vector of branch lengths,
$\mathbf{T}$. An element $T_{\omega}$ of $\mathbf{T}$ is the branch length
subtending only individuals in the set $\omega$. For example, $T_{\{a,b\}}$ is
the length of the branch ancestral only to individuals $a$ and $b$. If such a
branch does not exist for a given genealogy, $T_{\{a,b\}}$ is set to zero. In
this way $\mathbf{T}$ encodes both the branch lengths and topology of a
genealogy. $\Omega$ is the set of all possible branches. For three sampled
individuals, $a$, $b$, and $c$,
$\Omega=\{\{a\},\{b\},\{c\},\{a,b\},\{a,c\},\{b,c\}\}$ and
$\mathbf{T}=(T_{\{a\}},T_{\{b\}},T_{\{c\}},T_{\{a,b\}},T_{\{a,c\}},T_{\{b,c\}})$.
The distribution of genealogies is also described by its mgf,
$\varphi_{\mathbf{T}}$. Genealogies are independent between loci because of the
lack of linkage.

Phenotypic trait values are random quantities of ultimate interest and are
hereafter referred to simply as trait values. The vector of trait values in the
sampled individuals is $\mathbf{Y}$. If we had sampled individuals $a$, $b$, and
$c$, then $\mathbf{Y}=(Y_a,Y_b,Y_c)$. The contribution to the trait values from
a single locus is the change relative to the value in the most recent common
ancestor (MRCA) of the sample at that locus. $\mathbf{Y}$ is the sum over
contributions from $L$ loci, each measured with respect to an arbitrary
ancestral value. Since we do not know the ancestral value, we cannot directly
observe $\mathbf{Y}$. However, $\mathbf{Y}$ determines measurable quantities
such as differences in trait values between individuals as well as the sample
variance. The distribution of trait values is also studied through its mgf,
$\varphi_{\mathbf{Y}}$.

\begin{figure}
  \centering
  \includegraphics[width=0.9\textwidth]{./figures/schema.png}

  \caption{\textbf{How trait distributions arise from genealogical and
      mutational processes under the model.} $L$ causal loci potentially affect
    the trait in a set of individuals and have independent genealogies.
    Mutations occur within loci as a Poisson process and act additively to
    determine individual trait values. Many loci with the potential to affect
    the trait may receive no mutations.}

  \label{fig:schema}
\end{figure}

We call those quantities not influenced by the genealogical process ($L$, $\T$,
and $\psi$) the genetic parameters of the trait. Another quantity useful for
describing a trait is its sparsity. Sparsity should reflect the number of
segregating mutations that influence the trait, with a more `sparse' trait being
affected by fewer segregating mutations. Formally, we measure sparsity as the
average number of pairwise differences between two randomly chosen haplotypes at
loci affecting the trait. A trait with fewer causal pairwise differences is more
sparse. Sparsity thus depends both on the genetic parameters through the
mutation rate and the number of potentially causal loci, and on demography
through the distribution of coalescence times. A trait with fewer causal loci in
a larger population may therefore have lower sparsity than a trait with more
causal loci in a smaller population.

%% We define sparsity using pairwise differences rather than the total number of
%% segregating mutations because this latter quantity depends on sample size and is
%% ill-defined for an entire population.

In populations of exchangeable individuals, a useful way to summarize the
distribution of genealogies is through the moments of $\mathbbm{T}_{k,n}$ which
denotes the amount of time that $k$ lineages remain in the genealogy of a sample
of size $n$. The pairwise coalescent time between a lineage in individual $a$
and a lineage in individual $b$ is written as $\mathcal{T}_{a,b}$. When
considering structured populations, $\mathcal{T}_{a,b}$ is also used to denote
the coalescence time between a randomly chosen lineage from subpopulation $a$
and a randomly chosen lineage from subpopulation $b$. Table \ref{notation}
provides a reference for the notation used in this article.

%%% Local Variables:
%%% TeX-master: "quant_gen_manu.tex"
%%% End:
