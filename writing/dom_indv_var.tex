When modeling dominance each individual receives two copies of alleles from the
population. These may contain the same or different alleles. An individual's
genotype is then made up of contributions from $L$ loci.
\begin{equation}
  Y=\sum_{l=1}^L f(Y_{l,1},Y_{l,2}).
\end{equation}
The function $f$ is a function that describes the ``dominance'' relationship for
the trait. It is natural to set $f(0,0)=0$ so that there is no effect when an
individual receives no mutations at a locus. If we assume that each locus has an
independent genealogy, then the variance of $Y$ can be computed by calculating
$\Var[f(Y_{l,1}.Y_{l,2})]$. One way to do this is by using the law of total
variance and conditioning on the mutational configuration at the locus.
\begin{equation*}
  \E[ \Var(f(Y_{l,1},Y_{l,2}) | \mbox{mutation} )] +
  \Var( \E[ f(Y_{l,1}.Y_{l,2}) | \mbox{mutation} ]).
\end{equation*}
The second term here will be zero if the mutational distribution and
transformation $f$ are both symmetric. We'll then assume that at most one
mutation occurs per locus genealogy. This means it is only necessary to
calculate $\Var(f(Y_{l,1}.Y_{l,2})| 1 \mbox{ mutation}))$ and
$\Var(f(Y_{l,1}.Y_{l,2})| 2 \mbox{ mutations}))$.

To proceed one will need to assume a specific form for $f$. A very simple
function is
\begin{equation}
  f(Y_{l,1}.Y_{l,2}) =
  \begin{cases}
    bY_{l,1}, & \text{if } Y_{l,2} = 1\\
    2Y_{l,1}, & \text{if } Y_{l,2} \neq 1.
  \end{cases}
\end{equation}
In this model every mutation has the same degree of dominance regardless of how
big of an effect it has on the trait. More complicated models are possible, but
this one is good so far for building intuition. Under this model,
\begin{equation*}
  \Var(f(Y_{l,1}.Y_{l,2})| 1 \mbox{ mutation})) = b^2m_2,
\end{equation*}
and
\begin{equation*}
  \Var(f(Y_{l,1}.Y_{l,2})| 2 \mbox{ mutations})) = 4m_2.
\end{equation*}

The expected variance at a locus is then
\begin{equation*}
  \Var(f(Y_{l,1}.Y_{l,2})| 1 \mbox{ mutation}))\P(1 \mbox{ mutation}) +
  \Var(f(Y_{l,1}.Y_{l,2})| 2 \mbox{ mutations}))\P(2 \mbox{ mutations}).
\end{equation*}
Which evaluates overall to
\begin{equation}
  2\T \tau_{2,2} b^2 m_2 + \T(T_{MRCA} - \tau_{2,2})4m_2.
\end{equation}
We only need to take the expectation over genealogies at each locus to finish
the derivation
%%% Local Variables:
%%% TeX-master: "notes.tex"
%%% End:
