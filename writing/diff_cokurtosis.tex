Since the mean of a sample is uninformative, we have $n-1$ meaningful
observations in a sample of size $n$. Let $Y_0$ be the reference trait value and
$Y_1-Y_0,Y_2-Y_0,\ldots,y_{n-1}-Y_0$ are the observed trait differences. Let $X$
be the random vector of these differences. We are now interested in the
distribution of these differences. The moment generating function is
\begin{equation}
  \varphi_{\mathbf{X}}(\mathbf{k}) =
  \int e^{\mathbf{k} \cdot \mathbf{x} }
  \int P(\mathbf{X}=\mathbf{x}|\mathbf{T} = \mathbf{t})
  P(\mathbf{T}=\mathbf{t})d\mathbf{t}d\mathbf{x}.
\end{equation}
For the present we're going to look at $Cokurt[X_1,X_1,X_2,X_2]$, the cokurtosis between
two trait differences in a sample. This should hopefully tell us something about
the propensity of extreme trait values to be shared because of shared ancestry
of the individuals. The mgf for this is
\begin{equation}
  \int \int e^{k_1(y_a-y_0) + k_2(y_b-y_0)} P(Y_a-Y_0=y_a-y_0,Y_b-Y_0=y_b-y_0|\mathbf{T}-\mathbf{t})
  P(\mathbf{T}=\mathbf{t})d\mathbf{y}d\mathbf{t}.
\end{equation}
As before we can break this into different sections of the genealogy because the
changes in trait values along these branches are independent conditional on $T$.
Ignoring sets of branches where $Y_a-Y_0=Y_b-Y_0=0$, the relevant branch sets
are $(\Omega_{a/(b,0)},\Omega_{(0+b)/a},\Omega_{b/(0,a)},\Omega_{(0+a)/b},\Omega_{0/(a,b)},\Omega_{(a+b)/0})$.
Only considering these gives the following mgf
\begin{align*}
  \varphi_{Y_a-Y_0,Y_b-Y_0}(k_1,k_2) &=
  \int \prod_{\omega \in \Omega_{a/(0,b)}}\exp\left( \T t_{\omega} (\phi(k_1)-1) \right) \\
  &\times \prod_{\omega \in \Omega_{(0+b)/a}}\exp\left( \T t_{\omega} (\phi(-k_1)-1) \right) \\
  &\times \prod_{\omega \in \Omega_{b/(0,a)}}\exp\left( \T t_{\omega} (\phi(k_2)-1) \right) \\
  &\times \prod_{\omega \in \Omega_{(0+a)/b}}\exp\left( \T t_{\omega} (\phi(-k_2)-1) \right) \\
  &\times \prod_{\omega \in \Omega_{0/(a,b)}}\exp\left( \T t_{\omega} (\phi(-k_1-k_2)-1) \right) \\
  &\times \prod_{\omega \in \Omega_{(a+b)/0}}\exp\left( \T t_{\omega} (\phi(k_1+k_2)-1) \right)
  P(\mathbf{T}=\mathbf{t})d\mathbf{t}
\end{align*}
Application of the low mutation rate approximation and assuming that $m_1$ and
$m_3$ are zero gives
\begin{align*}
  (1 &+ \sum_{\omega \in \Omega_{a/(0,b)}} E[t_\omega] \T 
       \left[ \frac{m_2}{2} k_1^2 + \frac{m_4}{24}k_1^4\right] \\
     &+ \sum_{\omega \in \Omega_{(0+b)/a}} E[t_\omega] \T 
       \left[ \frac{m_2}{2} k_1^2 + \frac{m_4}{24}k_1^4 \right] \\
     &+ \sum_{\omega \in \Omega_{b/(0,a)}} E[t_\omega] \T
       \left[ \frac{m_2}{2} k_2^2 + \frac{m_4}{24}k_2^4 \right] \\
     &+ \sum_{\omega \in \Omega_{(0+a)/b}} E[t_\omega] \T 
       \left[ \frac{m_2}{2} k_2^2 + \frac{m_4}{24}k_2^4 \right] \\
     &+ \sum_{\omega \in \Omega_{0/(a,b)}} E[t_\omega] \T
       \left[ \frac{m_2}{2} (k_1 + k_2)^2 + \frac{m_4}{24}(k_1 + k_2)^4\right] \\
     &+ \sum_{\omega \in \Omega_{(a+b)/0}} E[t_\omega] \T 
       \left[ \frac{m_2}{2} (k_1 + k_2)^2 + \frac{m_4}{24}(k_1 + k_2)^4 \right])^L.
\end{align*}
If we ignore terms that won't contribute to the cokurtosis and do some grouping
we get
\begin{align*}
  (1 &+ \sum_{\omega \in \Omega_{0/a}} E[t_\omega] \T \frac{m_2}{2}k_1^2 \\
     &+ \sum_{\omega \in \Omega_{a/0}} E[t_\omega] \T \frac{m_2}{2}k_1^2 \\
     &+ \sum_{\omega \in \Omega_{0/b}} E[t_\omega] \T \frac{m_2}{2}k_2^2 \\
     &+ \sum_{\omega \in \Omega_{b/0}} E[t_\omega] \T \frac{m_2}{2}k_2^2 \\
     &+ \sum_{\omega \in \Omega_{0/(a,b)}} E[t_\omega] \T 
       \left[ \frac{m_2}{2} 2k_1k_2 + \frac{m_4}{24}k_1^2k_2^2\right]\\
     &+ \sum_{\omega \in \Omega_{(a,b)/0}} E[t_\omega] \T 
       \left[ \frac{m_2}{2} 2k_1k_2 + \frac{m_4}{24}k_1^2k_2^2\right])^L.
\end{align*}
These sums over internal branches can be interpreted in terms of coalescence
times. 
\begin{align*}
(1 &+ 2E[\tau_{a/0}] \T \frac{m_2}{2}k_1^2 \\
   &+ 2E[\tau_{b/0}] \T \frac{m_2}{2}k_2^2 \\
   &+ E[\tau_{0/(a,b)}] \T \left[ \frac{m_2}{2} 2k_1k_2 + \frac{m_4}{24}6k_1^2k_2^2\right]\\
   &+ E[\tau_{(a,b)/0}] \T \left[ \frac{m_2}{2} 2k_1k_2 + \frac{m_4}{24}6k_1^2k_2^2\right])^L.
\end{align*}y
As before $E[\tau_{a/0}]$ is the expected coalescence time between individuals
$a$ and $0$. $E[\tau_{0/(a,b)}]$ is the expected total branch length subtending
individual $0$ before this lineage coalesces with either $a$ or $b$.
$E[\tau_{(a,b)/0}]$ is the expected total branch length subtending both $a$ and
$b$ before either coalesces with $0$. This is a much less common a genealogical
quantity than the expected coalescent time, but it seems to make sense in this
context. Taking the appropriate derivatives we get
\begin{align}
\label{eq:diffcokurt}
  E[Y_a-Y_0,Y_b-Y_0] &= 4L(L-1)E[\tau_{a/0}]E[\tau_{b/0}]\left(\T\right)^2m_2^2 \nonumber \\
                     &+ 2L(L-1)(E[\tau_{0/(a,b)}]+E[\tau_{(a,b)/0}])^2\left( \T \right)^2 m_2^2 \nonumber \\
                     &+ L E[\tau_{0/(a,b)}] \T m_4 + L E[\tau_{(a,b)/0}] \T m_4.
\end{align}
The cokurtosis is then 
\begin{equation}
  \left( 1 - \frac{1}{L} \right) \left( 1 + 
  \frac{(E[\tau_{0/(a,b)}]+E[\tau_{(a,b)/0}])^2}{2E[\tau_{a/0}]E[\tau_{b/0}]} \right)+ 
  \frac{Kurt[M]\left(E[\tau_{0/(a,b)}] +  E[\tau_{(a,b)/0}]\right)}{4L \T E[\tau_{a/0}]E[\tau_{b/0}]}.
\end{equation}
The cokurtosis of a multivariate normal distribution is $1+2\rho$ where $\rho$
is the correlation between the two variables. With this in mind we can
rewrite \eqref{eq:diffcokurt} as
\begin{equation}
  \left( 1 - \frac{1}{L} \right) \left( 1 + 
  2\frac{(E[\tau_{0/(a,b)}]+E[\tau_{(a,b)/0}])^2}{4E[\tau_{a/0}]E[\tau_{b/0}]} \right)+ 
  \frac{Kurt[M]\left(E[\tau_{0/(a,b)}] +  E[\tau_{(a,b)/0}]\right)}{4L \T E[\tau_{a/0}]E[\tau_{b/0}]}.
\end{equation}
Here,
$\frac{(E[\tau_{0/(a,b)}]+E[\tau_{(a,b)/0}])^2}{4E[\tau_{a/0}]E[\tau_{b/0}]}$
makes sense as the correlation between $Y_a-Y_0$ and
$Y_b-Y_0$. \eqref{eq:diffcokurt} therefore has the same form as the other
kurtosis formulas where the excess kurtosis above the normal expectation is
proportional to the mutational kurtosis.
%%% Local Variables:
%%% TeX-master: "notes.tex"
%%% End:
