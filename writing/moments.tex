The main utility of moment generating functions is to calculate moments. We
showed how the distribution of trait values converges to a normal distribution
when there are a large number of sites with small effect per mutation relative
to the rate of mutation. In order to see how much deviation there is from this
normal model we can calculate moments and compare them to the moments of the
multivariate normal distribution.

At first we will do this using the same low mutation rate approximation that was
used for the normal limit. The moment generation function of the trait values can 
be approximated as
\begin{equation}
  \label{eq:mgf_approx_1}
  \varphi_{\mathbf{Y}}(\mathbf{k}) \approx \left[ 1 + \sum_{\omega \in \mathcal{O}}
    E[t_\omega] \T \left( \psi\left( \sum_{a \in \omega} k_a\right) -1 \right) \right]^L.
\end{equation}
This expression ignores any terms of the genealogy moment generation function
shown in equation \ref{eq:mgf_L} above order one. These terms are not written
because they contain products of mutation rates and coalescent times of order
two and greater. Removing these terms means that equation \ref{eq:mgf_approx_1}
is free of terms giving the expectations of products such as $E[\T t_{\omega_1}
  \T t_{\omega_2}]$.
