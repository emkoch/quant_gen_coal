In the derivation of \eqref{eq:m31}, how the second term is arrived at is not
immediately obvious. To see where this comes from, consider the two types of
terms within \eqref{eq_mgf_approx_sum} that will contribute $k_a^3k_b$ from
pairs of branches. These are those pairs where both contain $a$ and $b$, and
those pairs where one contains $a$ and $b$ and the other contains $a$ only.
Pairs with the same branch repeated twice have multiplicity $L(L-1)/2$ and pairs
with two different branches have multiplicity $L(L-1)$. Branch pairs where both
contain $a$ and $b$ have $4k_a^3k_b$ because this term can be made four times
from $(k_a+k_b)^4$, and branch pairs where one contains $a$ and the other $a$
and $b$ have $2k_a^3k_b$ because this term can be made two ways from
$k_a^2(k_a+k_b)^2$. These two sets of pairs are
\begin{equation*}
  \frac{L(L-1)}{2} \left( \T \frac{m_2}{2}\right)^2 2k_a^2\times 2k_ak_b
  \left( \sum_{\omega: a,b \in \omega} E[t_\omega] \right)^2
\end{equation*}
and
\begin{equation*}
  L(L-1) \left( \T \frac{m_2}{2}\right)^2 k_a^2\times 2k_ak_b
  \left( \sum_{\omega: a,b \in \omega} E[t_\omega] \right)\left( \sum_{\omega: a/b \in \omega} E[t_\omega] \right).
\end{equation*}
These two can clearly be added together to yield
\begin{equation*}
  L(L-1) \left( \T \frac{m_2}{2}\right)^2 k_a^2\times 2k_ak_b
  \left( \sum_{\omega: a,b \in \omega} E[t_\omega] \right)\left( \sum_{\omega: a \in \omega} E[t_\omega] \right).
\end{equation*}

The derivation of \eqref{eq:m211} is even more challenging because more
potential branch pairs need to be considered between the three descendants
included in the moment. As before pairs containing the same branch twice have a
multiplicity of $L(L-1)/2$ while pairs containing two different branches have a
multiplicity of $L(L-1)$. Depending on what individuals are on each branch
$k_a^2k_bk_c$ will also have a different coefficient. I'll consider each
possible type of pair in sequence then add them together at the end.

\begin{flushleft}
  \textbf{$a$, $b$, and $c$ are present on all branches}\\
\end{flushleft}
Since all individuals are present on both branches, the coefficient of
$k_a^2k_bk_c$ is multinomial ($12$). The term for this set of pairs is then
\begin{equation*}
  \frac{L(L-1)}{2} 12 k_a^2k_bk_c \left( \sum_{\omega: a,b,c \in \omega} E[t_\omega] \right)^2.
\end{equation*}
\begin{flushleft}
  \textbf{Only $a$ and $b$ are on one branch while $a$, $b$ and $c$ are on the other}\\
\end{flushleft}
In this case $k_a^2k_bk_c$ has coefficient $6$. Since these are non-overlapping
sets of branches we get
\begin{equation*}
  L(L-1) 6 k_a^2k_bk_c \left( \sum_{\omega: a,b/c \in \omega} E[t_\omega] \right)
  \left( \sum_{\omega: a,b,c \in \omega} E[t_\omega] \right).
\end{equation*}
\begin{flushleft}
  \textbf{Only $a$ and $c$ are on one branch while $a$, $b$ and $c$ are on the other}\\
\end{flushleft}
This is the same as above.
\begin{equation*}
  L(L-1) 6 k_a^2k_bk_c \left( \sum_{\omega: a,c/b \in \omega} E[t_\omega] \right)
  \left( \sum_{\omega: a,b,c \in \omega} E[t_\omega] \right).
\end{equation*}
\begin{flushleft}
  \textbf{Only $a$ is present on one branch while $a$, $b$ and $c$ are on the other}\\
\end{flushleft}
In this case $k_a^2k_bk_c$ has coefficient $2$. These branches contain
non-overlapping sets of descendants again so we can write
\begin{equation*}
  L(L-1) 2 k_a^2k_bk_c \left( \sum_{\omega: a/b,c \in \omega} E[t_\omega] \right)
  \left( \sum_{\omega: a,b,c \in \omega} E[t_\omega] \right).
\end{equation*}
\begin{flushleft}
  \textbf{$a$ is present on one branch while only $b$ and $c$ are present on the other}\\
\end{flushleft}
In this case, no matter what other descendants join $a$ on its branch, the coefficient of
$k_a^2k_bk_c$ will be $2$. These branches are also non-overlapping, so we get
\begin{equation*}
  L(L-1) 2 k_a^2k_bk_c \left( \sum_{\omega: a \in \omega} E[t_\omega] \right)
  \left( \sum_{\omega: b,c/a \in \omega} E[t_\omega] \right).
\end{equation*}
\begin{flushleft}
  \textbf{Only $a$ and $b$ are present on one branch while only $a$ and $c$ are
    present on the other.}\\
\end{flushleft}
In this final case the coefficient of $k_a^2k_bk_c$ is $4$ because we are taking
this term from $(k_a+k_b)^2(k_a+k_c)^2$. These are again different branches
always so
\begin{equation*}
  4L(L-1)k_a^2k_bk_c \left( \sum_{\omega: a,b/c \in \omega} E[t_\omega] \right)
  \left( \sum_{\omega: a,c/b \in \omega} E[t_\omega] \right).
\end{equation*}

This can be simplified with a small amount of branch arithmetic. Being very lazy
about notation this is
\begin{align*}
  6\tau_{a+b+c}^2 + 6\tau_{a+b/c}\tau_{a+b+c} + 6\tau_{a+c/b}\tau_{a+b+c} + 2\tau_{a/b+c}\tau_{a+b+c} +
  2\tau_a\tau_{b+c/a} + 4\tau_{a+b/c}\tau_{a+c/b}\\
  = 4\tau_{a+b+c}^2 + 4\tau_{a+b/c}\tau_{a+b+c} + 4\tau_{a+c/b}\tau_{a+b+c} + 2\tau_{a}\tau_{a+b+c} +
  2\tau_a\tau_{a+b} - 2\tau_a\tau_{a+b+c} + 4\tau_{a+b/c}\tau_{a+c/b}\\
  = 2\tau_a\tau_{a+b} + 4\tau_{a+b+c}^2 + 8\tau_{a+b/c}\tau_{a+b+c} + 4\tau_{a+b/c}^2\\
  = 2\tau_a\tau_{a+b} + 4(\tau_{a+b+c} + \tau_{a+b/c})^2\\
  = 2\tau_a\tau_{a+b} + 4\tau_{a+b}^2.
\end{align*}
This then gives us the same result as \eqref{eq:m211}.
%%% mode: latex
%%% TeX-master: "notes.tex"
%%% End: 
