We have investigated the fact that, when the number of loci expected to
experience mutations affecting a trait is low, the kurtosis is higher than that
for a normal distribution. This kurtosis is over evolutionary realizations. If
we were to replay evolution then individuals would have more extreme values
relative to the variance of the distribution than under a normal distribution.
It is worth asking how this affects selection. Even though the models so far
have been strictly neutral, we can imagine selection acting on the population
and producing a change in the mean an variance of trait values. The classical
theory of quantitative genetics assumes a normal distribution of breeding values
in the population. In one of a series of papers, \citet{Turelli1990} extended
the theory of selection on a polygenic character to a more realistic model of
multilocus population genetics. One conclusion of this work is that selection
can cause deviations from normality in higher order moments of the distribution
of breeding values, affecting the progress of selection.

A central result of \citet{Turelli1990} is that the change in the mean phenotype
due to one generation of selection is

\begin{equation}
  \label{eq:dz}
  \Delta \bar{Z} = V_gL_1 + M_{3,g}L_2 + \gamma_4V^2_gL_3 +
  \left( M_{5,g}-4M_{3,g}V_g\right)L_4 + \ldots.
\end{equation}

Here, $Z$ refers to a phenotypic value which has an environmental component as
opposed the breeding value $Y$ which is due entirely to genetics. $V_g$ is the
variance in breeding values in the population and $\gamma_4$ is the excess
kurtosis above a normal distribution. The terms $M_{i,g}$ are the $i^{th}$
central moments of the breeding value distribution in the population. The terms
$L_i$ describe the effect of selection on the breeding values. These are the
selection gradients in terms of the genotypic moments.

\begin{equation}
  \label{eq:Li}
  L_i = \frac{\partial \ln(\bar{w})}{\partial M_{i,g}}
\end{equation}

and

\begin{equation}
  \label{eq:L1}
  L_1 = \frac{\partial \ln(\bar{w})}{\partial \bar{Y}}.
\end{equation}

What we can tell immediately from these equations is that the importance of
higher order moments of the breeding value distribution depends on the specific
fitness function. For instances, the effect of the skew of the breeding value
distribution $M_{3,g}$ depends on the effect on the mean fitness of changing the
variance of the breeding values $\frac{\partial \ln(\bar{w})}{V_g}$. Each of the
$L_i$ terms correspond to the effects on mean fitness of changing one moment
while holding the others constant. Therefore, whether or not the excess kurtosis
that arises due to a sparse trait architecture has a meaningful effect on the
response to selection depends on the precise shape of that selection.
\citet{Turelli1990} use trick to calculate the $L_i$. By taking a Taylor series
expansion of $w_g(Y)$ they get

\begin{equation}
  \label{eq:wbar}
  \bar{w} = w_g(\bar{Z}) + \sum_{i=2}^\infty \frac{M_{i,g}w^{(i)}_g(\bar{Z})}{i!}.
\end{equation}

Differentiating this gives

\begin{equation}
  \label{eq:l1}
  L_1= \frac{w_g^{(1)}(\bar{Z)}}{\bar{w}} + \sum_{i=2}^\infty \frac{ M_{i,g}w^{(i+1)}( \bar{Z} ) }{ i! \bar{w} },
\end{equation}

and

\begin{equation}
  \label{eq:li}
  L_i = \frac{w_g^{(i)}(\bar{Z})}{i!\bar{w}}.
\end{equation}

What this shows is that it is sufficient when calculating $L_i$ to calculate the
$i^{th}$ derivative of $w_g$ and evaluate this at the population mean.

%%% Local Variables:
%%% TeX-master: "notes.tex"
%%% End:
