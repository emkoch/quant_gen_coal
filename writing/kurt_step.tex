To get a sense for the effect that demography can have on the expected
population kurtosis we have to consider a particular population size history. In
general, finding expressions for expected coalescent times beyond a sample size
of two are quite difficult under nonequilibrium population histories. A
relatively simple case is that of a single step change in population size. Let
$N_0$ be the current effective population size, $z$ be the number of generations
in the past when the population size changes, and $N_1$ be the size it changes
to. A somewhat nicer parameterization uses $b=\frac{z}{N_0}$ and
$c=\frac{N_1}{N_0}$. The expected times to the most recent common ancestor for
samples of size two, three, and four are
\begin{align}
  \E[T_2] &= N_0\left( 1-e^{-b} + ce^{-b} \right) \nonumber \\
  \E[T_3] &= N_0\left( \frac{1}{6} e^{-3b}(1-c) - \frac{3}{2}e^{-b}(1-c) + \frac{4}{3} \right) \nonumber \\
  \E[T_4] &= N_0\left( \frac{-1}{30}e^{-6b} + \frac{1}{3}e^{-3b} - \frac{9}{5}e^{-b} + \frac{3}{2}
  + c\left( \frac{1}{30}e^{-6b} - \frac{1}{3}e^{-3b} +\frac{9}{5}e^{-b} \right)\right). \nonumber \\
\end{align}

Using these we can see that the scaling factor for the kurtosis due to
demography is
\begin{equation}
  Q = \frac{ 4\E[T_2] - 6\E[T_3] + 3\E[T_4]}{\E[T_2]} =
    \frac{\left( e^{2b} - e^{-2b} \right)(1-c)}{2\left(e^{b} -1 + c\right)}.
\end{equation}
\begin{figure}
  \label{fig:kurtland}
  \centering
  \includegraphics[width=0.9\textwidth]{kurt_land.pdf}
  \caption{The expected population kurtosis under different step population size
    changes. Red lines show parameter combinations where the expected number of
    mutations is constant. Green lines show parameter combinations where the
    expected kurtosis  stays constant.}
\end{figure}
Figure \ref{fig:kurtland} shows how the expected population excess kurtosis
depends on the parameters $b$ and $c$ of the population size change. Of course,
part of the change in the kurtosis is due to populations with a deeper mean
pairwise coalescent time having a lower expected kurtosis, while part is due to
the factor $Q$ that depends on the demography. The difference between the red
and green lines in this figure is due to the effects of demography aside from
the mean pairwise coalescent time. 
%%% mode: latex
%%% TeX-master: "notes.tex"
%%% End: 
